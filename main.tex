\documentclass[a4paper,12pt,reqno]{amsart}
%\usepackage{times} %,fullpage}%,backref}
\usepackage{latexsym,amscd,amssymb,url}
\pagestyle{headings}
\usepackage{xcolor, mathtools}
\definecolor{dblue}{rgb}{0,0,.6}
\usepackage[colorlinks=true, linkcolor=dblue, citecolor=dblue, filecolor = dblue, menucolor = dblue, urlcolor = dblue]{hyperref}

\textwidth=450pt 
\oddsidemargin=12pt
\evensidemargin=12pt
\renewcommand{\baselinestretch}{1.2}
\setlength{\footskip}{25pt}
%
%\setlength{\headheight}{6.15pt}
%\setlength{\headsep}{0.5cm}
%\setlength{\headheight}{6.15pt}
%\setlength{\headsep}{1.0cm}
%\usepackage[latin1]{inputenc} 	
%\usepackage[T1]{fontenc}         
%\usepackage{ae}									
%\usepackage[arrow, matrix, curve]{xy}  
\usepackage[all,cmtip]{xy}  
\usepackage{graphicx}
%\usepackage{MnSymbol} 
\usepackage{color}
\usepackage{hyperref}
\usepackage{enumerate}

	
\usepackage[T1]{fontenc}         
\usepackage{ae}									
%\usepackage[arrow, matrix, curve]{xy}  
\usepackage[all,cmtip]{xy}  
\usepackage{graphicx}
%\usepackage{MnSymbol} 
\usepackage{color}

\interfootnotelinepenalty=10000


%----------------------------------------------
% Theorem like environments
%
\newtheorem{theorem}{Theorem}[section]
\newtheorem*{theoremnocount}{Theorem}
\theoremstyle{plain}
\newtheorem*{acknowledgement}{Acknowledgement}
\newtheorem{question}[theorem]{Question}
\newtheorem{statement}{Statement}
\newtheorem{case}{Case}
\newtheorem{claim}{Claim}
\newtheorem{conclusion}{Conclusion}
\newtheorem{condition}{Condition}
\newtheorem{conjecture}[theorem]{Conjecture}
\newtheorem{corollary}[theorem]{Corollary}
\newtheorem{criterion}{Criterion}
\newtheorem{definition}[theorem]{Definition}
\newtheorem{example}[theorem]{Example}
\newtheorem{exercise}{Exercise}
\newtheorem{lemma}[theorem]{Lemma}
\newtheorem{notation}{Notation}
\newtheorem{problem}{Problem}
\newtheorem{proposition}[theorem]{Proposition}
\theoremstyle{remark}
\newtheorem{remark}[theorem]{Remark}
\newtheorem{solution}{Solution}
\newtheorem{fact}{Fact}
\newtheorem*{convention}{Convention}

\setcounter{tocdepth}{1}

%---------------------------------------------------
%newcommads
\newcommand{\delbar}{\overline{\partial}}
\newcommand{\del}{\partial}
\newcommand{\Z}{\mathbb Z}
\newcommand{\Q}{\mathbb Q}
\newcommand{\PP}{\mathbb P}
\newcommand{\Qbar}{\overline{\mathbb Q}}
\newcommand{\C}{\mathbb C}
\newcommand{\N}{\mathbb N}
\newcommand{\R}{\mathbb R}
\newcommand{\HH}{\mathbb H}
\newcommand{\CP}{\mathbb P}
\newcommand{\OO}{\mathcal O}
\newcommand{\K}{\mathcal F}
\newcommand{\D}{\operatorname{D}}
\newcommand{\sV}{\mathcal V}



\newcommand{\Card}{\operatorname{Card}}
\newcommand{\alb}{\operatorname{alb}}
\newcommand{\im}{\operatorname{im}}
\newcommand{\Td}{\operatorname{Td}}
\newcommand{\Hom}{\operatorname{Hom}}
\newcommand{\Pic}{\operatorname{Pic}}
\newcommand{\Div}{\operatorname{Div}}
\newcommand{\End}{\operatorname{End}}
\newcommand{\rk}{\operatorname{rk}}
\newcommand{\Fix}{\operatorname{Fix}}
\newcommand{\Aut}{\operatorname{Aut}}
\newcommand{\sign}{\operatorname{sign}}
\newcommand{\Sym}{\operatorname{Sym}}
\newcommand{\id}{\operatorname{id}}
\newcommand{\PGL}{\operatorname{PGL}}
\newcommand{\GL}{\operatorname{GL}}
\newcommand{\Spec}{\operatorname{Spec}}
\newcommand{\Eig}{\operatorname{Eig}}
\newcommand{\RE}{\operatorname{Re}}
\newcommand{\Gal}{\operatorname{Gal}}
\newcommand{\res}{\operatorname{res}}
\newcommand{\pr}{\operatorname{pr}}
\newcommand{\NS}{\operatorname{NS}}
\newcommand{\Sing}{\operatorname{Sing}}
\newcommand{\Alb}{\operatorname{Alb}}
\newcommand{\PD}{\operatorname{PD}}
\newcommand{\SING}{\operatorname{Sing}}
\newcommand{\TD}{\operatorname{TD}}
\newcommand{\T}{\mathcal V}
\newcommand{\Bir}{\operatorname{Bir}}
\newcommand{\Amp}{\operatorname{Amp}}
\newcommand{\codim}{\operatorname{codim}}
\newcommand{\rank}{\operatorname{rank}}
\newcommand{\Nef}{\operatorname{Nef}}
\newcommand{\NE}{\operatorname{NE}}
\newcommand{\None}{\operatorname{N}_1}
\newcommand{\Ext}{\operatorname{Ext}}
\newcommand{\can}{\operatorname{can}}
\newcommand{\var}{\operatorname{var}}
\newcommand{\rat}{\operatorname{rat}}
\newcommand{\Per}{\operatorname{Per}}
\newcommand{\Br}{\operatorname{Br}}
\newcommand{\Gr}{\operatorname{Gr}}
\newcommand{\CH}{\operatorname{CH}}
\newcommand{\supp}{\operatorname{supp}}
\newcommand{\tor}{\operatorname{tors}} 
\newcommand{\red}{\operatorname{red}} 
\newcommand{\Proj}{\operatorname{Proj}} 
\newcommand{\sing}{\operatorname{sing}} 
\newcommand{\discr}{\operatorname{discr}} 
\newcommand{\cl}{\operatorname{cl}} 
\newcommand{\ram}{\operatorname{ram}} 
\newcommand{\Frac}{\operatorname{Frac}}
%\newcommand{\char}{\operatorname{char}}
\newcommand{\Val}{\operatorname{Val}}
\newcommand{\F}{\mathbb F}
\newcommand{\Char}{\operatorname{Char}}
\newcommand{\coker}{\operatorname{coker}}
\newcommand{\Loc}{\operatorname{Loc}}
\newcommand{\length}{\operatorname{length}}
\newcommand{\Jac}{\operatorname{Jac}}

 
\newcommand{\dashedlongrightarrow}{\xymatrix@1@=15pt{\ar@{-->}[r]&}}
\renewcommand{\longrightarrow}{\xymatrix@1@=15pt{\ar[r]&}}
\renewcommand{\mapsto}{\xymatrix@1@=15pt{\ar@{|->}[r]&}}
\renewcommand{\twoheadrightarrow}{\xymatrix@1@=15pt{\ar@{->>}[r]&}}
\newcommand{\hooklongrightarrow}{\xymatrix@1@=15pt{\ar@{^(->}[r]&}}
\newcommand{\congpf}{\xymatrix@1@=15pt{\ar[r]^-\sim&}}
\renewcommand{\cong}{\simeq}
%--------------------------------------------------------


%-----------
%-----------Yagna's typsets--------
\usepackage{tikz-cd}
\usepackage{enumitem}

\PassOptionsToPackage{pdfusetitle,pagebackref,colorlinks}{hyperref}
\usepackage{bookmark}
\hypersetup{
  linkcolor={red!70!black},
  citecolor={green!70!black},
  urlcolor={blue!80!black}
}
\newtheorem{alphtheorem}{Theorem}
\renewcommand{\thealphtheorem}{\Alph{alphtheorem}}

%Mathcal Letters =====================
\newcommand{\sA}{\mathcal{A}}
\newcommand{\sB}{\mathcal{B}}
\newcommand{\sD}{\mathcal{D}}
\newcommand{\sF}{\mathcal{F}}
\newcommand{\sG}{\mathcal{G}}
\newcommand{\sH}{\mathcal{H}}
\newcommand{\sK}{\mathcal{K}}
\newcommand{\sL}{\mathcal{L}}
\newcommand{\sM}{\mathcal{M}}
\newcommand{\sN}{\mathcal{N}}
\newcommand{\sO}{\mathcal{O}}
\newcommand{\sP}{\mathcal{P}}
\newcommand{\sQ}{\mathcal{Q}}
\newcommand{\sT}{\mathcal{T}}

%mathbb Letters======
\newcommand{\bbA}{\mathbb{A}}
\newcommand{\bbB}{\mathbb{B}}
\newcommand{\bbC}{\mathbb{C}}
\newcommand{\bbG}{\mathbb{G}}
\newcommand{\bbH}{\mathbb{H}}
\newcommand{\bbK}{\mathbb{K}}
\newcommand{\bbL}{\mathbb{L}}
\newcommand{\bbM}{\mathbb{M}}
\newcommand{\bbN}{\mathbb{N}}
\newcommand{\bbP}{\mathbb{P}}
\newcommand{\bbQ}{\mathbb{Q}}
\newcommand{\bbR}{\mathbb{R}}
\newcommand{\bbV}{\mathbb{V}}
\newcommand{\bbZ}{\mathbb{Z}}



%Script Letters ======================
\newcommand{\frf}{\mathfrak{f}}
\newcommand{\frM}{\mathfrak{M}}

\newcommand{\scrL}{\mathscr{L}}
\newcommand{\crI}{\mathscr{I}}
\newcommand{\scrK}{\mathscr{K}}
\newcommand{\scrB}{\mathscr{B}}
\newcommand{\scrC}{\mathscr{C}}
\newcommand{\scrD}{\mathscr{D}}
\newcommand{\scrE}{\mathscr{E}}
\newcommand{\scrI}{\mathscr{I}}
\newcommand{\scrQ}{\mathscr{Q}}
\newcommand{\scrR}{\mathscr{R}}
\newcommand{\scrX}{\mathscr{X}}
\newcommand{\scrY}{\mathscr{Y}}
\newcommand{\scrF}{\mathscr{F}}
\newcommand{\Dred}{\lceil D\rceil}

%Arrow Style =========================
\newcommand{\into}{\hookrightarrow}
\newcommand{\onto}{\rightarrow\hspace*{-.14in}\rightarrow}

%Math Operators ======================



\DeclareMathOperator{\alg}{alg}
\DeclareMathOperator{\BM}{BM}
\DeclareMathOperator{\Bs}{Bs}
\DeclareMathOperator{\Bsp}{\mathbf{B}_+}
\DeclareMathOperator{\SB}{\mathbf{B}}

\DeclareMathOperator{\coh}{coh}
\DeclareMathOperator{\Coker}{Coker}
 \renewcommand{\div}{\text{div}}

\DeclareMathOperator{\DR}{DR}
\DeclareMathOperator{\discrep}{discrep}
\DeclareMathOperator{\exc}{exc}

\DeclareMathOperator{\free}{free}


\DeclareMathOperator{\HHom}{\mathcal{H}\!\mathit{om}}
\DeclareMathOperator{\image}{Im}


\DeclareMathOperator{\Ind}{Ind}
\DeclareMathOperator{\Ker}{Ker}
\DeclareMathOperator{\Lie}{Lie}
\DeclareMathOperator{\op}{op}


\DeclareMathOperator{\qcoh}{qcoh}

\DeclareMathOperator{\reg}{reg}

\DeclareMathOperator{\RHHom}{\mathbf{R}\mathcal{H}\!\mathit{om}}


\DeclareMathOperator{\Spf}{Spf}
\DeclareMathOperator{\Supp}{Supp}
\DeclareMathOperator{\tors}{tors}
\DeclareMathOperator{\torsion}{torsion}
\DeclareMathOperator{\Var}{Var}
%\DeclareMathOperator{\Sym}{Sym}


\newcommand{\Ab}{\mathbf{Ab}}
\newcommand{\Aff}{\mathbf{Aff}}
\newcommand{\tB}{\tilde{B}}
\newcommand{\et}{{\acute{e}t}}

\newcommand{\Sets}{\mathbf{Sets}}
\newcommand{\cX}{\bar{X}}
\newcommand{\cP}{\bar{P}}
\newcommand{\cV}{\bar{V}}
\newcommand{\cW}{\bar{W}}
\newcommand{\sorry}[1]{\textcolor{red}{#1}}
\newcommand{\dual}[1]{\sD^{\Omega}_{M^{#1}}}


%----------------



\begin{document}  
\title[On linearity of holomorphic 1-forms with zeros]{On linearity of holomorphic 1-forms with zeros} 

\author{Yajnaseni Dutta}

%\address{}
%\email{}

\author{Feng Hao}

%\address{}
%\email{}

\author{Yongqiang Liu}

%\date{\today}
%\date{October 10, 2017; 
%\copyright{\ Stefan Schreieder 2017}}
%\subjclass[2010]{primary 14E08, 14M20; secondary 14J35, 14D06} 
%
% 14J45 Fano varieties
% 14J10 Families, moduli, classification: algebraic theory
% 	14J35   	$4$-folds
% 14M20   	Rational and unirational varieties
% 	14M22   	Rationally connected varieties
% 14D06   	Fibrations, degenerations
%  	14E08   	Rationality questions

%\keywords{rationality problem, stable rationality, decomposition of the diagonal, unramified cohomology, Brauer group, L\"uroth problem.} 



\begin{abstract} 
The goal of this article is two-fold, first we 
\end{abstract}

\maketitle

\section{Introduction}

In this article we study the set of global holomorphic 1-forms  on smooth projective varieties that admit zeros. Interestingly, it has been
indicated by plethora of results (\cite{HK05, LZ05, EL,
SS19, SH19, PS14} to name just a few) that a lot of the geometry
and topology of the variety depends on vanishing of such forms.
We first state our result and then put it in the context of 
the existing vast literature.
\begin{alphtheorem}\label{thm:smooth}
Let $f:X\to A$ be a morphism from a smooth projective variety to a simple abelian variety $A$. Then $f$ is smooth if and only if there is a holomorphic 1-form $\omega$ on $A$ such that $f^*\omega$ is nowhere vanishing. 
\end{alphtheorem}
\sorry{1. subject to change if Tischler can be proved. 2. Add the remark about Tischler for simple albanese case}
In fact we prove a bit more; When $A$ is not necessarily simple
existence of a non-vanishing global holomorphic 1-form
is equivalent to certain restrictions on the so called \emph{higher discriminants} (see \ref{def:higherdiscriminants}) of the morphism $f$ in the sense
of Migliorini and Shende \cite{MiSh18}.
See Theorem \ref{thm:nonvanishing} for more details.
\begin{remark}[Previous results]
\begin{enumerate}
\item \label{item:ps} When $X$ is of general type, it was conjectured in
	\cite{HK05, LZ05} and was proved
	by Popa and Schnell \cite{PS14} every global holomorphic 1-		
	form vanishes on $X$. 
\item \label{item:hk} On the opposite extreme for any $A$, not necessarily simple when $f$ is singular along a divisor of general
		type in $A$, Hacon and Kov\'acs \cite[Proposition 3.5.]{HK05} show that
		$f^*\omega$ always admits zero.
\item The more general version of Theorem \ref{thm:smooth}
recovers both Item (\ref{item:ps}) (see Corollary \ref{cor:ps})
and Item (\ref{item:hk}) (see Corollary \ref{cor:hk}).
\item A special case of a result of 
Schreider \cite[Corollary 1.5]{SS19} shows for any $A$, not necessarily simple when $X$
admits a nowhere vanishing holomorphic 1-form of the form
$f^*\omega$, then $f(X)$ is fibered by tori. In 
Theorem \ref{thm:nonvanishing} we show more generally that 
existence of non-vanishing forms is in fact equivalent to existence of a higher discriminant $Z\subseteq f(X)$ fibered by tori. 
\end{enumerate}
\end{remark}

In a different direction we deal with the question
of linearity of the set of vanishing 1-forms. This is
inspired by the work of  \cite{CL73}, Carrell and Lieberman, who show that for a smooth complex projective variety $X$, the set of holomorphic tangent vector fields with zeros is a subvector space of $H^0(X, T_X)$. On the
other hand its incarnation for 1-form is far from complete.
It is known due to the theory of cohomology jump
loci that only the forms satisfying certain resonance property (see Definition \ref{def:res} below) admitting zeros form
a finite union of subvector spaces inside $H^0(X, \Omega_X^1)$.
Henceforth we call finite unions of subvector spaces as \emph{linear subvarieties}.
In this article we complete the picture. To this end we need the 
following notations.
Let $X$ be a smooth complex projective variety of $\dim X=n$. The zero set of a global holomorphic 1-form $\omega$
is given by 
\[Z(\omega) \coloneqq \{x\in X| \omega(T_xX) = 0\}\]
where $T_xX$ is the holomorphic tangent space of $X$ at $x$. 
We further denote by 
\[V^i(X):=\{ \omega\in H^0(X, \Omega_X^1) |\ \codim Z(\omega)\leq i\},\] when $i=n$ we simply use the notation $V(X)$.
%:=V^n(X)=\{\omega\in H^0(X, \Omega^1_{X}) | \ Z(\omega)\not =\emptyset\},$$  
Furthermore, for the morphism $f:X\to A$ from $X$ to an abelian variety $A$, we denote
\[V^i(f):=\{ \omega\in f^*H^0(A, \Omega_A^1) |\ \codim Z(\omega)\leq i\},\]
As before when $i=n$ we use the notation $V(f)$.
%:=V^n(f)=\{\omega\in f^*H^0(A, \Omega_A^1) | \ Z(\omega)\not =\emptyset\},$$

%and  the holomorphic resonance varieties $$R_h^i(X):=\{\omega\in H^0(X, \Omega_X^1) | \ \dim H^i(H^*(X, \C), \wedge \omega)\not=0\}.$$ Mainly we will study the linearity of $V(X)$ or $V(f)$, i.e., the following problem.

%For 1-forms, 
%by cohomology jump loci theory, we know that the set $R^i_h(X)$ of holomorphic resonant 1-forms (see the following notation) form linear subvarieties. 
\begin{alphtheorem}\label{main1}
Let $f: X\to A$ be a morphism from a smooth complex projective variety $X$ to an abelian variety $A$. Then 
$V(f)$ is a linear subvariety of $H^0(X, \Omega_X^1)$. In particular, $V(X)$ is a linear subvariety. Moreover, each irreducible components of $V(f)$ are $\Q$-sub Hodge structures of $H^0(X, \Omega_X^1)$. In particular, $V(f)$ and $V(X)$ are defined over $\Q$.

%(2) If the discriminant locus of $f$ is not fibred by tori or of dimension 0, then $V(f)=f^*H^0(A, \Omega_A^1)$, i.e., every holomorphic 1-forms pullback from $A$ via $f$ has zeros.
\end{alphtheorem}
\begin{remark}
\begin{enumerate}
	\item When $X$ is of general type, the 
	result of Popa and Schnell (see \ref{cor:ps}) 
	it follows that $V(X) = H^0(X,\Omega_X^1)$ which is of course
	linear. 
	\item  When $\dim X\leq 3$ an immediate corollary of the main theorems in \cite{SS19} and \cite{HS19} is that $V(X)$ is linear.
	\item \label{item:resonance} The \emph{holomorphic resonance varieties} are roughly speaking, the ``topological parts'' $R_h^i(X)$ of $V(X)$.
	More precisely they are defined as follows
	\[R_h^i(X):=\{\omega\in H^0(X, \Omega_X^1)|\ H^i(H^*(X, \mathbb{C}), \wedge \omega)\not= 0\}\subseteq V(X),\]
	is linear (see, e.g., \cite{DiPa13}). Note that
	when
	$R_h^i(X) = V(X)$, our 	theorem is a direct consequence of
	the linearity of the resonance variety. This 
	for instance is the case when $\chi(X)\not=0$ (see Corollary \ref{ECNT}). 
\end{enumerate}

\end{remark}
The equality $R_h^i(X) = V(X)$ in Item (\ref{item:resonance}) may not always be true. An example of Debarre, Jiang and Lahoz 
\cite[Example 1.11]{DJL17} shows that the
there exists a bi-elliptic surface $S$ admitting a 1-form 
$\omega$ for which $(H^*(X, \mathbb{C}), \wedge \omega)$
is exact, yet $\omega$ admits zeros on $S$. 

More generally we expect the following to be true.
\begin{conjecture} \label{linear-vi}
Let $f: X\to A$ be a morphism from a smooth complex projective variety $X$ to an abelian variety $A$. Then $V^i(f)$ are linear subvarieties of $H^0(X, \Omega_X^1)$, i.e., $V^i(X)$ is a finite union of linear subspaces of $H^0(X, \Omega_X^1)$ for each nonnegative integer $i$. In particular $V^i(X)$ are linear.
\end{conjecture}

To this end, first using a result of Spurr \cite{Sp88}, we directly get $V^1(X)$ is linear.

\begin{alphtheorem}
Let $f: X\to A$ be a morphism from a smooth complex projective variety $X$ to an abelian variety $A$. Then $V^1(f)$ is linear. In particular, $V^1(X)$ is linear.
\end{alphtheorem}


 %What's more, we show that $V^1(f)$ consists the resonant holomorphic 1-forms vanishing along a divisor and ``negative'' 1-forms (holomorphic 1-forms, which is not  necessarily resonant, but vanishes along a rigid divisor). Also, in this article, 
We also generalize a result of Spurr \cite{Sp88} to quasi-projective varieties. 

\begin{alphtheorem} \label{main3}
Let $(X, D)$ be a pair with $X$ a complex smooth projective variety and $D$ a simple normal crossing divisor of $X$. Let $H$ be a very ample divisor on $X$. If $(X, D)$ carries a holomorphic log 1-form $0\not=\omega\in H^0(X, \Omega_X^1(\log D))$  which pullbacks to zero on an effective divisor $E$ with $E^2\cdot H^{n-2}\geq0$, then there is a morphism $f: X-D\to C$ to a smooth quasi-projective curve $C=\bar{C}-B$ (where $\bar{C}$ is the completion of $C$ and $B$ can be empty) with 

(1)  $\omega=f^*\eta$ for some $\eta\in H^0(X, \Omega_{\bar{C}}^1(\log B))$.

(2) $E$ is contained in the fiber of $f$ and $E^2\cdot H^{n-2}=0$.
\end{alphtheorem}

As a corollary we prove the linearity of the set of logarithmic holomorphic 1-forms admitting codimension one zeros.

\begin{corollary}
Let $(X, D)$ be a pair with $X$ a complex smooth projective variety and $D$ a simple normal crossing divisor of $X$. Then the set $$ V^1(X,D):=\{ w \in W(U) \mid \codim_X Z(w) \leq i \}$$ is linear.
\end{corollary}

We show that $V^1(X,D)$ consists of resonant holomorphic logarithmic 1-forms vanishing along a divisor, holomorphic logarthmic 1-forms vanishing along a rigid divisor, and holomorphic logarithmic 1-forms vanishing along some components of the boundary divisor $D$.




 
\subsection{Comments about the proof}
For a morphism $f: X\to A$,  we consider the following diagram 
\begin{center}
\begin{tikzcd}
	X\times f^*H^0(A, \Omega_A^1) \ar[r, "df"] \ar[d, "f\times \id"] & T^*X\\
	A\times H^0(A, \Omega_A^1)\simeq T^*A\dar["\pr_2"]\\
	H^0(A, \Omega_A^1)
\end{tikzcd}
\end{center}
 where $T^*X$ denotes the $2n$ dimensional 
cotangent bundle of $X$, $df$ is the usual differential, and $\tilde{f} = pr_2\circ (f\times \id)$ is the natural projection. Note that
$V(f) \simeq \tilde{f}(df^{-1}(0))$. 
The key ingredient of 
Theorem \ref{thm:smooth} and \ref{main1} is
a result of Migliorini and Shende \cite[Theorem C]{MiSh18} which ensures that $(f\times\id)(df^{-1}(0))$
is a finite union of closures of conormal bundles
$T^*_ZX$ along various subvarieties of $Z\subset A$.
Then the linearity follows from the following proposition
\begin{proposition}\label{van-nonsimple}
Let $A$ be an abelian variety and $X$ be a proper subvariety of $A$. Then the following are equivalent
\begin{enumerate}
	\item $X$ is not fibred by tori or of dimension 0. 
	\item Any holomorphic 1-form $\omega\in H^0(A, \Omega_A^1)$ restricted to $X^{\reg}$, i.e.\ $\omega|_{X_{\textnormal{reg}}}$ admits zeros on the smooth locus $X_{\textnormal{reg}}$.
\end{enumerate}
\end{proposition}
When $A$ is a
simple abelian variety this proposition says that any 1-form restricted to $X$ admits zeros on the regular locus of $X$. A proof of this statement seems to be abundantly found in the literature
see for instance \cite[Proposition 3.1]{HK05}
or \cite[Proposition 5.12]{LMW20}. Our contribution
is to provide a proof in general (see \ref{proof:van-nonsimple}).

\sorry{words about the proof of second part}


%In fact Theorem\ref{main1} (2) implies  that for a variety of general type $X$, every holomorphic 1-form over $X$ admits zeros. This is proved in \cite{PS14}. Also, Theorem \ref{main1} (2) implies \cite[Proposition 3.4]{HK05}, which dealt with the codimension one case. As a corollary of our main theorem we have






\subsection*{Notation}

































\subsection*{Acknowledgement} 

Stefan, Jiang, Kovacs, Nero, Sabbah

\section{Preliminary}



\begin{definition}
For $\omega\in H^0(X, \Omega_X^1)$, the zero set of $\omega$ is the algebraic set of closed point $x$ in $X$, such that $\omega(x)=0$. The zero scheme of $\omega$ is defined by the ideal sheaf $\mathcal{I}_{\omega}$, where  $\mathcal{I}_{\omega}$ is the image of the morphism $$\mathcal{T}_X\overset{\langle\omega, \cdot\rangle}{\longrightarrow} \mathcal{O}_X$$ with $\mathcal{T}_X$ being the tangent sheaf of $X$ and $\langle\omega, \cdot\rangle$ being the pairing of tangent field with 1-form $\omega$.
\end{definition}


\begin{definition}
Let $X$ be a smooth projective variety. We call a holomorphic 1-form $\omega\in H^0(X, \Omega_X^1)$ nonresonant if the zero scheme $Z(\omega)$ is nonempty and the complex $$\ldots\to H^{i-1}(X,\C)\overset{\wedge\omega}{\longrightarrow}H^{i}(X,\C)\overset{\wedge\omega}{\longrightarrow}H^{i+1}(X,\C)\to\ldots$$ is exact. A holomorphic 1-form $\omega\in H^0(X, \Omega_X^1)$ is called universally nonresonant if the zero scheme $Z(\omega)$ is nonempty and the complex $$\ldots\to H^{i-1}(X',\C)\overset{\wedge\tau^*\omega}{\longrightarrow}H^{i}(X',\C)\overset{\wedge\tau^*\omega}{\longrightarrow}H^{i+1}(X',\C)\to\ldots$$ is exact for any \'etale over $\tau: X'\to X$.
\end{definition}

\begin{remark}

(1) (Universally) nonresonant  holomorphic 1-forms were first studied in \cite{SS19}.

(2) In general by \cite[Proposition 3.4]{GL87}, if a holomorphic $\omega$ has zero scheme of codimension greater than or equal to $k$, then the complex $$\ldots\to H^{i}(X,\C)\overset{\wedge\omega}{\rightarrow}H^{i}(X,\C)\overset{\wedge\omega}{\rightarrow}H^{i}(X,\C)\to\ldots$$ is exact whenever $i<k.$ 
\end{remark}

The set of resonant 1-forms is very well understood {\color{red} (Provide references as much as we can)}. We recall the following theorem {\color{red} (Due to Voisin?)}. For $i=1$, this is due to Dimca, Papadima, Suciu. For higher $i$, this is due to Dimca and Papadima:

\begin{theorem}{\cite[Theorem C, Corollary 1.7]{DiPa13}}\label{resonant}
Let $X$ be a compact K\"ahler manifold. Each irreducible component of resonance varieties $R^i(X)$ is a linear subspace of $H^1(X, \C)$ defined over $\Q$, i.e., $$R^i(X)=(R^i(X)\cap H^1(X, \Q))\otimes\C.$$ Also,  each irreducible component of $R^i(X)$ carries $\Q$-Hodge substructures of $H^1(X, \C)$.
\end{theorem} 

\begin{corollary}
Let $f:X\to A$ be a morphism from smooth projective variety $X$ to a simple abelian  variety $A$. Then $R^i(X)\cap f^*H(A, \Omega_A^1)$ is $f^*H(A, \Omega_A^1)$ or $0$.
\end{corollary}

\begin{proof}
$R^i(X)\cap f^*H(A, \Omega_A^1)$ is a $\Q$-Hodge substructure of $H^1(X, \C)$ of the simple Hodge structure $f^*H(A, \Omega_A^1)$.
\end{proof}

\begin{corollary}\label{ECNT}
For $X$ being a smooth projective variety $X$ such that the topological Euler characteristic class $\chi(X)\not=0$, $V(X)$ is a linear subvariety.
\end{corollary}


%\section{Linearity of holomorphic 1-forms with zeros}
%
%In this section, we consider the holomorphic 1-forms with zero locus of arbitrary dimension.




\subsection{Stratification of morphisms after Migliorini and Shende}

For a morphism $f: X\to A$ as in Theorem \ref{main1},  we consider the following diagram 
%$$\xymatrix{
%T^*X 
%& X\times_Y T^*Y \ar[l]_-{df} \ar[r]^-{\tilde{f}} &T^*Y
%},$$
\begin{center}
\begin{tikzcd}
	X\times f^*H^0(A, \Omega_A^1) \ar[r, "df"] \ar[d, "f\times \id"] & T^*X\\
	A\times H^0(A, \Omega_A^1)\simeq T^*A\dar["\pr_2"]\\
	H^0(A, \Omega_A^1)
\end{tikzcd}
\end{center}
 where $T^*X$ and $T^*A$ are the holomorphic cotangent bundles, 
$df$ is the usual differential. Then note
that $pr_2(f\times \id)(df^{-1}(0_X))$ is precisely the 
set of 1-forms of the form $f^*\omega$ for $\omega\in H^0(A, \Omega_A^1)$ such that $f^*\omega$ admits zeros on $X$,
i.e.\
\[V(f) = pr_2(f_{\dagger}(0_X))\]
where following \emph{loc.\ cit.} we write $f_{\dagger} = f\times \id)\circ df^{-1}$ and $0_X = T_X^*X$ is the zero section
of the cotangent bundle. 

We are now ready to state Migliorini and Shende's
result \cite{MiSh} in the form that will be useful in the proof of our Theorem \ref{main1} and Theorem \ref{thm:smoothness}.

\begin{definition}[Migliorini and Shende]\label{def:higherdiscriminants}
Let $f: X\to Y$ be a proper morphism between smooth algebraic  varieties $X$ and $Y$. The higher discriminants of $f $ are defined to be
$$\Delta^i(f):=\{ y\in Y\ |\ \textnormal{no}\ (i-1)\textnormal{-dimensional subspace of}\  T_yY \ \textnormal{is transversal to} \ f \},$$ for each nonnegative integer $i$.
\end{definition}

\begin{remark}
(1) In the above definition an $i$-dimensional subspace $W$ of $T_yY$ is transversal to $f$ means $$df_x(T_xX)+W=T_yY,$$ for any closed point $x\in f^{-1}(y)$. Thus no $i$-dimensional subspace of $T_yY$ is transversal to $f$ means the rank of $f$ at some point in $f^{-1}(y)$ is less than $n-i$, where $n=\dim Y$.

(2) By definition, $\Delta^1(f)$ is the usual discriminant, i.e., the locus in $Y$ such that $f$ is not smooth. Also by \cite[Proposition 10.6]{Har77}, we have the filtration  $$Y=\Delta^0(f)\supset \Delta^1(f)\supset\Delta^2(f)\supset\Delta^3(f)\supset\ldots.$$

(3) $\Delta^i(f)$ are closed subvarieties of $Y$, which measure how singular $f$ can be.
\end{remark}



\begin{theorem}[Migliorini, Shende]\label{MiSh}
Let $f: X\to Y$ be a proper morphism between smooth algebraic varieties $X$ and $Y$. Then $$\tilde{f}(df^{-1}(0_X))=\bigcup_{i\geq0}\overline{T^*_{D^i(f)_{\textnormal{reg}}}Y},$$ where $0_X$ is the zero section of the cotangent bundle $T^*X$, and $\overline{T^*_{D^i(f)_{\textnormal{reg}}}Y}$ is the closure of the conormal bundle of the smooth locus $D^i(f)_{\textnormal{reg}}$ of codimension $i$ components  $D^i(f)$ of $\Delta^i(f)$ in $Y$.
\end{theorem}

Since the codimension $i$ components $D^i(f)$ of $\Delta^i(f)$ will be used for many times, we give the following definition.

\begin{definition}
Let $f: X\to Y$ be a proper morphism between smooth algebraic varieties $X$ and $Y$. The pure $i$-th discriminants of $f $ are defined to be
the union of codimension $i$ components of $\Delta^i(f)$, denoted by $D^i(f)$.
\end{definition}

From now on we will focus on a morphism $f: X\to A$ from smooth projective variety $X$ to abelian variety $A$. Also, we fix the following notation
$$\xymatrix{
T^*X
& X\times T_eA \ar[l]_-{df} \ar[r]^-{\tilde{f}} &T^*A\ar@{->>}[r]^-p& T^*_e A=H^0(A, \Omega_A^1)
},$$ where $T_eA$ is the tangent space of $A$ at a chosen origin $e$. Notice first $p(\tilde{f}(df^{-1}(0_X)))$ is exactly the set of holomorphic 1-forms $\omega\in H^0(A, \Omega_A^1)$ such that $f^*\omega$ has zeros on $X$.

\begin{corollary}\label{Cor:MiShCor}
Let $f: X\to A$ be a morphism from smooth projective variety $X$ to an abelian variety. Suppose there is a nontrivial holomorphic 1-form $\omega\in H^0(A, \Omega_A^1)$ such that $f^*\omega$ has zeros. Then there exists $i\geq1$ such that the pure $i$-th discriminant $D^i(f)$ is nonempty. In other words, $f$ is smooth if and only if $D^i(f)=\emptyset$ for all $i\geq 1$.
\end{corollary}

\begin{proof}
Assume that for all $i\geq 1$, $D^i(f)=\emptyset$, then $$\tilde{f}(df^{-1}(0_X))=\overline{T^*_{D^0(f)_{\textnormal{reg}}}Y}=\overline{T^*_{\Delta^0(f)_{\textnormal{reg}}}Y}=A\times \{0\}.$$ Hence $p(\tilde{f}(df^{-1}(0_X)))=\{0\}$, i.e., the only holomorphic 1-form admitting zeros is the trivial one, which is a contradiction.
\end{proof}



\subsection{Proof of Theorem \ref{thm:smooth}}















\subsection{subvarieties of abelian varieties}

First we generalize a proposition of Hacon and Kov\'acs \cite[Proposition 3.1]{HK05}. Part of the proof of the following proposition is due to \cite{HK05}.

\begin{proposition}\label{van-nonsimple}
Let $A$ be an abelian variety and $X$ be an irreducible proper subvariety of $A$. Then the following are equivalent

(1) $X$ is not fibred by tori or of dimension 0. 

(2) For general holomorphic 1-forms $\omega\in H^0(A, \Omega_A^1)$, the restricted holomorphic 1-form $\omega|_{X_{\textnormal{reg}}}$ admit zeros on the smooth locus $X_{\textnormal{reg}}$.
\end{proposition}

\begin{proof}\label{proof:van-nonsimple} (2)$\Rightarrow$ (1): Suppose $X$ is fibred by tori and $\dim X>0$, i.e.\ there exists a subabelian variety $B\subset A$ such that the fibres of the composition 
$X\into A \onto A/B$ are $B$. Taking an \'etale cover $\tau: A'\to A$, we can assume $A'=B\times C$ and $X':=\tau^{-1}(X)=B\times Y\subset A'$. Hence every 1-form coming from $B$ does not vanish on smooth locus of $X'$, hence on $X$. 

(1)$\Rightarrow$(2): Denote $d=\dim X$, $g=\dim A$. If $d=0$ it is trivial, so we assume $d>0$. For any point $x\in X_{\textnormal{reg}}$, we have $$T_xX_{\textnormal{reg}}\cong \C^{g-d}\subset T_xA\cong H^0(A, \Omega_A^1)^{\vee}\cong \C^g.$$
Hence we have the following diagram 
$$\xymatrix{
\mathbb{P}T^*_{X_{\textnormal{reg}}}A \ar[d]^{\pi} \ar@{^{(}->}[r]^-i
&X_{\textnormal{reg}}\times \mathbb{P}H^0(A, \Omega_A^1)^{\vee}\ar[dl] \ar@{->>}[r]^-p&\mathbb{P}H^0(A, \Omega_A^1)^{\vee} \\
X_{\textnormal{reg}}
}.$$
We then denote $S$ be the image of $\mathbb{P}T^*_{X_{\textnormal{reg}}}A$ in $\mathbb{P}H^0(A, \Omega_A^1)^{\vee}$ through the map $p\circ i$ in the above diagram. Notice that it suffices to show $S$ is dense in $\mathbb{P}H^0(A, \Omega_A^1)^{\vee}$ if $X$ is not fibred by tori and $\dim X>0$.  

Noticed that $S$ is irreducible, since $X$ is irreducible. By contradiction we assume $S$ is not dense in $\mathbb{P}H^0(A, \Omega_A^1)^{\vee}$. Denote $m:=\dim S$. Claim that $\dim S=m>0$. In fact, if $m=0$, then there is a nontrivial holomorphic 1-form $\eta$ corresponding to $s\in S$ such that the restriction $\eta|_X=0$. Also, notice that since $\dim S=0,$ $\dim X=g-1$. Also, since $X$ is not fibred by tori, $X$ generates the whole abelian variety $A$. Then this implies $\eta=0$, which is a contradiction. 

Then for a general point $s\in S$, $$\dim (p\circ i)^{-1}(s)=g-m-1>0,$$ hence $$\dim \pi((p\circ i)^{-1}(s))=g-m-1>0.$$ We denote $Z_s:= \pi((p\circ i)^{-1}(s))\subset X_{\textnormal{reg}}$. For each $Z_s$, we consider the nontrivial subabelian variety $A_s:=\langle Z_s \rangle$ generated by $Z_s$.  Notice first $A_s$ is a proper subabelian variety, because for general $x\in Z_s$, and the line $L_s\subset \mathbb{P}H^0(A, \Omega_A^1)^{\vee}$ corresponding to $s$, we have $$T_x(Z_s)\subset T_xX_{\textnormal{reg}}\subset L_s^{\perp}.$$
Since $A$ only contains at most countably many subabelian varieties and $S$ is irreducible, we have for general $s\in S$, $A_s$ are isomorphic to each other.  We denote this subabelian variety to be $A_0$ with a chosen origin $e$. Notice first that $T_e(A_0)$ is perpendicular to all $L_s$, since it is true for general $s$. Also, notice that since $X$ is not fibred by tori, $Z_s$ is a proper subvariety of the translated tori $A_0+x$ for general $s$ and any $x\in Z_s$ (Notice that $A_0+x$ does not depend on the choice of $x\in Z_s$). Thus $\dim A_0>g-m-1$, i.e., $\dim T_eA_0\geq g-m$. However, $\dim \bigcup_{s\in S}L_s=m+1$ ($\dim S=m$). So $T_e(A_0)$ cannot perpendicular to all $L_s$ for $s\in S$. This proves $S$ is dense in $\mathbb{P}H^0(A, \Omega_A^1)^{\vee}$ and hence the lemma.
\end{proof}

For any subvariety $X$ of an abelian variety $A$, we consider the following diagram 
$$\xymatrix{
T^*_{X_{\textnormal{reg}}}A\ar@{->>}[r]^-p& T^*_e A=H^0(A, \Omega_A^1)
},$$ where $p$ is the projection given by moving the cotangent vectors to $T^*_eA$ via group action. Notice that $p(T^*_{X_{\textnormal{reg}}}A)$ is exactly the set of holomorphic 1-forms in $H^0(A, \Omega_A^1)$ whose restriction on $X_{\textnormal{reg}}$ admitting zeros. Then we have the following corollary of Proposition \ref{van-nonsimple}.

\begin{corollary}\label{Main-coro}
Let $i: X\to A$ be a irreducible subvariety of an abelian variety $A$. Then $\overline{p(T^*_{X_{\textnormal{reg}}}A)}$ is linear. 
\end{corollary}


\begin{proof}
By Poincar\'e complete reducibility theorem, after passing to a \'etale covering $\tau:A'\to A$ we have a Cartesian diagram 
$$\xymatrix{
X'\ar[r]\ar@{^{(}->}[d]^-{i'}& X\ar@{^{(}->}[d]^-{i}\\
A'\ar[r]& A},$$ and we may assume $A'\cong B\times C$ with $B$ and $C$ being abelian varieties, $X'\cong Y\times C$ with $Y\subset B$ being a subvariety which is not fibred by tori in B. Then we have $$T^*_{X'_{\textnormal{reg}}}A'\cong T^*_{Y_{\textnormal{reg}}}B\times C.$$ As the above, we denote $p'$ be the natural projection $$\xymatrix{
T^*_{X'_{\textnormal{reg}}}A'\ar@{->>}[r]^-{p'}& T^*_e A'=H^0(A', \Omega_A'^1)
}.$$  Then by Proposition \ref{van-nonsimple}, we have $\overline{p'(T^*_{X'_{\textnormal{reg}}}A')}=H^0(B, \Omega_B^1)$, which is linear. Hence $\overline{p(T^*_{X_{\textnormal{reg}}}A)}$ is linear, since $X'\to X$ is \'etale.
\end{proof}

Next we will use Corollary \ref{Main-coro} and Theorem \ref{MiSh} to prove our first main result Theorem \ref{main1}.

\begin{proof}[Proof of Theorem \ref{main1}]
%For the morphism $f: X\to A$, we consider the following diagram 
%$$\xymatrix{
%T^*X
%& X\times T_eA \ar[l]_-{df} \ar[r]^-{\tilde{f}} &T^*A\ar@{->>}[r]^-p& T^*_e A=H^0(A, \Omega_A^1)
%},$$ where $T_eA$ is the tangent space of $A$ at a chosen origin $e$. Notice first $p(\tilde{f}(df^{-1}(0_X)))$ is exactly the set of holomorphic 1-forms $\omega\in H^0(A, \Omega_A^1)$ such that $f^*\omega$ has zeros on $X$.  
By Theorem \ref{MiSh}, we have $$\tilde{f}(df^{-1}(0_X))=\bigcup_{i\geq0}\overline{T^*_{D^i(f)_{\textnormal{reg}}}Y},$$ where $\overline{T^*_{D^i(f)_{\textnormal{reg}}}Y}$ is the closure of the conormal bundle of the smooth locus $D^i(f)_{\textnormal{reg}}$ of pure $i$-th discriminant $D^i(f)$. By Corollary \ref{Main-coro}, we have $p(\overline{T^*_{D^i(f)_{\textnormal{reg}}}Y)}=\overline{p(T^*_{D^i(f)_{\textnormal{reg}}}Y)}$ is linear. Hence $$p(\tilde{f}(df^{-1}(0_X)))=\bigcup_{i\geq0}p(\overline{T^*_{D^i(f)_{\textnormal{reg}}}Y})$$ is linear. Also, by the proof of Corollary \ref{Main-coro}, it is clear that each irreducible component of $V(f)$ are $\Q$-sub Hodge structures associated to subtori which are generated by $D^i(f)$.
\end{proof}




{\color{red} Add two corollaries, the above main theorem implies the main theorems of [Papa Schnell], [Hacon Kovacs]}

An immediate consequence of the above is the
following

\begin{proposition}
Given a morphism $f\colon X\to A$ from a smooth 
projective variety to an abelian variety $A$, there is a holomorphic 1-form $\omega\in H^0(A, \Omega_A^1)$ such that $f^*\omega$ has no zeros if and only
if for all $i\geq1$, and the pure $i$-th discriminant $D^i(f)$ of $f$

(1) $D^i(f)$ are either empty, i.e., $f$ is smooth (by Corollary \ref{Cor:MiShCor});

(2) or each irreducible component of $D^i(f)$ is fibered by tori and of positive dimension. 
\label{thm:nonvanishing}
\end{proposition}

\begin{proof} ``$\Leftarrow$'' By Theorem \ref{MiSh}, (1) implies that for every holomorphic 1-form $\omega\in H^0(A, \Omega_A^1)$, $f^*\omega$ has no zeros. By Proposition \ref{van-nonsimple}, (2) implies that there exist a holomorphic 1-form $\omega\in H^0(A, \Omega_A^1)$, $f^*\omega$ has no zeros.


``$\Rightarrow$'' 
If there is $i$ such that the pure $i$-th discriminant $D^i(f)$ is not fibred by tori and $\dim\geq 0$, applying Proposition \ref{van-nonsimple}, we have for general holomorphic 1-form $\omega\in H^0(A, \Omega_A^1)$, $\omega|_{D^i(f)_{\textnormal{reg}}}$ has zeros. Hence $$p(\overline{T^*_{D^i(f)_{\textnormal{reg}}}Y})=H^0(A, \Omega_A^1).$$ Hence $$p(\tilde{f}(df^{-1}(0_X)))=H^0(A, \Omega_A^1),$$ i.e.,$$V(f)=f^*H^0(A, \Omega_A^1).$$
\end{proof}

\sorry{This is a strengthening of a ``special case" of Stefan's Theorem 1.4 (2)}


\section{Holomorphic Tischler's theorem}


\begin{lemma}\label{Lem:inclusion-fiber-by-tori}
Let $A$ be any abelian variety. Let $Y_1\times B_1$ and $B_2$ be two subvarieties of $A$, where $Y_1$ is not fibred by tori, $B_2$ is a simple abelian variety and $B_1$ is an abelian variety. Suppose $B_2\subset Y_1\times B_1$. Then $B_2$ is a simple factor of $B_1$ upto isogeny.
\end{lemma}

\begin{proof}
It suffices to show that the natural map $p: B_2\to Y_1\times B_1\to Y_1$ maps $B_2$ to a point. Assume not, then we have a positive dimensional abelian variety $C:=p(B_2)$ contained in $Y_1$. Then we consider the inclusion $$C\subset Y_1\subset A.$$ Without losing of generality we can assume that $C$ contains a chosen origin $0$ by translating $C\subset Y_1$ using group action if necessary. Then we want to find contradiction from  $$0\in C\subset Y_1\subset A,$$ where $Y_1$ is not fibred by tori.

TBA
\end{proof}

With Lemma \ref{Lem:inclusion-fiber-by-tori} and the above theorem we prove, then we have the following strong result.

\begin{theorem}
Let $X$ be any smooth complex projective variety. Then $X$ admits a nowhere vanishing holomorphic 1-form $\omega$ if and only if $X$ admits a smooth morphism to positive dimensional abelian variety $A$
\end{theorem}

\begin{proof}
``$\Leftarrow$'' is trivial.

``$\Rightarrow$'' Consider the albanese map $a: X\to A_X$ of $X$. By assumption we have a holomorphic 1-form $\omega\in H^0(A_X, \Omega_{A_X}^1)$ such that $a^*\omega$ is nowhere vanishing. By Corollary \ref{Cor:MiShCor}, we may assume that not all $D^i(f)$ are empty for $i\geq 1$. %After passing to \'etale covering $\tau: A\to A_X$, we have the following  Cartesian diagram 
%$$\xymatrix{
%X'\ar[r]\ar[d]^-{f}& X\ar[d]^-{a}\\
%A\ar[r]^{\tau}& A_X},$$ $A'=A_1\times\ldots A_m$ being product of simple abelian varieties $A_i$, and a holomorphic 1-form $\omega'\in H^0(A, \Omega_{A}^1)$ such that $f^*\omega'$ is nowhere vanishing.
Then by Theorem \ref{thm:nonvanishing}, we get all the nonempty pure $i$-th discriminants $D^i(a)$ are fibred by tori and $\dim D^i(a)>0$. Noticed also, by ({\color{red} Hartshorne generic smooth}), we claim$$D^i(a)\supset D^j(a),$$ if $j>i$ and $D^i$ and $D^j$ are nonempty pure $i$-th discriminants in $\{D^k(a)\}_{k=1}^{g}$, where $g=\dim A$. {\color{blue}  I Need to think about this, which seems not OK! } Assume the highest nontrivial discriminant is $D^m(a)$, which is fibred by tori, then we can take simple abelian subvariety$A_0$ of $A_X$ so that $A_0$ is a factor of the torus part of  $\Delta^m(a)$. By Lemma \ref{Lem:inclusion-fiber-by-tori}, $A_0$ is a factor of all the discriminants. Notice that there is a \'etale covering $\tau: A_X\to A_0\times B$ with $B$ another abelian variety. Then from the proof of Corollary \ref{Main-coro}, we have all the nontrivial holomorphic 1-forms in $(\tau\circ a)^*H^0(A_0, \Omega_{A_0}^1)$ has no zeros on $X$. But since $A_0$ is simple, we have $\tau\circ a: X\to A_0$ is smooth by Theorem \ref{thm:smooth}.   
\end{proof}

{\color{red}(This theorem covers theorem A, remember to replace this as Theorem A)}



\section{Linearity of holomorphic 1-forms with codimension one zeros}

\begin{proof}[Proof of Theorem \ref{thm:smooth}]
``$\Rightarrow$'' is trivial.

``$\Leftarrow$'' Assume $f$ is not smooth, then there is a holomorphic 1-form has zeros. Then there exist $i\geq 1$ such that $D^i(f)$ is nonempty by Corollary \ref{Cor:MiShCor}. Since $A$ is simple, $\omega|_{D^i(f)_{reg}}$ has zero for general holomorphic 1-forms $\omega\in H^0(A, \Omega_A^1)$ by \cite[Proposition 3.1]{HK05}, i.e.,  $p(\overline{T^*_{D^i(f)_{\textnormal{reg}}}A})=H^0(A, \Omega_A^1)$, which means every holomorphic 1-form pulled back from $A$ has zero.
\end{proof}

{\color{red} $\bullet$ Use Normal bundle also to explain nonresonant 1-forms form finite points in $\mathbb{P}H^0(X, \Omega_X^1)$.

$\bullet$ Application in bounding the genus of 1-dimensional locus of nonresonant  1-forms on surfaces.}




















\begin{thebibliography}{HKLR} 


%\bibitem[AMN]{AMN} J. Amor\'os, M. Manjar\'in,  M. Nicolau, \textit{Deformations of K\"ahler manifolds with nonvanishing holomorphic vector fields}, J. Eur. Math. Soc. \textbf{14}, (2012), 997--1040.


%\bibitem[BeFa08]{BeFa08} K. Behrend, B. Fantechi,  \textit{Symmetric obstruction theories and Hilbert schemes of points on threefolds},  Algebra and Number Theory,  \textbf{2} (2008), 313--345.



\bibitem[CL73]{CL73} J. B. Carrell, D. I. Lieberman,  \textit{Holomorphic vector fields and K\"ahler manifolds}, Invent. Math. \textbf{21} (1973), 303--309.

\bibitem[DiPa13]{DiPa13}A. Dimca, S. Papadima, \textit{Nonabelian cohomology jump loci from ananalytic view point}. Commun. Contem. Math. \textbf{15} (5), 1350025 (2013).


\bibitem[DiSu]{disu} A. Dimca, A. I. Suciu, \textit{Which 3-manifold groups are K\"ahler groups?}, J. Eur. Math. Soc., \textbf{11} (2009), no. 3, 521-528.

\bibitem[GH18]{GH18} I. Glazer, Y. I. Hendel, \textit{On singularity properties of convolutions of algebraic morphisms--the general case (with an appendix joint with Gady Kozma)}, arXiv:1811.09838.

\bibitem[GH19]{GH19}  I. Glazer, Y.I. Hendel, \textit{On singularity properties of convolutions of algebraic morphisms}, Selecta Mathematica (2019), 25--15


\bibitem[GL87]{GL87} M. Green, R. Lazarsfeld, \textit{Deformation theory, generic vanishing theorems and some conjectures of Enriques, Catanese and Beauville}, Inv. Math. \textbf{90} (1987), 389--407.

\bibitem[GW10]{GW10} U. G\"ortz, T. Wedhorn, \textit{Algebraic geometry I. Schemes with examples and exercises.} Advanced Lectures in Mathematics. Vieweg+Teubner, Wiesbaden, 2010. 

\bibitem[HK05]{HK05} C. D. Hacon, S. J. Kov\'acs, \textit{Holomorphic one-forms on varieties of general type}, Ann. Sci. \'Ecole Norm. Sup. \textbf{38} (2005), no. 4, 599--607.

\bibitem[Har77]{Har77} R. Hartshorne, \textit{Algebraic Geometry}, Graduate Texts in Mathematics, vol. 52, Springer, New York,
1977.

\bibitem[HS19]{HS19} F. Hao, S. Schreieder, \textit{Holomorphic one-forms without zeros on threefolds}, to appear Geometry $\&$ Topology , arXiv:1906.07606.


\bibitem[Laz04]{Laz04} R. Lazarsfeld, \textit{Positivity in algebraic geometry, II: Positivity for Vector Bundles, and Multiplier Ideals}, Ergebnisse der Mathematik und ihrer Grenzgebiete. (3) \textbf{49}, Springer, Berlin, 2004.


%\bibitem[KM05]{KM05} J. Koll\'ar and S. Mori, \textit{Birational geometry of algebraic varieties}, Cambridge University Press, Cambridge, 2008.

%\bibitem[Mo85]{Mo85} S. MORI S, \textit{Classification of higher-dimensional varieties}, in: Algebraic Geometry, Bowdoin, (1985),Brunswick, Maine, (1985), in: Proc. Sympos. Pure Math., \textbf{46}, American Mathematical Society,Providence, RI, (1987), 269--331.

%\bibitem[LMW18]{LMW18} Y. Liu, L. Maxim, B. Wang, {\it Perverse sheaves on semi-abelian varieties}, arXiv: 1804.05014. 

\bibitem[LMW20]{LMW20} Y. Liu, L. Maxim, B. Wang, \textit{Aspherical manifolds, Mellin transformation and a question of Bobadilla-Koll\'ar}, arXiv:2006.09295.

\bibitem[MiSh18]{MiSh18} L. Migliorini and V. Shende, \textit{Higher discriminants and the topology of algebraic maps}, Algebraic geometry, (2018), 114--130.

\bibitem[Mum]{Mum} D. Mumford, \text{Abelian Varieties}, Tata Institute of Fundamental Research, (1985).

\bibitem[PS14]{PS14} M. Popa and C. Schnell, \textit{Kodaira dimension and zeros of holomorphic one-forms}, Ann. of Math. \textbf{179} (2014), 1109--1120.


\bibitem[Sha74]{Sha74} I. Shafarevich, \textit{Basic Algebraic Geometry I}. Berlin, Springer--Verlag, (1974).

\bibitem[Sch15]{Sch}
C. Schnell, {\it  Holonomic $\mathcal{D}$-modules on abelian varieties},
Publ. Math. Inst. Hautes \'{E}tudes Sci. 121 (2015), 1--55.

\bibitem[Schm76]{Schm76} W. Schmid, Variation of Hodge structure: the singularities of the period mapping, Inv. Math. \textbf{22} (1973), 211--319.

\bibitem[Ste76]{Ste76} J. Steenbrink, \textit{Limits of Hodge structures}, Inv. Math. \textbf{31} (1976), 229--257.




\bibitem[Sim93]{Sim93} C. Simpson, \textit{Lefschetz theorems for the integral leaves of a holomorphic one-form}, Compositio Math. \textbf{87} (1993), 99--113.

\bibitem[Sp88]{Sp88} M. Spurr, {\it On the zero set of a holomorphic one-form on a compact complex manifold.} Trans. Am.
Soc. \textbf{308} (1988), 329-339. 


\bibitem[SS19]{SS19} S.Schreieder, \textit{Zeros of holomorphic one-forms and topology of K\"ahler manifolds, (Appendix written jointly with H.-Y. Lin) }, to appear IMRN. (2019).


\end{thebibliography}

\end{document}