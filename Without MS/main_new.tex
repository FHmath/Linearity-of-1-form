\pdfoutput=1
\documentclass[12pt,reqno]{amsart}
\usepackage[letterpaper,margin=1in,footskip=0.25in]{geometry}
\usepackage{mathrsfs}
\usepackage{amssymb}
\usepackage{mathtools}
\usepackage{tikz-cd}
\usepackage{enumitem}

\PassOptionsToPackage{pdfusetitle,pagebackref,colorlinks}{hyperref}
\usepackage{bookmark}
\hypersetup{
  linkcolor={red!70!black},
  citecolor={green!70!black},
  urlcolor={blue!80!black}
}

\newtheorem{theorem}{Theorem}[section]
\newtheorem{lemma}[theorem]{Lemma}
\newtheorem{proposition}[theorem]{Proposition}
\newtheorem{corollary}[theorem]{Corollary}
\newtheorem{claim}[theorem]{Claim}
\newtheorem{conjecture}[theorem]{Conjecture}
\newtheorem{step}{Step}[subsection]
\renewcommand{\thestep}{\arabic{step}}

\newtheorem{alphtheorem}{Theorem}
\renewcommand{\thealphtheorem}{\Alph{alphtheorem}}
\theoremstyle{question}
\newtheorem{question}[theorem]{Question}
\theoremstyle{definition}
\newtheorem{definition}[theorem]{Definition}
\newtheorem{notation}[theorem]{Notation}

\theoremstyle{remark}
\newtheorem{remark}[theorem]{Remark}

\newtheoremstyle{cited}{.5\baselineskip\@plus.2\baselineskip\@minus.2\baselineskip}{.5\baselineskip\@plus.2\baselineskip\@minus.2\baselineskip}{\itshape}{}{\bfseries}{\bfseries .}{5pt plus 1pt minus 1pt}{\thmname{#1}\thmnumber{~#2}\thmnote{ \normalfont#3}}
\theoremstyle{cited}
\newtheorem{citedthm}[theorem]{Theorem}
\newtheorem{citedconj}[theorem]{Conjecture}
\newtheorem{citedlem}[theorem]{Lemma}
\newtheorem{citedprop}[theorem]{Proposition}

\newtheoremstyle{citeddef}{.5\baselineskip\@plus.2\baselineskip\@minus.2\baselineskip}{.5\baselineskip\@plus.2\baselineskip\@minus.2\baselineskip}{}{}{\bfseries}{\bfseries .}{5pt plus 1pt minus 1pt}{\thmname{#1}\thmnumber{~#2}\thmnote{ \normalfont#3}}
\theoremstyle{citeddef}
\newtheorem{citednot}[theorem]{Notation}


%%==============Yongqiang's typsets====================%%

\newcommand{\CN}{\mathbb{C}^{n+1}}
\newcommand{\CP}{\mathbb{CP}^{n+1}}
\newcommand{\U}{\mathcal{U}}
\newcommand{\C}{\mathbb{C}}
\newcommand{\Z}{\mathbb{Z}}
\newcommand{\Hom}{\mathrm{Hom}}
\newcommand{\Q}{\mathbb{Q}}
\newcommand{\K}{\mathcal{L}}
\newcommand{\V}{\mathcal{V}}

\def\be{\begin{equation}}
\def\ee{\end{equation}}

\def\bt{\begin{theorem}}
\def\et{\end{theorem}}

\def\bc{\begin{corollary}}
\def\ec{\end{corollary}}

\def\br{\begin{remark}}
\def\er{\end{remark}}

\def\bp{\begin{proposition}}
\def\ep{\end{proposition}}

\def\bl{\begin{lemma}}
\def\el{\end{lemma}}

%\def\bn{\begin{enumerate}}
%\def\en{\end{enumerate}}

\def\bex{\begin{ex}}
\def\eex{\end{ex}}

\def\bd{\begin{definition}}
\def\ed{\end{definition}}








%\DeclareMathOperator{\Pic}{Pic}                  % Pic

%\newcommand{\PP}{\mathcal{P}}           % Poincare line bunle

\DeclareMathOperator{\Supp}{Supp}                % Supp
\DeclareMathOperator{\codim}{codim}              % codim
\DeclareMathOperator{\mreg}{mreg}                % mreg
\DeclareMathOperator{\reg}{reg}                  % reg
\DeclareMathOperator{\sing}{sing}                  
\DeclareMathOperator{\id}{id}                    % id
\DeclareMathOperator{\obj}{Obj}
\DeclareMathOperator{\ad}{ad}
\DeclareMathOperator{\morph}{Morph}
\DeclareMathOperator{\enom}{End}
\DeclareMathOperator{\iso}{Iso}
\DeclareMathOperator{\Exp}{Exp}
\DeclareMathOperator{\homo}{Hom}
\DeclareMathOperator{\enmo}{End}
\DeclareMathOperator{\spec}{Spec}
\DeclareMathOperator{\fitt}{Fitt}
\DeclareMathOperator{\odr}{\Omega^\bullet_{\textrm{DR}}}
\DeclareMathOperator{\rank}{Rank}
\DeclareMathOperator{\gdeg}{gdeg}
\DeclareMathOperator{\Alb}{Alb}
\DeclareMathOperator{\alb}{alb}
\DeclareMathOperator{\Ann}{Ann}



%\DeclareMathOperator

\DeclareMathOperator{\Char}{Char}
\DeclareMathOperator{\CC}{SS}
\DeclareMathOperator{\kn}{Ker}
\DeclareMathOperator{\im}{Im}


\def\ra{\rightarrow}


\def\bone{\mathbf{1}}
\def\bC{\mathbb{C}}
\def\cM{\mathcal{M}}
\def\cV{\mathcal{V}}
\def\Def{{\rm {Def}}}
\def\cR{\mathcal{R}}
\def\om{\omega}
\def\wti{\widetilde}
\def\al{\alpha}
\def\End{{\rm {End}}}
\def\Pic{{\rm Pic}}
\def\bP{\mathbb{P}}
\def\cH{\mathcal{H}}
\def\bL{\mathbb{L}}
\def\cX{\mathcal{X}}
\def\cI{\mathcal{I}}
\def\pa{\partial}
\def\cY{\mathcal{Y}}
\def\cD{\mathcal{D}}
\def\cO{\mathcal{O}}
\def\lra{\longrightarrow}
\def\bQ{\mathbb{Q}}
\def\ol{\overline}
\def\cL{\mathcal{L}}
\def\bH{\mathbb{H}}
\def\bZ{\mathbb{Z}}
\def\bW{\mathbf{W}}
\def\bV{\mathbf{V}}
\def\bM{\mathbf{M}}
\def\eps{\epsilon}
\def\ul{\underline}
\def\lam{\lambda}
\def\sX{\mathscr{X}}
\def\bN{\mathbb{N}}


%%==========Yagna's typsets============%%
\usepackage{tikz-cd}
\usepackage{enumitem}

\PassOptionsToPackage{pdfusetitle,pagebackref,colorlinks}{hyperref}
\usepackage{bookmark}
\hypersetup{
  linkcolor={red!70!black},
  citecolor={green!70!black},
  urlcolor={blue!80!black}
}

%Mathcal Letters =====================
\newcommand{\sA}{\mathcal{A}}
\newcommand{\sB}{\mathcal{B}}
\newcommand{\sD}{\mathcal{D}}
\newcommand{\sF}{\mathcal{F}}
\newcommand{\sG}{\mathcal{G}}
\newcommand{\sH}{\mathcal{H}}
\newcommand{\sK}{\mathcal{K}}
\newcommand{\sL}{\mathcal{L}}
\newcommand{\sM}{\mathcal{M}}
\newcommand{\sN}{\mathcal{N}}
\newcommand{\sO}{\mathcal{O}}
\newcommand{\sP}{\mathcal{P}}
\newcommand{\sQ}{\mathcal{Q}}
\newcommand{\sR}{\mathcal{R}}
\newcommand{\sT}{\mathcal{T}}
\newcommand\sV{{\mathcal V}}
\newcommand\sW{{\mathcal W}}
\newcommand{\sZ}{\mathcal{Z}}

%mathbb Letters======
\newcommand{\bbA}{\mathbb{A}}
\newcommand{\bbB}{\mathbb{B}}
\newcommand{\bbC}{\mathbb{C}}
\newcommand{\bbG}{\mathbb{G}}
\newcommand{\bbH}{\mathbb{H}}
\newcommand{\bbK}{\mathbb{K}}
\newcommand{\bbL}{\mathbb{L}}
\newcommand{\bbM}{\mathbb{M}}
\newcommand{\bbN}{\mathbb{N}}
\newcommand{\bbP}{\mathbb{P}}
\newcommand{\bbQ}{\mathbb{Q}}
\newcommand{\bbR}{\mathbb{R}}
\newcommand{\bbV}{\mathbb{V}}
\newcommand{\bbZ}{\mathbb{Z}}



%Script Letters ======================
\newcommand{\frf}{\mathfrak{f}}
\newcommand{\frM}{\mathfrak{M}}

\newcommand{\scrL}{\mathscr{L}}
\newcommand{\crI}{\mathscr{I}}
\newcommand{\scrK}{\mathscr{K}}
\newcommand{\scrB}{\mathscr{B}}
\newcommand{\scrC}{\mathscr{C}}
\newcommand{\scrD}{\mathscr{D}}
\newcommand{\scrE}{\mathscr{E}}
\newcommand{\scrI}{\mathscr{I}}
\newcommand{\scrQ}{\mathscr{Q}}
\newcommand{\scrR}{\mathscr{R}}
\newcommand{\scrX}{\mathscr{X}}
\newcommand{\scrY}{\mathscr{Y}}
\newcommand{\scrF}{\mathscr{F}}
\newcommand{\Dred}{\lceil D\rceil}

%Arrow Style =========================
\newcommand{\into}{\hookrightarrow}
\newcommand{\onto}{\rightarrow\hspace*{-.14in}\rightarrow}
\def\acts{\curvearrowright}

%Math Operators ======================



%\DeclareMathOperator{\alg}{alg}
\DeclareMathOperator{\BM}{BM}
\DeclareMathOperator{\Bs}{Bs}
\DeclareMathOperator{\Bsp}{\mathbf{B}_+}
\DeclareMathOperator{\SB}{\mathbf{B}}

\DeclareMathOperator{\coh}{coh}
\DeclareMathOperator{\Coker}{Coker}
 \renewcommand{\div}{\text{div}}

\DeclareMathOperator{\DR}{DR}
\DeclareMathOperator{\Ch}{Ch}
\DeclareMathOperator{\discrep}{discrep}
\DeclareMathOperator{\exc}{exc}

\DeclareMathOperator{\free}{free}


\DeclareMathOperator{\HHom}{\mathcal{H}\!\mathit{om}}
\DeclareMathOperator{\image}{Im}


\DeclareMathOperator{\Ind}{Ind}
\DeclareMathOperator{\Ker}{Ker}
\DeclareMathOperator{\Lie}{Lie}
\DeclareMathOperator{\op}{op}


\DeclareMathOperator{\qcoh}{qcoh}

\DeclareMathOperator{\dr}{DR}

\DeclareMathOperator{\RHHom}{\mathbf{R}\mathcal{H}\!\mathit{om}}


\DeclareMathOperator{\Spf}{Spf}
%\DeclareMathOperator{\Supp}{Supp}
\DeclareMathOperator{\tors}{tors}
\DeclareMathOperator{\torsion}{torsion}
\DeclareMathOperator{\Var}{Var}
%\DeclareMathOperator{\Sym}{Sym}


\newcommand{\Ab}{\mathbf{Ab}}
\newcommand{\Aff}{\mathbf{Aff}}
\newcommand{\tB}{\tilde{B}}
%\newcommand{\et}{{\acute{e}t}}

\newcommand{\Sets}{\mathbf{Sets}}
%\newcommand{\cX}{\bar{X}}
\newcommand{\cP}{\bar{P}}
%\newcommand{\cV}{\bar{V}}
\newcommand{\cW}{\bar{W}}
\newcommand{\sorry}[1]{\textcolor{red}{#1}}
\newcommand{\dual}[1]{\sD^{\Omega}_{M^{#1}}}



\title{}








\begin{document}  
\title[Cohomology jump loci and holomorphic 1-forms with zeros]{On linearity of holomorphic 1-forms with zeros} 

\author{Yajnaseni Dutta}

%\address{}
%\email{}

\author{Feng Hao}

%\address{}
%\email{}

\author{Yongqiang Liu}

%\address{}
%\email{}


%\date{\today}
%\subjclass[2010]{} 
%\keywords{} 



\begin{abstract} 
The goal of this article is two-fold, first we 
\end{abstract}

\maketitle
\tableofcontents
\section{Introduction}\label{intro}
Given a smooth porjective variety $X$, in this article we relate the cohomology jump loci of $\bbC_X$ to the 
stratification arising in the decomposition theorem for the albnese morphism. Such relation fell out of our interest in the
study of holomorphic 1-forms. It has been indicated by plethora of results
(\cite{GL87, HK05, LZ05,
Sch19, HS19, PS14} to name just a few) that a lot of the geometry and topology of the
variety depends on vanishing of such forms. Conjecturally the following two statements
are also a bridge between the real and the complex worlds.
\begin{enumerate}
	\item $X$ admits a global holomorphic 1-form $\omega$
	such that $Z(\omega) = \emptyset$, i.e.\ there does \textsl{not}
	exist any $x\in X$ such that $\omega(T_xX) = 0$. 
	\item $X$ admits a $\scrC^{\infty}$-fibration over the circle.
\end{enumerate}
The direction (1) $\Rightarrow$ (2) was shown by Tischler,
\cite{Tis70}. The opposite direction is conjectural (attributed
to Kotschik) and
was recently established by Shreieder and the first second author
\cite{HS19}. We show the following 



\begin{alphtheorem}\label{thm:smooth}
Given a $f\colon X\to A$ be from a smooth projective variety $X$ to a simple abelian variety $A$ such that
both $X$ and $A$ admit $\scrC^{\infty}$-fibre bundle structure over $S^1$ with the commutative diagram
\[\begin{tikzcd}
	X\ar[dr]\ar[d, "f"]& \\
	A\ar[r]&S^1
\end{tikzcd}.\] 
Then $f$ is cohomologically trivial, i.e.\ we have the following
decomposition for any semi-simple complex local system $\bbL$ on $X$
\[\bbR f_*\bbL \simeq \bigoplus_i \bbR^i f_*\bbL[-i]\]
and $\bbR^i f_*\bbL$ are local system. In particular, the
singular fibres of $f$ carry a pure Hodge structure.

Furthermore, the decomposition $\bbR f_*\bbC$ underlies an integral lattice, in other words, $R^ia_*\bbZ$ are also local systems for 
all $i\geq 0$.
%Then $f$ is cohomologically a fiber bundle if and only if there is a global holomorphic 1-form $\omega$ on $A$ such that $f^*\omega$ is nowhere vanishing.  

\end{alphtheorem}
In fact a bit more can be said; When $A$ is not necessarily simple existence of a non-vanishing global holomorphic 1-form
is equivalent to certain restrictions on the singular support of 
these pushforwards. See Theorem \ref{thm:nonvanishing} for more details. 
The link between such
 real geometry and complex geometry of $X$ is dictated
by the generic vanishing theory. More precisely, Schreieder 
showed \cite[Theorem 1.2]{Sch19} that such real fibre bundle structure gives rise to the existence of certain
holomorphic 1-form and vanishing of cohomologies of certain rank 1 complex local system. We prove a more general
version of this statement in a relative setting. 
Another key input here is the result of
Kashiwara giving an estimate on the behaviour of the singular
support of $\bbR f_*\bbL$ (see
Theorem \ref{th:kas-ss}). This prompted us to look further into the relationship between the generic vanishing theory and the decomposition
theorem. To state our next result, we need the following notations
Denote by $\SS(Ra_*\bbC)\subset T^*\Alb_X$ the singular support under the albanese morphism $a\colon X\to \Alb_X$ and $\pi\colon T^*\Alb_X\to 
H^0(\Alb_X, \Omega_{\Alb_X}^1)$ is the porjection. $\sR(X)$ denote the (1,0)-piece of the union of generic vanishing loci
\[\sigma^{p,q}(X) \coloneqq \{(\sL,\omega)\in \Pic^0(X)\times H^0(\Alb_X,\Omega_{\Alb_X}^1)|H^q(H^{p}(X, \Omega_X^{\bullet}\otimes \sL), \wedge a^*\omega) \neq 0\}.\]
Via the non-abelian Hodge correspondence a different way to understand this set is via the Lie algebra associated to the  the cohomology jump loci 
\[\sV(X,\bbC) \coloneqq \{\rho\in\Char(X)| \bbH^i(X, \bbC_{\rho})\neq  0\text{ for some } i\} \]
where $\bbC_{\rho}$ is the local system associated 
to the character $\rho\in Hom(H_1(X,\bbZ)/\torsion, \bbC^{\star})$.
The Lie algebra of this set lies in $H^1(X,\bbC)$. Its holomorphic part coincides with $\sR(X)$ above
(see \S \ref{sec:gv} for more details). 

\begin{alphtheorem}
Let $a\colon X\to \Alb_X$ denotes the albanese morphism. Then
\[\sR(X) =  \pi(\SS(Ra_*\bbC)). \]
\label{thm:linearity}
\end{alphtheorem}
%\begin{remark}[Previous results]
%\begin{enumerate}
%\item \label{item:ps} When $X$ is of general type, it was conjectured in
	%\cite{HK05, LZ05} and was proved
	%by Popa and Schnell \cite{PS14} every global holomorphic 1-		
	%form vanishes on $X$. 
%\item \label{item:hk} On the opposite extreme for any $A$, not necessarily simple when $f$ is singular along a divisor of general
		%type in $A$, Hacon and Kov\'acs \cite[Proposition 3.5.]{HK05} show that
		%$f^*\omega$ always admits zero. See Corollary \ref{cor:hk}
		%for a reinterpretation of their argument. 
%\end{enumerate}	
%\end{remark}

The theorem follows from statement
about perverse sheaves on abelian varieties. Recall
\begin{definition}
Associated to $\rho\in \Char(X)$ the collection of local systems $\bbC_{\rho}$ defined as
\[\sV^i(A, \sP) \coloneqq \{\rho\in\Char(X)| H^i(A, \sP\otimes \bbC_{\rho}) \neq 0\}\]
is called the cohomology jump loci of $\sP$.
\label{def:cjl}
\end{definition}
\begin{alphtheorem}
Let $A$ be an abelian variety. Let $\sP$ be a complex of perverse sheaves with complex coefficient on $A$. 
Then we have the equality that
$$\pi(\SS(\sP)) = \text{(1,0)-piece of }\bigcup_{\rho} \rho^{-1} TC_{\rho} \sV^0(A,\sP), $$
where the union is running over representative points from every irreducible components of $\sV^0(A,P)$
$TC_{\rho} \sV^0(A,P) \subseteq H^1(X, \bbC)$ denotes the tangent cone at $\rho$. 
\label{thm:perverse}
\end{alphtheorem}

The key technique we
use to relate the the generic vanishing theory on abelian varieties with the support of $Rf_*\bbC$ is Kashiwara's
global index theorem. In terms of simple holonomic D-modules
on abelian variety we can interpret the above result as

\begin{alphtheorem}
Let $\sM$ be a simple holonomic D-module on an abelian variety $A$.
Then the set
\[\sR_d(X)\coloneqq \{\omega\in H^0(A, \Omega_A^1)|\exists \sL\in \Pic^0(A) \text{ with }
H^{k}(A, \dr(\sM\otimes(\sL, \nabla_{\omega}))) \neq 0\}\]
satisfies
\[\sR_d(X) = \pi(\SS(\sM))\]
where $\nabla_{\omega}\colon \sL \to \Omega_A^1\otimes \sL)$
denotes the integrable connection on $\sL$ associated to the Higgs
field $\omega$ and $\SS(\sM)$ is the singular support of $\sM$.
\label{thm:dlinearity}
\end{alphtheorem}

This brings us to the question of linearity. It is well-known \cite[p.\ 311]{Ara92} $\sR(X)$ that is a finite union of linear subspaces of $H^0(\Alb_X, \Omega_{\Alb_X}^1)$. 
On the other hand, by Proposition \ref{van-nonsimple} that for any subvariety $Z$ of an abelian variety $A$, the collection
of 1-forms appearing in the conormal spaces $T^*_ZA$ is also linear in the above sense. Furthermore, both sets are subsets of the set $V(X)$ of holomorphic 1-forms that admits zeros. Hence, the result of Theorem \ref{thm:linearity} can be interpretted as identifying the linear piece of $V(X)$. 
However, in the spirit of Carrell and Lieberman \cite{CL73}, who showed that
the set of global holomorphic tangent vector fields with zeros
is linear,
one can ask the following 
\begin{question}
Is the following set
\[V(X):=\{ \omega\in H^0(X, \Omega_X^1) | Z(\omega)\neq \emptyset\}\]
os holomorphic 1-forms admitting zeros linear, i.e.\ a finite
union of linear subspaces of the vector space $H^0(X, \Omega_X^1)$.
\end{question} 

The answer to this question is yes up-to 3-folds by
a result of the second author obtained jointly with Schreieder
\sorry{insert theorem reference}.  


%\cite[Theorem 4.2]{Sim93} (see also \cite{DiPa13} from where
%we borrow the terminology). 
%Henceforth we shall call finite unions of subvector spaces as \emph{linear subvarieties}. 
%In the relative setting
%of a morphism $f\colon X\to A$ from a smooth projective variety to an abelian variety $A$,
%we similarly define
%\[V(f):=\{ \omega\in H^0(A, \Omega_A^1) | Z(f^*\omega)\neq \emptyset\}.\]



\begin{remark}
\begin{enumerate}
\item Note that a consequence of Theorem \ref{thm:linearity} is that $\sR(X)\subseteq V(X)$. This was known by the generic vanishing theory
(see \cite{BWY} for more general statement). 
\item Note that if $\chi(X)>0$, we have $\sR(X) = V(X) = H^0(X, \Omega_X^1)$. Hence in this case the set of 1-forms that admit zeros is linear as already mentioned above. 
This follows immediately from the generic vanishing theory. Indeed, by Hodge decomposition
we have
$H^k(X,\bbC) \simeq \bigoplus_{p+q = k} H^p(X,\Omega_X^q)$. Hence the complexes $(H^p(X, \Omega_X^{\bullet}), \wedge\omega)$ can be summed together to form the complex $(H^p(X, \bbC), \wedge\omega)$. Since $\chi(X)>0$ the latter complex cannot be exact. Hence $H^0(X,\Omega_X^1)
= \sR(X)$.
\item The equality $\sR(X) = V(X)$ is known to fail
in case of \cite[Example 1.11]{DJL17} given by the surface $X \coloneqq C_1\times C_2/\sigma_1\times \sigma_2$ where $C_1$ is a curve of genus $>1$, $C_2$ is an Elliptic curve and the diagonal action is induced by $\sigma_1 \acts C_1$ with $E \coloneqq C_1/\sigma_1$ an elliptic curve and $\sigma_2\acts C_2$ is a free and properly discontinuous action. In this case, $\sR(X, f) = 0$
but $V(X, f) = H^0(E, \Omega_E^1)$ for $f\colon X\to A$. 
However the
finite \'etale cover given by $C_1\times (C_2/\sigma_2)=:X'\xrightarrow{\tau} X$ such that for all 1-form $\omega$ on $X'$ coming from $E$,
satisfies $(H^{\bullet}(X', \bbC), \wedge\omega)$ is not exact. In the light of this example and Schreieder's result (see \ref{Thm:schreieder}), it is natural to ask if it always happen.
\end{enumerate}
\end{remark}
\begin{question}
Soes there exist an \'etale finite morphism $\tau\colon  X' \to X$ 
such that for all $\omega \in H^0(X', \Omega_{X'}^1)$ with
$Z(\omega)\neq \emptyset$, there is a line bundle $\sL\in\Pic(X)$
and integers $p, q$ such that
\[ H^q(H^{p}(X', \Omega_{X'}^{\bullet}\otimes \tau^*\sL), \wedge \tau^*\omega)\neq 0?\]
Or equivalently, does there exist a line bundle $\sL\in\Pic(X)$ such that the local system $\bbL$ corresponding to
$(\sL, \omega)$ under the non-abelian Hodge correspondence satisfy
\[ H^{i}(X', \tau^*\bbL) =0\]
for all $i\geq 0$?
\end{question}
%An example of Debarre, Jiang and Lahoz 
%\cite[Example 1.11]{DJL17} shows that the
%there exists a bi-elliptic surface $S$ admitting a 1-form 
%$\omega$ for which $(H^*(X, \mathbb{C}), \wedge \omega)$
%is exact, yet $\omega$ admits zeros on $S$. Theorem \ref{thm:linearity} shows that such forms are not under the realm of generic vanishing theory. 









In this paper, all complex of sheaves and perverse sheaves are defined with complex
coeffcients. All the varieties are complex quasi-projective varieties. 

\subsection*{Acknowledgements}



\section{Preliminary}
\subsection{Resonant 1-forms and Hodge decomposition}
Given a holomorphic 1-form $\omega\in H^0(X,\Omega_X^1)$
the kernel of the associated Koszul complex
\begin{equation}
\sK^{\bullet}_{\omega} \coloneqq [\sO_X\overset{\wedge\omega}{\to} \Omega_X^1 \to \cdots\to \Omega_X^{n-1}\to \Omega^n_X.]
\label{eq:koszul}
\end{equation}
defines a rank 1 local system $L(\omega)$ (see \cite[\S 2.1]{sch}
for a construction). This gives in the way of the generic vanishing theory developed by \cite{GL, Ara, Sim} into the 
study of zeros of holomorphic 1-forms. In this section we discuss 
the relevant bits of this vast theory that we will use in various
proofs.
%
%\begin{definition}[Zero scheme of 1-forms]\label{def:zeroscheme}
%For $\omega\in H^0(X, \Omega_X^1)$, the \emph{zero set} $Z(\omega)$ of $\omega$ is the algebraic set of closed point $x$ in $X$, such that $\omega(v)=0$
%for all tangent vectors $v\in T_xX$ at $x$. 
%
%The \emph{zero scheme} of $\omega$ is the closed subscheme $\sZ(\omega)$ defined by the ideal sheaf $\mathcal{I}_{\omega}$ given by the image of the morphism 
%\[\mathcal{T}_X\overset{\langle\omega, \cdot\rangle}{\longrightarrow} \mathcal{O}_X.\] 
%Here $\mathcal{T}_X$ is the tangent sheaf of $X$ and $\langle\omega, \cdot\rangle$ denotes the pairing of tangent field with 
%the 1-form $\omega$.
%\end{definition}

 
The generic vanishing theory 
\cite[Proposition 3.4]{GL} ensures that
if $Z(\omega)\neq \emptyset$ then the sequence
\[\cdots\overset{\wedge\omega}{\to} H^k(X, \Omega^{i-1})
\overset{\wedge\omega}{\to}H^k(X, \Omega_X^{i})
\overset{\wedge\omega}{\to} H^k(X,\Omega_X^{i+1})
\overset{\wedge\omega}{\to}\cdots\]
is not exact for all $k\geq 0$. Putting these together 
by the Hodge decomposition for $H^k(X,\bbC)$ we get
\begin{equation}
(H^{\bullet}(X,\bbC), \wedge\omega)\coloneqq [\ldots\to H^{i-1}(X,\C)\overset{\wedge\omega}{\longrightarrow}H^{i}(X,\C)\overset{\wedge\omega}{\longrightarrow}H^{i+1}(X,\C)\to\ldots]
\label{eq:resonance}
\end{equation}
is not exact whenever $Z(\omega)\neq \emptyset$. This prompts the following


\begin{definition}[Resonant forms]\label{def:resonance}
We call a 1-form $\omega$ \emph{resonant} if the complex 
in Equation (\ref{eq:resonance}) is not exact. The set of all such forms 
is denoted $\sR(X,\bbC)$.

On the other hand for when the the complex 
in Equation (\ref{eq:resonance}) is exact and\footnote{Note that
with this assumption it is not always the case that $Z(\omega) = \emptyset$. See Example \cite{DJL17}}
$Z(\omega) \neq \emptyset$ then we call such form non-resonant. 

%Moreover a 1-forms $\omega\in V(X)$ is called \emph{universally nonresonant} if the complex
%\newline $(H^{\bullet}(X',\bbC), \wedge\tau^*\omega)$
%%\[\ldots\to H^{i-1}(X',\C)\overset{\wedge\tau^*\omega}{\longrightarrow}H^{i}(X',\C)\overset{\wedge\tau^*\omega}{\longrightarrow}H^{i+1}(X',\C)\to\ldots\]
 %is exact for any \'etale over $\tau\colon X'\to X$. 

We will refer to the sequence $(H^{\bullet}(X,\bbC), \wedge\omega)$ above as the \emph{resonance sequence}.
%Also, we call a holomorphic 1-form to be a resonant 1-form if it is not a nonresonant 1-form and has zeros.
\end{definition}




More generally for any local system we consider a similar situation. 



To make our notations precise recall 
\begin{definition}[Character variety]
$\Char(X) \coloneqq Hom(H_1(X, \bbZ)/\torsion, \bbC^{\star})$
\footnote{This is connected component around the trivial character of the Betti moduli $M_B(X) \coloneqq \Hom(\pi_1(X), \bbC^{\star})$ via the short exact sequence \[0\to \frac{H^1(X, \bbC)}{H^1(X, \bbZ(1))}\to M_B(X)\to H^2(X, \bbZ(1))_{\torsion}\to 0.\] Unlike us, in literature this $M_B(X)$ is often called the character variety $\Char(X)$.} and is isomorphic to
$(\bbC^{\star})^{2q}$ where $q = \dim \Alb(X)$.
\end{definition}
By the Riemann--Hilbert correspondence any such representation $\rho\in \Char(X)$ 
uniquely gives a local system of rank 1. 
Recall also the non-abelian Hodge
correspondence
\[\Char(X) \xrightarrow{\Psi} \Pic^{0}(X)\times H^0(X,\Omega_X^1).\]
mapping a local system $\bbL_{\rho}$ to the line bundle $\sL_{\rho}\simeq \bbL\otimes_{\bbC}\sO_X$
along with the 1-form $\omega_{\rho}$ given by the inverse image of $\bbL$ under
the exponential map
\begin{equation}
H^1(X,\bbC) \xrightarrow{\exp} \Char(X).
\label{eq:exponential}
\end{equation}
More precisely, $\omega_{\rho} = $ (0,1)-piece of $|\log(\rho)|$. Conversely, a pair $(\sL, \omega)$ 
is mapped to the kernel of $\sL\xrightarrow{\partial_{\sL}+\omega} \Omega_X^1\otimes \sL$ under this correspondence.
This induces an isomorphism of topological groups. 

\begin{remark}[Unitary local systems]
Note that a unitary local system given by
\[\eta\in\Char(X)^u \coloneqq \Hom(H_1(X,\bbZ)/\torsion , U(1))\]
corresponds to $(\bbC_{\eta}\otimes_{\bbC}\sO_X, 0)$
under the above non-abelian Hodge correspondence. Thus
we get an embedding 
\[\Pic^0(X) \to \Char(X).\]
This is not a complex submanifold.

\end{remark}

Define the cohomology groups of 
$(\sL, \omega)$ as
\[H^{p,q}(X, (\sL, \omega)) \coloneqq 
\frac{\ker(\omega\colon H^q(X, \Omega_X^p\otimes \sL)
\to H^q(X, \Omega_X^{p+1}\otimes\sL))}{\im(\omega\colon
H^q(X,\Omega_X^{p-1}\otimes\sL)\to H^q(X,\Omega_X^{p}\otimes\sL)}\]
We have the following (see \cite[Theorem 3]{Ara92})
\begin{theorem}
Let $\rho\in \Char(X)$ be the character corresponding to 
$(\sL,\omega)$ then we have
\[H^k(X, \bbC_{\rho}) \simeq \bigoplus_{p+q =k}H^{p,q}(X,\sL,\omega)\]
\label{thm:genhodgedecomp}
\end{theorem}

Note that when $\omega = 0$, i.e.\ $\bbC_{\rho}$ is unitary
this recovers the usual Hodge decomposition for 
unitary local systems corresponding to $\rho \in Hom(\pi_1(X)/\torsion, U(1))$
\[H^p(X, \bbC_{\rho}) \simeq H^p(X, \Omega_X^q\otimes\sL).\]

We now defined
\begin{definition}[generalised resonant 1-forms]
Given a local system $\bbC_{rho}$ associated to a unitary character $\rho$, we define
\[\sR^k(X, \bbC_{\rho}) \coloneqq
\{\omega\in H^0(X, \Omega_X^1)| H^k(H^{\bullet}(X, \bbC_{\rho}), \wedge\omega) \neq 0 .\}\]
We denote as usual 
$\sR(X, \bbC_{\rho}) \coloneqq \bigcup_k \sR^k(X, \bbC_{\rho})$.
\end{definition}

\begin{remark}
From our discussion above it immediately follows that
if $\sL = \Psi(\rho)$ then
\[\sR^k(X, \bbC_{\rho}) =\bigcup_{p+q = k} \{\omega| H^{p,q}(X, \sL,\omega) \neq 0\}.\]
\end{remark}


\subsection{Tangent cones and cohomology jump loci}
\label{sub:tc} Another way to understand the resonant 1-forms is via the tangent cone of the cohomology jump loci defined
as follows.
\begin{definition}
Given a perverse sheaf $\sP$ on $X$, define
\[\sV^i(X,\sP) \coloneqq \{\rho\in\Char(X)|
H^i(X,\sP\otimes \bbC_{\rho})\neq 0\}.\]
\end{definition}
Recall from (\ref{eq:exponential}) that given $\rho\in \Char(X)$, $\omega_{\rho} = $ (0,1)-piece of $\log|\rho|\in H^1(X, \bbC)$. 
%Here we state the relevant results in the relative setting when $X$ admits a map $f\colon X\to A$ 
%to an abelian variety. The main reference for this part is \cite{sch}. We first set 
%\[\sR(f) \coloneqq \{\omega\in H^0(A,\Omega_A^1)| (H^{\bullet}(X, L(f^*\omega)), \wedge f^*\omega) \text{ is exact }\]
%Then $(\sL, \wedge\omega)\in \sM_{Higgs}^0$
%the identity component of the Higgs moduli space. $\sM_{Higgs} \simeq \Pic^{\tau}\times H^0(X,\Omega_X^1)$.
%This is isomorphic to $\Char(X) = Hom(\pi_1(X), \bbC^*)$ as a complex manifold under the map
%\[\Phi\colon \sM_{Higgs} \to \Char(X)\text{ define by } (\sL, \omega) \mapsto L(\omega).\]
By the generic vanishing theory \cite[Theorem 3]{Ara92} we have 
\begin{equation}
\Psi(\sV^i(X,\bbC))
=\bigcup_{p+q = i}\{(\sL,\omega)| H^{p,q}(X,(\sL,\omega)) \neq 0
 \text{ for some } p,q\in\bbZ_{\geq 0}\}.
\label{eq:arapura}
\end{equation}
Therefore we have the following
\begin{proposition} 
Given a character $\rho\in \sV^k(X,\bbC)$,
we identify $TC_{\rho}(\sV^k(X,\bbC))$ as a subspace in the
 Lie algebra
$H^1(X,\bbC)$. Then
\[\bigcup_{\rho\in\sV^k(X,\bbC)}TC_{\rho}(\sV^k(X,\bbC))^{(1,0)} = \bigcup_{\eta\in \Char(X)^{u}}\sR^k(X, \bbC_\eta).\]
Here $TC_{\rho}(\sV^k(X,\bbC))^{(1,0)}$ denotes the holomorphic
piece of $TC_{\rho}(\sV^k(X,\bbC))$.
\label{prop:equivalence}
\end{proposition}
\begin{proof}
The result follows essentially from (\ref{eq:arapura})
above. Indeed, by \ref{thm:genhodgedecomp} $\omega \in \sR^k(X, \bbC_{\eta})$
implies $H^k(X, \Psi^{-1}(\sL, \omega)) \neq 0$
where $\sL\coloneqq \bbC_{\eta}\otimes_{\bbC}\sO_X$.
Similarly, if $\rho\in \sV^k(X, \bbC)$ i.e.\ $H^k(X, \bbC_{\rho})
\neq 0$ then again by \ref{thm:genhodgedecomp} letting
$(\sL, \omega) = \Psi(\rho)$,
we have $\omega\in \sR^k(X, \bbC_{\Psi^{-1}(\sL, 0)})$.

\end{proof}


\subsection{Linearity in generic vanishing theory}
We now define after Simpson \cite[p.\ 365]{Sim93}; the notion of linearity of subsets of 
$\Pic^0(X)\times H^0(X,\Omega_X^1)$. Via the albanese
map $a\colon X\to \Alb_X$ we may and do identify
$\Pic^0(X)\times H^0(X,\Omega_X^1)$ with
$A^{\natural} \coloneqq \Pic^0(A)\times H^0(A,\Omega_A^1)$.
\begin{definition}[Linearity]\label{def:linhiggs}
A subset $Z\subset A^{\natural}$ is said to be \textsl{linear}
or \textsl{translates of triple tori}
if there exists finitely many morphisms of abelian varities
$p_i\colon A\to B_i$ and pairs $(\sL_i,\omega_i)$
such that $Z= (\sL_i,\omega_i)\otimes \im(B^{\natural}
\to A^{\natural})$.
\end{definition}

The notion of linearity has an obvious incarnation for subsets of
 $\Char(A)$ as well. 
\begin{definition}[Linearity]\label{def:char}

\end{definition}

It is a result of Simpson \cite[Theorem 3.1]{Sim93} that a closed algebraic subset $Z \subset Char(A)$ is linear if and only if its image $Z'\subset A^{\natural}$ remains algebraic. 


By the generic vanishing theory \cite{Ara92} and Proposition
\ref{prop:equivalence}
we have the following
\begin{theorem}
The cohomology jump loci $\sV^k(X, \bbC)$ are linear. In particular, $\sR(X)\coloneqq \bigcup_{\eta}\sR(X,\eta)\subset H^0(X,\Omega_X^1)$ is linear, i.e.\ a finite union of linear subspaces.
\label{thm:}
\end{theorem} 

Finally we need similar linearity statement
for perverse sheaves on abelian varieties. 
We collect together further properties of the
cohomology jump loci on an abelian variety $A$ of dimension $q$.
We put appropriate references wherever needed.

\begin{theorem}[Properties of CJL on $A$]
Let $\sP$ be a perverse sheave on an abelian variety $A$ with $\dim A = q$.
The
cohomology jump loci of $\sP$ satisfy the following
\begin{enumerate}
	\item Propagation property:
\[\sV^g(A, \sP) \subseteq \cdots\subseteq \sV^1(A, \sP) 
\subseteq \sV^0(A, \sP) \supseteq \sV^1(A, \sP) \supset\cdots\supset \sV^g(A, \sP).\]
Furthermore, $\sV^i(A, \sP) = \emptyset$, if $i \in [-q, q]$.
\item Codimension lower bound: for any $i \geq 0$,
$\codim \sV^i(A, \sP) \geq 2i$.
\item Generic vanishing: there exists a non-empty Zariski open subset $U \subset \Char(A)$
such that, for any $\rho\in U$, $H^i(A, \sP 
\otimes \sL_{\rho}) = 0$ for all $i \neq 0$.
\item Signed Euler characteristic property:
$\chi(A, P) \geq 0$.
Moreover, the equality holds if and only if $\sV^0(A, \sP) \neq \Char(A)$.
\item Structural property: $\sV^i(A, \sP)$ is a finite union of linear subvarieties of $Char(A)$ for any $i$.
\end{enumerate}
\label{thm:gvperverse}
\end{theorem} 



\subsection{Perverse sheaves and Decomposition theorem}
\begin{definition}[Singular Support]


\end{definition}
\begin{definition}[Characteristic Cycle]


\end{definition}

\begin{theorem}[Kashiwara's index]

\label{thm:indextheorem}
\end{theorem}
We refer the reader to
\cite[Section 4.5]{HTT08 }
Given a morphism of smooth projective varieties, 
$f\colon X\to Y$ then we have the following
\begin{theorem}[Decomposition theorem]


\label{thm:}
\end{theorem}

Given $f\colon X\to A$ a morphism to an abelian
variety, via Kashiwara's estimate the singular support of $\bbR f_*\bbC$ produces a breeding ground for 1-forms with zeros. This phenomenon was exploited in
\cite{PS14}
in showing that all 1-forms admit zeros on smooth projective
varieties of general type. Here we recall the estimate in our
current framework. For the most 
\begin{theorem}[Kashiwara's estimate on the behaviour of singular support]
Given $f\colon X\to A$, consider the following

\label{thm:}
\end{theorem}






\section{Proof of Theorem \ref{thm:smooth} and \ref{thm:cxsmooth}}



















\section{Generic vanishing theory on abelian varieties}
This section is devoted to the proof of Theorem \ref{thm:perverse}. First we note that Theorem \ref{thm:linearity} follows
from Theorem \ref{thm:perverse}


\begin{proof}[Proof of Theorem \ref{thm:linearity}]
By the decomposition Theorem \ref{thm:decomp} applied to $f\colon X\to A$
we have, 
\[Rf_*\bbC\simeq \bigoplus_{\alpha,i} \sP_{\alpha,i}[-i].\]
Here $\sP_{\alpha, i}$ are simple perverse sheaves supported on various strata of the morphism $f$. 
Note that
\[\{\rho\in \Char(A)| H^k(X, f^*\bbC_{\rho})\neq 0 \text{ for some } k\} = \bigcup_{\alpha, i, k}\sV^k(A, \sP_{\alpha, i}).\]
Applying Theorem \ref{thm:perverse} to each 
of these $\sP_{\alpha,i}$ we obtain the result.
\end{proof}


\begin{proof}[Proof of Theorem \ref{thm:pervese}]
By the propagation property of the cohomology jump loci
$\sV^i(A, \sP)$ as in Theorem \ref{thm:cjl}(1) we
only need to consider $\sV^0(A,\sP)$. We split the argument in 
two cases:
\noindent \textbf{Case I: }$\chi(A, \sP)>0$. 
For a general character $\rho$, $H^i(A, sP\otimes \bbC_{\rho})= 0$ for $i\neq 0$ by Theorem \ref{thm:cjl}(3). Hence for a general
$\rho$,
in this case we must have $H^0(A, \sP\otimes \bbC_{\rho})\neq 0$. Hence, 
\[TC_{\rho}\sV^0(A, \sP) = H^1(X,\bbC)\]
On the other hand we have
the Kashiwara index theorem \ref{thm:indextheorem}
that states
\[\chi(A,P) = \langle \CC(\sP), T^*_AA\rangle\]
where $CC(P)$ is the characteristic cycle of $\sP$ as
in Definition \ref{def:cc}.
Note that if $Z\subset A$ is fibered by a subabelian variety $B\subset A$, then $T^*_ZA = T^*_BA$. But $<T^*_BA, T^*_AA> = \chi(B, \bbC) = 0$ (see e.g.\ \cite[p.\ 124]{Dim}).  Therefore,
the $\SS(\sP)$ must contain a subvariety $Z\subset A$ such that
$Z$ is not fibered by tori. 

We conclude by Proposition \ref{van-nonsimple} that when $Z$ is not fibered by tori $\pi(T^*_ZA) = H^0(A,\Omega_A^1)$. Hence
we obtain the desired equality.

\noindent \textbf{Case II: } $\chi(A, \sP)=0$. 
By a result of Weissauer \cite[Theorem 2]{Wei}
we know that there exists an abelian variety $B$ and a surjective morphism $\varphi\colon A\to B$, a simple perverse sheaf $\sP_B$ on $B$, and a local system $\bbL$ on $A$
such that 
\[\sP\otimes \bbL\simeq \varphi^*\sP_B.\]
Since $\sP$ and $\sP\otimes \bbL$ have the same singular support
and upto translation by $\bbL$ the same cohomology jump loci, we 
may and do replace $\sP$ by $\sP\otimes \bbL$.

On the other hand, by a result of
Franecki and Kapranov
\cite[Corollary 1.4]{FK} the formula of
characterstic cucle, i.e\ \[\CC(\sP) = \sum_Z n_Z Z\]
satisfies $n_Z\geq 0$. Hence from Kashiwara's index theorem 
\ref{thm:indextheorem}, one can deduce that all $Z\subset \SS(\sP)$ must be fibered by tori. Similarly, there must be 
$Z$, with $Z' = \varphi(Z)$ not fibered by tori.
Therefore, from the previous case we have 
\[(TC_{\rho}\sV^0(B, \sP_B))^{1,0} = \pi(\SS(\sP_B).\]
We then conclude 
that $\varphi^*(\Char(B))\subset \sV^0(A, \sP)$ from observing that
\[H^0(B, \sP_B\otimes \bbC_{\rho'}) \overset{\oplus}{\into}
H^0(A, \sP\otimes \varphi^*\bbC_{\rho'}).\]
To see the converse note that 
for any $\rho \in \sV^0(A, \sP)$ we must 
have $\bbC_{\rho}|_{\operatorname{Fibre}(\varphi)} = \bbC$.
Indeed, $H^0(A, \sP\otimes \bbC_{\rho}) = H^0(B, \sP_B\otimes \varphi_*\bbC_{\rho})$.
Hence $\varphi_*\bbC_{\rho} = \bbC_{\rho'}$ and $\bbC_{\rho} = 
\varphi^*\bbC_{\rho'}$.







\end{proof}




\bibliographystyle{halpha}
\bibliography{main_new}

%\begin{thebibliography}{Ara92}
%\expandafter\ifx\csname url\endcsname\relax
  %\def\url#1{\texttt{#1}}\fi
%\expandafter\ifx\csname doi\endcsname\relax
  %\def\doi#1{\burlalt{doi:#1}{http://dx.doi.org/#1}}\fi
%\expandafter\ifx\csname urlprefix\endcsname\relax\def\urlprefix{URL }\fi
%\expandafter\ifx\csname href\endcsname\relax
  %\def\href#1#2{#2}\fi
%\expandafter\ifx\csname burlalt\endcsname\relax
  %\def\burlalt#1#2{\href{#2}{#1}}\fi
%
%\bibitem[Ara92]{Ara92}
%Donu Arapura.
%\newblock Higgs line bundles, {G}reen-{L}azarsfeld sets, and maps of
  %{K}\"{a}hler manifolds to curves.
%\newblock {\em Bull. Amer. Math. Soc. (N.S.)} {\bfseries 26}(2), pp.\ 310--314,
  %1992.
%\newblock \doi{10.1090/S0273-0979-1992-00283-5}.
%
%\bibitem[CL73]{CL73}
%James~B. Carrell and David~I. Lieberman.
%\newblock Holomorphic vector fields and {K}aehler manifolds.
%\newblock {\em Invent. Math.} 21, pp.\ 303--309, 1973.
%\newblock \doi{10.1007/BF01418791}.
%
%\bibitem[GL87]{GL87}
%Mark Green and Robert Lazarsfeld.
%\newblock Deformation theory, generic vanishing theorems, and some conjectures
  %of {E}nriques, {C}atanese and {B}eauville.
%\newblock {\em Invent. Math.} {\bfseries 90}(2), pp.\ 389--407, 1987.
%\newblock \doi{10.1007/BF01388711}.
%
%\bibitem[HK05]{HK05}
%Christopher~D. Hacon and S\'{a}ndor~J. Kov\'{a}cs.
%\newblock Holomorphic one-forms on varieties of general type.
%\newblock {\em Ann. Sci. \'{E}cole Norm. Sup. (4)} {\bfseries 38}(4), pp.\
  %599--607, 2005.
%\newblock \doi{10.1016/j.ansens.2004.12.002}.
%
%\bibitem[HS19]{HS19}
%Feng Hao and Stefan Schreieder.
%\newblock Holomorphic one-forms without zeros on threefolds.
%\newblock {\em to appear Geometry $\&$ Topology} , 2019.
%\newblock \urlprefix\url{arXiv:1906.07606}.
%
%\bibitem[LZ05]{LZ05}
%Tie Luo and Qi~Zhang.
%\newblock Holomorphic forms on threefolds.
%\newblock In {\em Recent progress in arithmetic and algebraic geometry}, volume
  %386 of {\em Contemp. Math.}, pages 87--94. Amer. Math. Soc., Providence, RI,
  %2005.
%\newblock \doi{10.1090/conm/386/07219}.
%
%\bibitem[PS14]{PS14}
%Mihnea Popa and Christian Schnell.
%\newblock Kodaira dimension and zeros of holomorphic one-forms.
%\newblock {\em Ann. of Math. (2)} {\bfseries 179}(3), pp.\ 1109--1120, 2014.
%\newblock \doi{10.4007/annals.2014.179.3.6}.
%
%\bibitem[Sch19]{Sch19}
%Stefan Schreieder.
%\newblock Zeros of holomorphic one-forms and topology of k\"ahler manifolds,
  %(appendix written jointly with h.-y. lin).
%\newblock {\em to appear Geometry $\&$ Topology} , 2019.
%\newblock \urlprefix\url{arXiv:1906.07606}.
%
%\end{thebibliography}


%------------------------------------------------------------------
\end{document}
%------------------------------------------------------------------




