\pdfoutput=1
\documentclass[12pt,reqno]{amsart}
\usepackage[letterpaper,margin=1in,footskip=0.25in]{geometry}
\usepackage{mathrsfs}
\usepackage{amssymb}
\usepackage{mathtools}
\usepackage{tikz-cd}
\usepackage{enumitem}

\PassOptionsToPackage{pdfusetitle,pagebackref,colorlinks}{hyperref}
\usepackage{bookmark}
\hypersetup{
  linkcolor={red!70!black},
  citecolor={green!70!black},
  urlcolor={blue!80!black}
}

\newtheorem{theorem}{Theorem}[section]
\newtheorem{lemma}[theorem]{Lemma}
\newtheorem{proposition}[theorem]{Proposition}
\newtheorem{corollary}[theorem]{Corollary}
\newtheorem{claim}[theorem]{Claim}
\newtheorem{conjecture}[theorem]{Conjecture}
\newtheorem{step}{Step}[subsection]
\renewcommand{\thestep}{\arabic{step}}

\newtheorem{alphtheorem}{Theorem}
\renewcommand{\thealphtheorem}{\Alph{alphtheorem}}
\theoremstyle{question}
\newtheorem{question}[theorem]{Question}
\theoremstyle{definition}
\newtheorem{definition}[theorem]{Definition}
\newtheorem{notation}[theorem]{Notation}

\theoremstyle{remark}
\newtheorem{remark}[theorem]{Remark}

\newtheoremstyle{cited}{.5\baselineskip\@plus.2\baselineskip\@minus.2\baselineskip}{.5\baselineskip\@plus.2\baselineskip\@minus.2\baselineskip}{\itshape}{}{\bfseries}{\bfseries .}{5pt plus 1pt minus 1pt}{\thmname{#1}\thmnumber{~#2}\thmnote{ \normalfont#3}}
\theoremstyle{cited}
\newtheorem{citedthm}[theorem]{Theorem}
\newtheorem{citedconj}[theorem]{Conjecture}
\newtheorem{citedlem}[theorem]{Lemma}
\newtheorem{citedprop}[theorem]{Proposition}

\newtheoremstyle{citeddef}{.5\baselineskip\@plus.2\baselineskip\@minus.2\baselineskip}{.5\baselineskip\@plus.2\baselineskip\@minus.2\baselineskip}{}{}{\bfseries}{\bfseries .}{5pt plus 1pt minus 1pt}{\thmname{#1}\thmnumber{~#2}\thmnote{ \normalfont#3}}
\theoremstyle{citeddef}
\newtheorem{citednot}[theorem]{Notation}


%%==============Yongqiang's typsets====================%%

\newcommand{\CN}{\mathbb{C}^{n+1}}
\newcommand{\CP}{\mathbb{CP}^{n+1}}
\newcommand{\U}{\mathcal{U}}
\newcommand{\C}{\mathbb{C}}
\newcommand{\Z}{\mathbb{Z}}
\newcommand{\Hom}{\mathrm{Hom}}
\newcommand{\Q}{\mathbb{Q}}
\newcommand{\K}{\mathcal{L}}
\newcommand{\V}{\mathcal{V}}

\def\be{\begin{equation}}
\def\ee{\end{equation}}

\def\bt{\begin{theorem}}
\def\et{\end{theorem}}

\def\bc{\begin{corollary}}
\def\ec{\end{corollary}}

\def\br{\begin{remark}}
\def\er{\end{remark}}

\def\bp{\begin{proposition}}
\def\ep{\end{proposition}}

\def\bl{\begin{lemma}}
\def\el{\end{lemma}}

%\def\bn{\begin{enumerate}}
%\def\en{\end{enumerate}}

\def\bex{\begin{ex}}
\def\eex{\end{ex}}

\def\bd{\begin{definition}}
\def\ed{\end{definition}}








%\DeclareMathOperator{\Pic}{Pic}                  % Pic

%\newcommand{\PP}{\mathcal{P}}           % Poincare line bunle

\DeclareMathOperator{\Supp}{Supp}                % Supp
\DeclareMathOperator{\codim}{codim}              % codim
\DeclareMathOperator{\mreg}{mreg}                % mreg
\DeclareMathOperator{\reg}{reg}                  % reg
\DeclareMathOperator{\sing}{sing}                  
\DeclareMathOperator{\id}{id}                    % id
\DeclareMathOperator{\obj}{Obj}
\DeclareMathOperator{\ad}{ad}
\DeclareMathOperator{\morph}{Morph}
\DeclareMathOperator{\enom}{End}
\DeclareMathOperator{\iso}{Iso}
\DeclareMathOperator{\Exp}{Exp}
\DeclareMathOperator{\homo}{Hom}
\DeclareMathOperator{\enmo}{End}
\DeclareMathOperator{\spec}{Spec}
\DeclareMathOperator{\fitt}{Fitt}
\DeclareMathOperator{\odr}{\Omega^\bullet_{\textrm{DR}}}
\DeclareMathOperator{\rank}{Rank}
\DeclareMathOperator{\gdeg}{gdeg}
\DeclareMathOperator{\Alb}{Alb}
\DeclareMathOperator{\alb}{alb}
\DeclareMathOperator{\Ann}{Ann}



%\DeclareMathOperator

\DeclareMathOperator{\Char}{Char}
\DeclareMathOperator{\CC}{SS}
\DeclareMathOperator{\kn}{Ker}
\DeclareMathOperator{\im}{Im}


\def\ra{\rightarrow}


\def\bone{\mathbf{1}}
\def\bC{\mathbb{C}}
\def\cM{\mathcal{M}}
\def\cV{\mathcal{V}}
\def\Def{{\rm {Def}}}
\def\cR{\mathcal{R}}
\def\om{\omega}
\def\wti{\widetilde}
\def\al{\alpha}
\def\End{{\rm {End}}}
\def\Pic{{\rm Pic}}
\def\bP{\mathbb{P}}
\def\cH{\mathcal{H}}
\def\bL{\mathbb{L}}
\def\cX{\mathcal{X}}
\def\cI{\mathcal{I}}
\def\pa{\partial}
\def\cY{\mathcal{Y}}
\def\cD{\mathcal{D}}
\def\cO{\mathcal{O}}
\def\lra{\longrightarrow}
\def\bQ{\mathbb{Q}}
\def\ol{\overline}
\def\cL{\mathcal{L}}
\def\bH{\mathbb{H}}
\def\bZ{\mathbb{Z}}
\def\bW{\mathbf{W}}
\def\bV{\mathbf{V}}
\def\bM{\mathbf{M}}
\def\eps{\epsilon}
\def\ul{\underline}
\def\lam{\lambda}
\def\sX{\mathscr{X}}
\def\bN{\mathbb{N}}


%%==========Yagna's typsets============%%
\usepackage{tikz-cd}
\usepackage{enumitem}

\PassOptionsToPackage{pdfusetitle,pagebackref,colorlinks}{hyperref}
\usepackage{bookmark}
\hypersetup{
  linkcolor={red!70!black},
  citecolor={green!70!black},
  urlcolor={blue!80!black}
}

%Mathcal Letters =====================
\newcommand{\sA}{\mathcal{A}}
\newcommand{\sB}{\mathcal{B}}
\newcommand{\sD}{\mathcal{D}}
\newcommand{\sF}{\mathcal{F}}
\newcommand{\sG}{\mathcal{G}}
\newcommand{\sH}{\mathcal{H}}
\newcommand{\sK}{\mathcal{K}}
\newcommand{\sL}{\mathcal{L}}
\newcommand{\sM}{\mathcal{M}}
\newcommand{\sN}{\mathcal{N}}
\newcommand{\sO}{\mathcal{O}}
\newcommand{\sP}{\mathcal{P}}
\newcommand{\sQ}{\mathcal{Q}}
\newcommand{\sR}{\mathcal{R}}
\newcommand{\sT}{\mathcal{T}}
\newcommand\sV{{\mathcal V}}
\newcommand\sW{{\mathcal W}}
\newcommand{\sZ}{\mathcal{Z}}

%mathbb Letters======
\newcommand{\bbA}{\mathbb{A}}
\newcommand{\bbB}{\mathbb{B}}
\newcommand{\bbC}{\mathbb{C}}
\newcommand{\bbG}{\mathbb{G}}
\newcommand{\bbH}{\mathbb{H}}
\newcommand{\bbK}{\mathbb{K}}
\newcommand{\bbL}{\mathbb{L}}
\newcommand{\bbM}{\mathbb{M}}
\newcommand{\bbN}{\mathbb{N}}
\newcommand{\bbP}{\mathbb{P}}
\newcommand{\bbQ}{\mathbb{Q}}
\newcommand{\bbR}{\mathbb{R}}
\newcommand{\bbV}{\mathbb{V}}
\newcommand{\bbZ}{\mathbb{Z}}



%Script Letters ======================
\newcommand{\frf}{\mathfrak{f}}
\newcommand{\frM}{\mathfrak{M}}

\newcommand{\scrL}{\mathscr{L}}
\newcommand{\crI}{\mathscr{I}}
\newcommand{\scrK}{\mathscr{K}}
\newcommand{\scrB}{\mathscr{B}}
\newcommand{\scrC}{\mathscr{C}}
\newcommand{\scrD}{\mathscr{D}}
\newcommand{\scrE}{\mathscr{E}}
\newcommand{\scrI}{\mathscr{I}}
\newcommand{\scrQ}{\mathscr{Q}}
\newcommand{\scrR}{\mathscr{R}}
\newcommand{\scrX}{\mathscr{X}}
\newcommand{\scrY}{\mathscr{Y}}
\newcommand{\scrF}{\mathscr{F}}
\newcommand{\Dred}{\lceil D\rceil}

%Arrow Style =========================
\newcommand{\into}{\hookrightarrow}
\newcommand{\onto}{\rightarrow\hspace*{-.14in}\rightarrow}

%Math Operators ======================



%\DeclareMathOperator{\alg}{alg}
\DeclareMathOperator{\BM}{BM}
\DeclareMathOperator{\Bs}{Bs}
\DeclareMathOperator{\Bsp}{\mathbf{B}_+}
\DeclareMathOperator{\SB}{\mathbf{B}}

\DeclareMathOperator{\coh}{coh}
\DeclareMathOperator{\Coker}{Coker}
 \renewcommand{\div}{\text{div}}

\DeclareMathOperator{\DR}{DR}
\DeclareMathOperator{\Ch}{Ch}
\DeclareMathOperator{\discrep}{discrep}
\DeclareMathOperator{\exc}{exc}

\DeclareMathOperator{\free}{free}


\DeclareMathOperator{\HHom}{\mathcal{H}\!\mathit{om}}
\DeclareMathOperator{\image}{Im}


\DeclareMathOperator{\Ind}{Ind}
\DeclareMathOperator{\Ker}{Ker}
\DeclareMathOperator{\Lie}{Lie}
\DeclareMathOperator{\op}{op}


\DeclareMathOperator{\qcoh}{qcoh}

%\DeclareMathOperator{\reg}{reg}

\DeclareMathOperator{\RHHom}{\mathbf{R}\mathcal{H}\!\mathit{om}}


\DeclareMathOperator{\Spf}{Spf}
%\DeclareMathOperator{\Supp}{Supp}
\DeclareMathOperator{\tors}{tors}
\DeclareMathOperator{\torsion}{torsion}
\DeclareMathOperator{\Var}{Var}
%\DeclareMathOperator{\Sym}{Sym}


\newcommand{\Ab}{\mathbf{Ab}}
\newcommand{\Aff}{\mathbf{Aff}}
\newcommand{\tB}{\tilde{B}}
%\newcommand{\et}{{\acute{e}t}}

\newcommand{\Sets}{\mathbf{Sets}}
%\newcommand{\cX}{\bar{X}}
\newcommand{\cP}{\bar{P}}
%\newcommand{\cV}{\bar{V}}
\newcommand{\cW}{\bar{W}}
\newcommand{\sorry}[1]{\textcolor{red}{#1}}
\newcommand{\dual}[1]{\sD^{\Omega}_{M^{#1}}}



\title{}








\begin{document}  
\title[Cohomology jump loci and holomorphic 1-forms with zeros]{On linearity of holomorphic 1-forms with zeros} 

\author{Yajnaseni Dutta}

%\address{}
%\email{}

\author{Feng Hao}

%\address{}
%\email{}

\author{Yongqiang Liu}

%\address{}
%\email{}


%\date{\today}
%\subjclass[2010]{} 
%\keywords{} 



\begin{abstract} 
The goal of this article is two-fold, first we 
\end{abstract}

\maketitle
\section{Introduction}\label{intro}
Given a smooth porjective variety $X$, in this article we relate the cohomology jump loci of $\bbC_X$ with the 
stratification arising in the decomposition theorem for the albnese morphism. Such relation fell out of our interest in the
study of holomorphic 1-forms. Interestingly, it has been indicated by plethora of results
(\cite{GL87, HK05, LZ05,
SS19, HS19, PS14} to name just a few) that a lot of the geometry and topology of the
variety depends on vanishing of such forms. We state our first result in this direction
and then put it in the context of the existing vast literature.



\begin{alphtheorem}\label{thm:smooth}
Let $f:X\to A$ be a morphism from a smooth projective variety to a simple abelian variety $A$. Then $f$ is cohomologically a fiber bundle if and only if there is a global holomorphic 1-form $\omega$ on $A$ such that $f^*\omega$ is nowhere vanishing.  In particular, the
singular fibres of $f$ carry a pure Hodge structure.
\end{alphtheorem}

By a \textsl{cohomological fiber bundle} we mean that the higher direct
images $\bbR^if_*\bbC$ are integrable connections for all $i$. In particular, the derived direct image
$\bbR f_*\bbC$ decomposes like the smooth morphism in the derived category $D^b(\bbC)$. In fact a bit more can be said; When $A$ is not necessarily simple existence of a non-vanishing global holomorphic 1-form
is equivalent to certain restrictions on the singular support 
(see Definition \ref{def:ss}) of 
these pushforwards. See Theorem \ref{thm:nonvanishing} for more details.

\begin{remark}[Previous results]
\begin{enumerate}
\item \label{item:ps} When $X$ is of general type, it was conjectured in
	\cite{HK05, LZ05} and was proved
	by Popa and Schnell \cite{PS14} every global holomorphic 1-		
	form vanishes on $X$. 
\item \label{item:hk} On the opposite extreme for any $A$, not necessarily simple when $f$ is singular along a divisor of general
		type in $A$, Hacon and Kov\'acs \cite[Proposition 3.5.]{HK05} show that
		$f^*\omega$ always admits zero. See Corollary \ref{cor:hk}
		for a reinterpretation of their argument. 
\end{enumerate}	
\end{remark}

This result is an easy consequence of Kashiwara's estimate for character variety together with the decomposition of $\bbR f_*\bbC$ in $D^b_{A}(\bbC)$. We began studying this pair of results with the
hope of showing linearity of the set of global holomorphic 1-forms
of smooth projective varieties much like the work of   Carrell and Lieberman \cite{CL73}
in case of the global holomorphic tangent vector fields with zeros. It is no surprise that for 1-forms the story is deeply connected to the generic
vanishing theory. The question we ask is the following:
\begin{question}
Is the following set
\[V(X):=\{ \omega\in H^0(X, \Omega_X^1) | Z(\omega)\neq \emptyset\}\]
os holomorphic 1-forms admitting zeros linear, i.e.\ a finite
union of linear subspaces of the vector space $H^0(X, \Omega_X^1)$.
\end{question} 

The answer to this question is yes up-to 3-folds by
a result of the second author obtained jointly with Schreieder
\sorry{insert theorem reference}. Moreover,
generic vanishing reasons the answer is also yes 
when $\chi(X)>0$ \sorry{why not $\chi(X)\neq 0$}. Furthermore
we show the following

\begin{alphtheorem}
Let $f: X\to A$ be a morphism from a smooth complex projective variety $X$ to an abelian variety $A$. Then 
\[V^1(f) \coloneqq \{\omega\in H^0(X,\Omega_X^1)| \codim Z(\omega) \geq 1\}\]
 is linear. In particular, $V^1(X)$ is linear.
\end{alphtheorem}

More generally we expect the following to be true.
\begin{conjecture} \label{linear-v1}
Let $f: X\to A$ be a morphism from a smooth complex projective variety $X$ to an abelian variety $A$. Then 
\[V^i(f)\coloneqq \{\omega\in H^0(X,\Omega_X^1)| \codim Z(\omega) \geq i\}\] are linear subvarieties of $H^0(X, \Omega_X^1)$, i.e., $V^i(X)$ is linear for each nonnegative integer $i$. In particular $V^i(X)$ are linear.
\end{conjecture}
In fact we show a similar result in the quasi-projective setting
(see \ref{log-linear-v1}). 


It is known due to the generic vanishing theory
%theory of cohomology jumploci 
that only a piece of $V(X)$, arising from the generic vanishing theory (see \S \ref{se:gv} for more details) is linear, i.e.\
%\emph{the resonant 1-forms} (see Definition \ref{def:resonance} below) 
this set forms a finite union of subvector spaces inside $H^0(X, \Omega_X^1)$. To be more specific, this set is given by
\cite[p.\ 311]{Ara92}
\begin{equation}
\begin{split}
\sR(X)\coloneqq \bigcup_i\{\omega\in H^0(X, \Omega_X^1)| \exists \sL\in \Pic^0(X) \text{ and } p\in \bbZ_{\geq 0} \text{ such that},
H^q(H^{p}(X, \Omega_X^{\bullet}\otimes \sL), \wedge\omega) \neq 0
\\\text{ for all } p+q=i\}
\label{eq:}
\end{split}
\end{equation}
This set is also the holomorphic piece of the tangent cone 
(see \ref{def:tc} for a definition) to the cohomology jump loci 
\[\sV^i(X,\bbC) \coloneqq \{\rho\in\Char(X)| \bbH^i(X, \bbC_{\rho})\neq  0\}\subseteq \Char(X) \coloneqq Hom(H_1(X,\bbZ)/\torsion, \bbC^{\star})\]
where $\bbC_{\rho}$ is the local system associated $\rho$
under the Riemann--Hilbert correspondence.
%\cite[Theorem 4.2]{Sim93} (see also \cite{DiPa13} from where
%we borrow the terminology). 
%Henceforth we shall call finite unions of subvector spaces as \emph{linear subvarieties}. 
%In the relative setting
%of a morphism $f\colon X\to A$ from a smooth projective variety to an abelian variety $A$,
%we similarly define
%\[V(f):=\{ \omega\in H^0(A, \Omega_A^1) | Z(f^*\omega)\neq \emptyset\}.\]
On the other hand by Kashiwara's estimate
of character varieties, it is known that given a proper morphism
$f\colon X\to A$ to an abelian variety $A$, we have
\[SS(Rf_*\bbC) \subseteq f_*(df^{-1}(0_X))\subset T^*A\]
where $SS(Rf_*\bbC)$ is the singular support of the constructible
sheaf $Rf_*\bbC$ (see Definition \ref{def:ss}) and $0_X$ is the
zero section $T^*_XX$. Furthermore, $f_*(df^{-1}(0_X))$ is
defined via the following diagram:
\begin{equation}
\begin{tikzcd}
A\times H^0(A,\Omega_A^1)  \ar[dr]
& X\times H^0(A, \Omega_A^1)\ar[r, "df"']\ar[d] \ar[l, "f\times id"]
& T^*X \ar[d]\\
&A  & X\ar[l, "f"]
\end{tikzcd}
\label{eq:maindiagram}
\end{equation}
Note that under the projection $\pi\colon A\times H^0(A,\Omega_A^1)\to H^0(A,\Omega_A^1)$ we have
\[\pi(f_*(df^{-1}(0_X))) = V(f) \coloneqq \{\omega\in H^0(A,\Omega_A^1)|
Z(f^*\omega) \neq \emptyset\}.\]
Therefore, when $f$ is the albanese morphism, we obtain
\[SS(Rf_*\bbC)\subseteq V(X).\]
We show in Proposition \ref{prop:vansimple} 
that $SS(Rf_*\bbC)$ is a finite union of linear subspaces of $H^0(A,\Omega_A^1)$. We
have the following
\begin{alphtheorem}
Let $a\colon X\to \Alb_X$ denotes the albanese morphism. Then
\[\sR(X) =  \SS(Ra_*\bbC). \]
\label{thm:linearity}
\end{alphtheorem}


The theorem follows from a more general statement
about perverse sheaves on abelian varieties. Recall
\begin{definition}
Associated to $\rho\in \Char(X)$ the collection of local systems $\bbC_{\rho}$ defined as
\[\sV^i(A, \sP) \coloneqq \{\rho\in\Char(X)| H^i(A, \sP\otimes \bbC_{\rho}) \neq 0\}\]
is called the cohomology jump loci of $\sP$.
\label{def:cjl}
\end{definition}
\begin{alphtheorem}
Let $A$ be an abelian variety. Let $P$ be a complex of perverse sheaves with complex coefficient on $A$. 
Then we have the equality that
$$\pi(CC(P)) = H^0(X, \Omega_X^1)\cap \bigcup_{\rho} \rho^{-1} TC_{\rho} \sV^0(A,\sP), $$
where the union is running over a representative point from every irreducible components of $\sV^0(A,P)$
$TC_{\rho} \sV^0(A,P) \subseteq H^1(X, \bbC)$ denotes the tangent cone at $\rho$. 
\label{thm:perverse}
\end{alphtheorem}

For an
interpretation of these results in the category of regular holonomic D-modules
see Theorem \ref{thm:linearitydmodule}. The key technique we
use to relate the the generic vanishing theory on abelian varieties with the support of $Rf_*\bbC$ is Kashiwara's
global index theorem. 


\begin{remark}
Note that it is not clear a priori why $\sR(X)\subseteq V(X)$. However, it is a consequence of our Theorem. Although this was known by a special case of an earlier result of Budur--Wang--Yoon
\cite{BWY}. However,
We do know at the moment whether the equality holds in $\sR(X) \subseteq V(X)$. 
\end{remark}
%An example of Debarre, Jiang and Lahoz 
%\cite[Example 1.11]{DJL17} shows that the
%there exists a bi-elliptic surface $S$ admitting a 1-form 
%$\omega$ for which $(H^*(X, \mathbb{C}), \wedge \omega)$
%is exact, yet $\omega$ admits zeros on $S$. Theorem \ref{thm:linearity} shows that such forms are not under the realm of generic vanishing theory. 

\begin{remark}
Note that if $\chi(X)>0$, we have $\sR(X) = V(X) = H^0(X, \Omega_X^1)$. Hence in this case the set of 1-forms that admit zeros is linear as already mentioned above. 
This follows immediately from the generic vanishing theory. Indeed, by Hodge decomposition
we have
$H^k(X,\bbC) \simeq \bigoplus_{p+q = k} H^p(X,\Omega_X^q)$. Hence the complexes $(H^p(X, \Omega_X^{\bullet}), \wedge\omega)$ can be summed together to form the complex $(H^p(X, \bbC), \wedge\omega)$. Since $\chi(X)>0$ the latter complex cannot be exact. Hence $H^0(X,\Omega_X^1)
= \sR(X)$.
\end{remark}










In this paper, all complex of sheaves and perverse sheaves are defined with complex
coeffcients. All the varieties are complex quasi-projective varieties. 

\subsection*{Acknowledgements}



\section{Preliminary}
\subsection{Resonant 1-forms and Hodge decomposition}
Given a holomorphic 1-form $\omega\in H^0(X,\Omega_X^1)$
the kernel of the associated Koszul complex
\begin{equation}
\sK^{\bullet}_{\omega} \coloneqq [\sO_X\overset{\wedge\omega}{\to} \Omega_X^1 \to \cdots\to \Omega_X^{n-1}\to \Omega^n_X.]
\label{eq:koszul}
\end{equation}
defines a rank 1 local system $L(\omega)$ (see \cite[\S 2.1]{sch}
for a construction). This gives in the way of the generic vanishing theory developed by \cite{GL, Ara, Sim} into the 
study of zeros of holomorphic 1-forms. In this section we discuss 
the relevant bits of this vast theory that we will use in various
proofs.
%
%\begin{definition}[Zero scheme of 1-forms]\label{def:zeroscheme}
%For $\omega\in H^0(X, \Omega_X^1)$, the \emph{zero set} $Z(\omega)$ of $\omega$ is the algebraic set of closed point $x$ in $X$, such that $\omega(v)=0$
%for all tangent vectors $v\in T_xX$ at $x$. 
%
%The \emph{zero scheme} of $\omega$ is the closed subscheme $\sZ(\omega)$ defined by the ideal sheaf $\mathcal{I}_{\omega}$ given by the image of the morphism 
%\[\mathcal{T}_X\overset{\langle\omega, \cdot\rangle}{\longrightarrow} \mathcal{O}_X.\] 
%Here $\mathcal{T}_X$ is the tangent sheaf of $X$ and $\langle\omega, \cdot\rangle$ denotes the pairing of tangent field with 
%the 1-form $\omega$.
%\end{definition}

 
The generic vanishing theory 
\cite[Proposition 3.4]{GL} ensures that
if $Z(\omega)\neq \emptyset$ then the sequence
\[\cdots\overset{\wedge\omega}{\to} H^k(X, \Omega^{i-1})
\overset{\wedge\omega}{\to}H^k(X, \Omega_X^{i})
\overset{\wedge\omega}{\to} H^k(X,\Omega_X^{i+1})
\overset{\wedge\omega}{\to}\cdots\]
is not exact for all $k\geq 0$. Putting these together 
by the Hodge decomposition for $H^k(X,\bbC)$ we get
\begin{equation}
(H^{\bullet}(X,\bbC), \wedge\omega)\coloneqq [\ldots\to H^{i-1}(X,\C)\overset{\wedge\omega}{\longrightarrow}H^{i}(X,\C)\overset{\wedge\omega}{\longrightarrow}H^{i+1}(X,\C)\to\ldots]
\label{eq:resonance}
\end{equation}
is not exact whenever $Z(\omega)\neq \emptyset$. This prompts the following


\begin{definition}[Resonant forms]\label{def:resonance}
We call a 1-form $\omega$ \emph{resonant} if the complex 
in Equation (\ref{eq:resonance}) is not exact. The set of all such forms 
is denoted $\sR(X,\bbC)$.

On the other hand for when the the complex 
in Equation (\ref{eq:resonance}) is exact and\footnote{Note that
with this assumption it is not always the case that $Z(\omega) = \emptyset$. See Example \ref{ex:DJL}.} 
$Z(\omega) \neq \emptyset$ then we call such form non-resonant. 

%Moreover a 1-forms $\omega\in V(X)$ is called \emph{universally nonresonant} if the complex
%\newline $(H^{\bullet}(X',\bbC), \wedge\tau^*\omega)$
%%\[\ldots\to H^{i-1}(X',\C)\overset{\wedge\tau^*\omega}{\longrightarrow}H^{i}(X',\C)\overset{\wedge\tau^*\omega}{\longrightarrow}H^{i+1}(X',\C)\to\ldots\]
 %is exact for any \'etale over $\tau\colon X'\to X$. 

We will refer to the sequence $(H^{\bullet}(X,\bbC), \wedge\omega)$ above as the \emph{resonance sequence}.
%Also, we call a holomorphic 1-form to be a resonant 1-form if it is not a nonresonant 1-form and has zeros.
\end{definition}




More generally for any local system we consider a similar situation. 



To make our notations precise recall 
\begin{definition}[Character variety]
$\Char(X) \coloneqq Hom(H_1(X, \bbZ)/\torsion, \bbC^{\star})$
\footnote{This is connected component around the trivial character of the Betti moduli $M_B(X) \coloneqq \Hom(\pi_1(X), \bbC^{\star})$ via the short exact sequence \[0\to \frac{H^1(X, \bbC)}{H^1(X, \bbZ(1))}\to M_B(X)\to H^2(X, \bbZ(1))_{\torsion}\to 0.\] Unlike us, in literature this $M_B(X)$ is often called the character variety $\Char(X)$.} and is isomorphic to
$(\bbC^{\star})^{2q}$ where $q = \dim \Alb(X)$.
\end{definition}
By the Riemann--Hilbert correspondence any such representation $\rho\in \Char(X)$ 
uniquely gives a local system of rank 1. 
Recall also the non-abelian Hodge
correspondence
\[\Char(X) \xrightarrow{\Psi} \Pic^{0}(X)\times H^0(X,\Omega_X^1).\]
mapping a local system $\bbL_{\rho}$ to the line bundle $\sL_{\rho}\simeq \bbL\otimes_{\bbC}\sO_X$
along with the 1-form $\omega_{\rho}$ given by the inverse image of $\bbL$ under
the exponential map
\begin{equation}
H^1(X,\bbC) \xrightarrow{\exp} \Char(X).
\label{eq:exponential}
\end{equation}
More precisely, $\omega_{\rho} = $ (0,1)-piece of $|\log(\rho)|$. Conversely, a pair $(\sL, \omega)$ 
is mapped to the kernel of $\sL\xrightarrow{\partial_{\sL}+\omega} \Omega_X^1\otimes \sL$ under this correspondence.
This induces an isomorphism of topological groups. 

\begin{remark}[Unitary local systems]
Note that a unitary local system given by
\[\eta\in\Char(X)^u \coloneqq \Hom(H_1(X,\bbZ)/\torsion , U(1))\]
corresponds to $(\bbC_{\eta}\otimes_{\bbC}\sO_X, 0)$
under the above non-abelian Hodge correspondence. Thus
we get an embedding 
\[\Pic^0(X) \to \Char(X).\]
This is not a complex submanifold.

\end{remark}

Define the cohomology groups of 
$(\sL, \omega)$ as
\[H^{p,q}(X, (\sL, \omega)) \coloneqq 
\frac{\ker(\omega\colon H^q(X, \Omega_X^p\otimes \sL)
\to H^q(X, \Omega_X^{p+1}\otimes\sL))}{\im(\omega\colon
H^q(X,\Omega_X^{p-1}\otimes\sL)\to H^q(X,\Omega_X^{p}\otimes\sL)}\]
We have the following (see \cite[Theorem 3]{Ara92})
\begin{theorem}
Let $\rho\in \Char(X)$ be the character corresponding to 
$(\sL,\omega)$ then we have
\[H^k(X, \bbC_{\rho}) \simeq \bigoplus_{p+q =k}H^{p,q}(X,\sL,\omega)\]
\label{thm:genhodgedecomp}
\end{theorem}

Note that when $\omega = 0$, i.e.\ $\bbC_{\rho}$ is unitary
this recovers the usual Hodge decomposition for 
unitary local systems corresponding to $\rho \in Hom(\pi_1(X)/\torsion, U(1))$
\[H^p(X, \bbC_{\rho}) \simeq H^p(X, \Omega_X^q\otimes\sL).\]

We now defined
\begin{definition}[generalised resonant 1-forms]
Given a local system $\bbC_{rho}$ associated to a unitary character $\rho$, we define
\[\sR^k(X, \bbC_{\rho}) \coloneqq
\{\omega\in H^0(X, \Omega_X^1)| H^k(H^{\bullet}(X, \bbC_{\rho}), \wedge\omega) \neq 0 .\}\]
We denote as usual 
$\sR(X, \bbC_{\rho}) \coloneqq \bigcup_k \sR^k(X, \bbC_{\rho})$.
\end{definition}

\begin{remark}
From our discussion above it immediately follows that
if $\sL = \Psi(\rho)$ then
\[\sR^k(X, \bbC_{\rho}) =\bigcup_{p+q = k} \{\omega| H^{p,q}(X, \sL,\omega) \neq 0\}.\]
\end{remark}


\subsection{Tangent cones and cohomology jump loci}
\label{sub:tc} Another way to understand the resonant 1-forms is via the tangent cone of the cohomology jump loci defined
as follows.
\begin{definition}
Given a perverse sheaf $\sP$ on $X$, define
\[\sV^i(X,\sP) \coloneqq \{\rho\in\Char(X)|
H^i(X,\sP\otimes \bbC_{\rho})\neq 0\}.\]
\end{definition}
Recall from (\ref{eq:exponential}) that given $\rho\in \Char(X)$, $\omega_{\rho} = $ (0,1)-piece of $\log|\rho|\in H^1(X, \bbC)$. 
%Here we state the relevant results in the relative setting when $X$ admits a map $f\colon X\to A$ 
%to an abelian variety. The main reference for this part is \cite{sch}. We first set 
%\[\sR(f) \coloneqq \{\omega\in H^0(A,\Omega_A^1)| (H^{\bullet}(X, L(f^*\omega)), \wedge f^*\omega) \text{ is exact }\]
%Then $(\sL, \wedge\omega)\in \sM_{Higgs}^0$
%the identity component of the Higgs moduli space. $\sM_{Higgs} \simeq \Pic^{\tau}\times H^0(X,\Omega_X^1)$.
%This is isomorphic to $\Char(X) = Hom(\pi_1(X), \bbC^*)$ as a complex manifold under the map
%\[\Phi\colon \sM_{Higgs} \to \Char(X)\text{ define by } (\sL, \omega) \mapsto L(\omega).\]
By the generic vanishing theory \cite[Theorem 3]{Ara92} we have 
\begin{equation}
\Psi(\sV^i(X,\bbC))
=\bigcup_{p+q = i}\{(\sL,\omega)| H^{p,q}(X,(\sL,\omega)) \neq 0
 \text{ for some } p,q\in\bbZ_{\geq 0}\}.
\label{eq:arapura}
\end{equation}
Therefore we have the following
\begin{proposition} 
Given a character $\rho\in \sV^k(X,\bbC)$,
we identify $TC_{\rho}(\sV^k(X,\bbC))$ as a subspace in the
 Lie algebra
$H^1(X,\bbC)$. Then
\[\bigcup_{\rho\in\sV^k(X,\bbC)}TC_{\rho}(\sV^k(X,\bbC))^{(1,0)} = \bigcup_{\eta\in \Char(X)^{u}}\sR^k(X, \bbC_\eta).\]
Here $TC_{\rho}(\sV^k(X,\bbC))^{(1,0)}$ denotes the holomorphic
piece of $TC_{\rho}(\sV^k(X,\bbC))$.
\label{prop:equivalence}
\end{proposition}
\begin{proof}
The result follows essentially from (\ref{eq:arapura})
above. Indeed, by \ref{thm:genhodgedecomp} $\omega \in \sR^k(X, \bbC_{\eta})$
implies $H^k(X, \Psi^{-1}(\sL, \omega)) \neq 0$
where $\sL\coloneqq \bbC_{\eta}\otimes_{\bbC}\sO_X$.
Similarly, if $\rho\in \sV^k(X, \bbC)$ i.e.\ $H^k(X, \bbC_{\rho})
\neq 0$ then again by \ref{thm:genhodgedecomp} letting
$(\sL, \omega) = \Psi(\rho)$,
we have $\omega\in \sR^k(X, \bbC_{\Psi^{-1}(\sL, 0)})$.

\end{proof}


\subsection{Linearity in generic vanishing theory}
We now define after Simpson \cite[p.\ 365]{Sim93}; the notion of linearity of subsets of 
$\Pic^0(X)\times H^0(X,\Omega_X^1)$. Via the albanese
map $a\colon X\to \Alb_X$ we may and do identify
$\Pic^0(X)\times H^0(X,\Omega_X^1)$ with
$A^{\natural} \coloneqq \Pic^0(A)\times H^0(A,\Omega_A^1)$.
\begin{definition}[Linearity]\label{def:linhiggs}
A subset $Z\subset A^{\natural}$ is said to be \textsl{linear}
or \textsl{translates of triple tori}
if there exists finitely many morphisms of abelian varities
$p_i\colon A\to B_i$ and pairs $(\sL_i,\omega_i)$
such that $Z= (\sL_i,\omega_i)\otimes \im(B^{\natural}
\to A^{\natural})$.
\end{definition}

The notion of linearity has an obvious incarnation for subsets of
 $\Char(A)$ as well. 
\begin{definition}[Linearity]\label{def:char}

\end{definition}

It is a result of Simpson \cite[Theorem 3.1]{Sim93} that a closed algebraic subset $Z \subset Char(A)$ is linear if and only if its image $Z'\subset A^{\natural}$ remains algebraic. 


By the generic vanishing theory \cite{Ara92} and Proposition
\ref{prop:equivalence}
we have the following
\begin{theorem}
The cohomology jump loci $\sV^k(X, \bbC)$ are linear. In particular, $\sR(X)\coloneqq \bigcup_{\eta}\sR(X,\eta)\subset H^0(X,\Omega_X^1)$ is linear, i.e.\ a finite union of linear subspaces.
\label{thm:}
\end{theorem} 

Finally we need similar linearity statement
for perverse sheaves on abelian varieties. 
We collect together further properties of the
cohomology jump loci on an abelian variety $A$ of dimension $q$.
We put appropriate references wherever needed.

\begin{theorem}[Properties of CJL on $A$]
Let $\sP$ be a perverse sheave on an abelian variety $A$ with $\dim A = q$.
The
cohomology jump loci of $\sP$ satisfy the following
\begin{enumerate}
	\item Propagation property:
\[\sV^g(A, \sP) \subseteq \cdots\subseteq \sV^1(A, \sP) 
\subseteq \sV^0(A, \sP) \supseteq \sV^1(A, \sP) \supset\cdots\supset \sV^g(A, \sP).\]
Furthermore, $\sV^i(A, \sP) = \emptyset$, if $i \in [-q, q]$.
\item Codimension lower bound: for any $i \geq 0$,
$\codim \sV^i(A, \sP) \geq 2i$.
\item Generic vanishing: there exists a non-empty Zariski open subset $U \subset \Char(A)$
such that, for any $\rho\in U$, $H^i(A, \sP 
\otimes \sL_{\rho}) = 0$ for all $i \neq 0$.
\item Signed Euler characteristic property:
$\chi(A, P) \geq 0$.
Moreover, the equality holds if and only if $\sV^0(A, \sP) \neq \Char(A)$.
\item Structural property: $\sV^i(A, \sP)$ is a finite union of linear subvarieties of $Char(A)$ for any $i$.
\end{enumerate}
\label{thm:gvperverse}
\end{theorem} 



\subsection{Perverse sheaves and Decomposition theorem}
\begin{definition}[Singular Support]


\end{definition}
\begin{definition}[Characteristic Cycle]


\end{definition}

\begin{theorem}[Kashiwara's index]

\label{thm:indextheorem}
\end{theorem}
We refer the reader to
\cite[Section 4.5]{HTT08 }
Given a morphism of smooth projective varieties, 
$f\colon X\to Y$ then we have the following
\begin{theorem}[Decomposition theorem]


\label{thm:}
\end{theorem}



\section{Generic vanishing theory on abelian varieties}
This section is devoted to the proof of Theorem \ref{thm:perverse}. First we note that Theorem \ref{thm:linearity} follows
from Theorem \ref{thm:perverse}


\begin{proof}[Proof of Theorem \ref{thm:linearity}]
By the decomposition Theorem \ref{thm:decomp} applied to $f\colon X\to A$
we have, 
\[Rf_*\bbC\simeq \bigoplus_{\alpha,i} \sP_{\alpha,i}[-i].\]
Here $\sP_{\alpha, i}$ are simple perverse sheaves supported on various strata of the morphism $f$. 
Note that
\[\{\rho\in \Char(A)| H^k(X, f^*\bbC_{\rho})\neq 0 \text{ for some } k\} = \bigcup_{\alpha, i, k}\sV^k(A, \sP_{\alpha, i}).\]
Applying Theorem \ref{thm:perverse} to each 
of these $\sP_{\alpha,i}$ we obtain the result.
\end{proof}


\begin{proof}[Proof of Theorem \ref{thm:pervese}]
By the propagation property of the cohomology jump loci
$\sV^i(A, \sP)$ as in Theorem \ref{thm:cjl}(1) we
only need to consider $\sV^0(A,\sP)$. We split the argument in 
two cases:
\noindent \textbf{Case I: }$\chi(A, \sP)>0$. 
For a general character $\rho$, $H^i(A, sP\otimes \bbC_{\rho})= 0$ for $i\neq 0$ by Theorem \ref{thm:cjl}(3). Hence for a general
$\rho$,
in this case we must have $H^0(A, \sP\otimes \bbC_{\rho})\neq 0$. Hence, 
\[TC_{\rho}\sV^0(A, \sP) = H^1(X,\bbC)\]
On the other hand we have
the Kashiwara index theorem \ref{thm:indextheorem}
that states
\[\chi(A,P) = \langle \CC(\sP), T^*_AA\rangle\]
where $CC(P)$ is the characteristic cycle of $\sP$ as
in Definition \ref{def:cc}.
Note that if $Z\subset A$ is fibered by a subabelian variety $B\subset A$, then $T^*_ZA = T^*_BA$. But $<T^*_BA, T^*_AA> = \chi(B, \bbC) = 0$ (see e.g.\ \cite[p.\ 124]{Dim}).  Therefore,
the $\SS(\sP)$ must contain a subvariety $Z\subset A$ such that
$Z$ is not fibered by tori. 

We conclude by Proposition \ref{van-nonsimple} that when $Z$ is not fibered by tori $\pi(T^*_ZA) = H^0(A,\Omega_A^1)$. Hence
we obtain the desired equality.

\noindent \textbf{Case II: } $\chi(A, \sP)=0$. 
By a result of Weissauer \cite[Theorem 2]{Wei}
we know that there exists an abelian variety $B$ and a surjective morphism $\varphi\colon A\to B$, a simple perverse sheaf $\sP_B$ on $B$, and a local system $\bbL$ on $A$
such that 
\[\sP\otimes \bbL\simeq \varphi^*\sP_B.\]
Since $\sP$ and $\sP\otimes \bbL$ have the same singular support
and upto translation by $\bbL$ the same cohomology jump loci, we 
may and do replace $\sP$ by $\sP\otimes \bbL$.

On the other hand, by a result of
Franecki and Kapranov
\cite[Corollary 1.4]{FK} the formula of
characterstic cucle, i.e\ \[\CC(\sP) = \sum_Z n_Z Z\]
satisfies $n_Z\geq 0$. Hence from Kashiwara's index theorem 
\ref{thm:indextheorem}, one can deduce that all $Z\subset \SS(\sP)$ must be fibered by tori. Similarly, there must be 
$Z$, with $Z' = \varphi(Z)$ not fibered by tori.
Therefore, from the previous case we have 
\[(TC_{\rho}\sV^0(B, \sP_B))^{1,0} = \pi(\SS(\sP_B).\]
We then conclude 
that $\varphi^*(\Char(B))\subset \sV^0(A, \sP)$ from observing that
\[H^0(B, \sP_B\otimes \bbC_{\rho'}) \overset{\oplus}{\into}
H^0(A, \sP\otimes \bbC_{\rho'}).\]
To see the converse note that 
for any $\rho \in \sV^0(A, \sP)$ we must 
have $\bbC_{\rho}|_{\operatorname{Fibre}(\varphi)} = \bbC$.
Indeed, $H^0(A, \sP\otimes \bbC_{\rho}) = H^0(B, \sP_B\otimes \varphi_*\bbC_{\rho})$.
Hence $\varphi_*\bbC_{\rho} = \bbC{\rho'}$ and $\bbC_{\rho} = 
\varphi^*\bbC_{\rho'}$.







\end{proof}




%\bibliographystyle{alpha}
%\bibliography{main_new}


\begin{thebibliography}{ADMSP}

\bibitem[Ara92]{Ara92} D.\ Arapura, ``Higgs line bundles, Green-Lazarsfeld sets, and maps of K\"ahler manifolds to curves", \textit{Bull., New Ser.,
Am. Math. Soc.}, vol.\ \textbf{26} (1992), pp.\ 310--314.

\bibitem[Ara97]{Ara97} D.\ Arapura, ``Geometry of cohomology support loci for local systems, I". \textit{Journal of Algebraic Geometry} vol.\ \textbf{3} (1997): pp.\ 563--597.


\bibitem[BWY16]{BWY16} N. Budur, B. Wang, Y. Yoon, \textit{Rank one local systems and forms of degree one}. Int. Math. Res. Not. \textbf{13} (2016), 3849--3855.



\bibitem[CL73]{CL73} J. B. Carrell, D. I. Lieberman,  \textit{Holomorphic vector fields and K\"ahler manifolds}, Invent. Math. \textbf{21} (1973), 303--309.

\bibitem[Deb07]{Deb} O. Debarre,  \textit{On the geometry of abelian varieties} \url{https://www.math.ens.fr/\textasciitilde debarre/Taiwan.pdf}, (2007).

\bibitem[DJL17]{DJL17} O.\ Debarre, Z.\ Jiang and M.\ Lahoz, Rational cohomology tori, \textit{Geometry \& Topology} vol.\ 21 (2017),
pp.\ 1095--1130.

\bibitem[DiPa13]{DiPa13}A. Dimca, S. Papadima, \textit{Nonabelian cohomology jump loci from ananalytic view point}. Commun. Contem. Math. \textbf{15} (5), 1350025 (2013).


\bibitem[DiSu]{disu} A. Dimca, A. I. Suciu, \textit{Which 3-manifold groups are K\"ahler groups?}, J. Eur. Math. Soc., \textbf{11} (2009), no. 3, 521-528.

\bibitem[GH18]{GH18} I. Glazer, Y. I. Hendel, \textit{On singularity properties of convolutions of algebraic morphisms--the general case (with an appendix joint with Gady Kozma)}, arXiv:1811.09838.

\bibitem[GH19]{GH19}  I. Glazer, Y.I. Hendel, \textit{On singularity properties of convolutions of algebraic morphisms}, Selecta Mathematica (2019), pp.\ 25--15


\bibitem[GL87]{GL87} M. Green, R. Lazarsfeld, \textit{Deformation theory, generic vanishing theorems and some conjectures of Enriques, Catanese and Beauville}, Inv. Math. \textbf{90} (1987), 389--407.

\bibitem[GW10]{GW10} U. G\"ortz, T. Wedhorn, \textit{Algebraic geometry I. Schemes with examples and exercises.} Advanced Lectures in Mathematics. Vieweg+Teubner, Wiesbaden, 2010. 

\bibitem[HK05]{HK05} C. D. Hacon, S. J. Kov\'acs, \textit{Holomorphic one-forms on varieties of general type}, Ann. Sci. \'Ecole Norm. Sup. \textbf{38} (2005), no. 4, 599--607.

\bibitem[Har77]{Har77} R. Hartshorne, \textit{Algebraic Geometry}, Graduate Texts in Mathematics, vol. 52, Springer, New York,
1977.

\bibitem[HS19]{HS19} F. Hao, S. Schreieder, \textit{Holomorphic one-forms without zeros on threefolds}, to appear Geometry $\&$ Topology , arXiv:1906.07606.

\bibitem[KK]{KK} J. Koll\'ar, S. Kov\'acs {\it Singularities of the Minimal Model Program.}
Series: Cambridge Tracts in Mathematics (No. 200)

\bibitem[Laz04a]{Laz04a}
    R. Lazarsfeld. \textit{Positivity in algebraic geometry.\ I.\ Classical
    setting: line bundles and linear series}.
    Ergeb.\ Math.\ Grenzgeb.\ (3), Vol.\ 48. Berlin: Springer-Verlag, 2004.
    \textsc{doi}: \href{https://doi.org/10.1007/978-3-642-18808-4}{\ttfamily 10.1007/978-3-642-18808-4}.
    \textsc{mr}: \href{http://www.ams.org/mathscinet-getitem?mr=2095471}{\ttfamily 2095471}.


\bibitem[Laz04b]{Laz04}
    R. Lazarsfeld. \textit{Positivity in algebraic geometry.\ II.\ Positivity
    for vector bundles, and multiplier ideals}.
    Ergeb.\ Math.\ Grenzgeb.\ (3), Vol.\ 49. Berlin: Springer-Verlag, 2004.
    \textsc{doi}: \href{https://doi.org/10.1007/978-3-642-18810-7}{\ttfamily 10.1007/978-3-642-18810-7}.
    \textsc{mr}: \href{http://www.ams.org/mathscinet-getitem?mr=2095472}{\ttfamily 2095472}.

%\bibitem[KM05]{KM05} J. Koll\'ar and S. Mori, \textit{Birational geometry of algebraic varieties}, Cambridge University Press, Cambridge, 2008.

%\bibitem[Mo85]{Mo85} S. MORI S, \textit{Classification of higher-dimensional varieties}, in: Algebraic Geometry, Bowdoin, (1985),Brunswick, Maine, (1985), in: Proc. Sympos. Pure Math., \textbf{46}, American Mathematical Society,Providence, RI, (1987), 269--331.

%\bibitem[LMW18]{LMW18} Y. Liu, L. Maxim, B. Wang, {\it Perverse sheaves on semi-abelian varieties}, arXiv: 1804.05014. 

\bibitem[LMW20]{LMW20} Y. Liu, L. Maxim, B. Wang, \textit{Aspherical manifolds, Mellin transformation and a question of Bobadilla-Koll\'ar}, arXiv:2006.09295.

\bibitem[LZ05]{LZ05} T.\ Luo and Q.\ Zhang, Holomorphic forms on threefolds, \textit{Recent Progress in Arithmetic and Algebraic Geometry, Contemp.\ Math, Amer.\ Math.\ Soc.}, vol.\ 386, pp.\ 87--94 Providence, RI, (2005). 
\textsc{doi}: \href{https://doi.org/org/10.1090/conm/386/07219}{\ttfamily 10.1090/conm/386/07219}.
\textsc{mr}: \href{http://www.ams.org/mathscinet-getitem?mr=2182772}{\ttfamily 2182772}.

\bibitem[MiSh18]{MiSh18} L. Migliorini and V. Shende, \textit{Higher discriminants and the topology of algebraic maps}, Algebraic geometry, (2018), 114--130.

\bibitem[Mum]{Mum} D. Mumford, \text{Abelian Varieties}, Tata Institute of Fundamental Research, (1985).

\bibitem[PS14]{PS14} M. Popa and C. Schnell, \textit{Kodaira dimension and zeros of holomorphic one-forms}, Ann. of Math. \textbf{179} (2014), 1109--1120.


\bibitem[Sha74]{Sha74} I. Shafarevich, \textit{Basic Algebraic Geometry I}. Berlin, Springer--Verlag, (1974).

\bibitem[Sch15]{Sch}
C. Schnell, {\it  Holonomic $\mathcal{D}$-modules on abelian varieties},
Publ. Math. Inst. Hautes \'{E}tudes Sci. 121 (2015), 1--55.

\bibitem[Sch19]{SS19} S.\ Schreieder, \textit{Zeros of holomorphic one-forms and topology of K\"ahler manifolds, (Appendix written jointly with H.-Y. Lin) }, to appear IMRN. (2019).

\bibitem[Schm76]{Schm76} W. Schmid, Variation of Hodge structure: the singularities of the period mapping, Inv. Math. \textbf{22} (1973), 211--319.

\bibitem[Ste76]{Ste76} J. Steenbrink, \textit{Limits of Hodge structures}, Inv. Math. \textbf{31} (1976), 229--257.




\bibitem[Sim93a]{Sim93} C. Simpson, \textit{Lefschetz theorems for the integral leaves of a holomorphic one-form}, Compositio Math. \textbf{87} (1993), 99--113.

\bibitem[Sim93b]{Sim93b} ---------, Subspaces of moduli spaces of rank one local systems, \textit{Ann. Sci. \'Ec. Norm. Super.}, vol.\ 26,
pp.\ 361–-401 (1993).

\bibitem[Sp88]{Sp88} M. Spurr, {\it On the zero set of a holomorphic one-form on a compact complex manifold.} Trans. Am.
Soc. \textbf{308} (1988), 329-339. 

\bibitem[stacks-project]{stacks-project} The {Stacks project authors}, \textit{The Stacks project}, \url{https://stacks.math.columbia.edu}, (2020).

\bibitem[Uen75]{Uen75} K.\ Ueno. ``Classification theory of algebraic varieties and compact complex spaces", \textit{Lecture Notes in Mathematics},
Springer-Verlag, Berlin, 1975, Notes written in collaboration with P. Cherenack, Vol.\ \textbf{439}. 
\textsc{mr}: \href{http://www.ams.org/mathscinet-getitem?mr=0506253}{\ttfamily 0506253}.


\bibitem[Dim]{Dim} A. Dimca, {\it Sheaves in Topology} Springer-Verlag Berlin Heidelberg New York in 2004.
\textsc{doi:} \href{https://doi.org/10.10071978-3-642-18868-8}{\ttfamily 10.10071978-3-642-18868-8}.
\textsc{isbn:} {\ttfamily 978-540-20665-1 }.

\bibitem[BWY]{BWY} N. Budur, B . Wang, Y. Yoon, {\it Rank one local systems and forms of degree one}, Int. Math. Res. Not. (2016), no.13, 3849-3855.

%
\bibitem[LMW]{LMW} Y. Liu, L. Maxim, B. Wang, \textit{Aspherical manifolds, Mellin transformation and a question of Bobadilla-Kollar}, arXiv:2006.09295.

\bibitem[HK05]{HK05} C. D. Hacon, S. J. Kov\'acs, \textit{Holomorphic one-forms on varieties of general type}, Ann. Sci. \'Ecole Norm. Sup. \textbf{38} (2005), no. 4, 599--607.


\bibitem[Suc16]{S} A. Suciu, {\it Around the tangent cone theorem}, Configuration Spaces: Geometry, Topology and Representation Theory, 1-39, Springer INdAM series, vol. 14, Springer, Cham, 2016

\bibitem[Wei12]{Wei12} R.\ Weissauer, ``Degenerate perverse sheaves on abelian varieties", \href{https://arxiv.org/abs/1204.2247}{\ttfamily arXiv:1204.2247[math.AG]}, 2012.




\end{thebibliography}



%------------------------------------------------------------------
\end{document}
%------------------------------------------------------------------




