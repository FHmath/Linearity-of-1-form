\pdfoutput=1
\documentclass[11pt,reqno]{amsart}
\usepackage[letterpaper,margin=1in,footskip=0.25in]{geometry}
\usepackage{mathrsfs}
\usepackage{amssymb}
\usepackage{mathtools}
\usepackage{tikz-cd}
\usepackage{enumitem}

\PassOptionsToPackage{pdfusetitle,pagebackref,colorlinks}{hyperref}
\usepackage{bookmark}
\hypersetup{
  linkcolor={red!70!black},
  citecolor={green!70!black},
  urlcolor={blue!80!black}
}

\newtheorem{theorem}{Theorem}[section]
\newtheorem{lemma}[theorem]{Lemma}
\newtheorem{proposition}[theorem]{Proposition}
\newtheorem{corollary}[theorem]{Corollary}
\newtheorem{claim}[theorem]{Claim}
\newtheorem{conjecture}[theorem]{Conjecture}
\newtheorem{step}{Step}[subsection]
\renewcommand{\thestep}{\arabic{step}}

\newtheorem{alphtheorem}{Theorem}
\renewcommand{\thealphtheorem}{\Alph{alphtheorem}}

\theoremstyle{definition}
\newtheorem{definition}[theorem]{Definition}
\newtheorem{notation}[theorem]{Notation}

\theoremstyle{remark}
\newtheorem{remark}[theorem]{Remark}

\newtheoremstyle{cited}{.5\baselineskip\@plus.2\baselineskip\@minus.2\baselineskip}{.5\baselineskip\@plus.2\baselineskip\@minus.2\baselineskip}{\itshape}{}{\bfseries}{\bfseries .}{5pt plus 1pt minus 1pt}{\thmname{#1}\thmnumber{~#2}\thmnote{ \normalfont#3}}
\theoremstyle{cited}
\newtheorem{citedthm}[theorem]{Theorem}
\newtheorem{citedconj}[theorem]{Conjecture}
\newtheorem{citedlem}[theorem]{Lemma}
\newtheorem{citedprop}[theorem]{Proposition}

\newtheoremstyle{citeddef}{.5\baselineskip\@plus.2\baselineskip\@minus.2\baselineskip}{.5\baselineskip\@plus.2\baselineskip\@minus.2\baselineskip}{}{}{\bfseries}{\bfseries .}{5pt plus 1pt minus 1pt}{\thmname{#1}\thmnumber{~#2}\thmnote{ \normalfont#3}}
\theoremstyle{citeddef}
\newtheorem{citednot}[theorem]{Notation}


%%==============Yongqiang's typsets====================%%

\newcommand{\CN}{\mathbb{C}^{n+1}}
\newcommand{\CP}{\mathbb{CP}^{n+1}}
\newcommand{\U}{\mathcal{U}}
\newcommand{\C}{\mathbb{C}}
\newcommand{\Z}{\mathbb{Z}}
\newcommand{\Hom}{\mathrm{Hom}}
\newcommand{\Q}{\mathbb{Q}}
\newcommand{\K}{\mathcal{L}}
\newcommand{\V}{\mathcal{V}}

\def\be{\begin{equation}}
\def\ee{\end{equation}}

\def\bt{\begin{theorem}}
\def\et{\end{theorem}}

\def\bc{\begin{corollary}}
\def\ec{\end{corollary}}

\def\br{\begin{remark}}
\def\er{\end{remark}}

\def\bp{\begin{proposition}}
\def\ep{\end{proposition}}

\def\bl{\begin{lemma}}
\def\el{\end{lemma}}

%\def\bn{\begin{enumerate}}
%\def\en{\end{enumerate}}

\def\bex{\begin{ex}}
\def\eex{\end{ex}}

\def\bd{\begin{definition}}
\def\ed{\end{definition}}








%\DeclareMathOperator{\Pic}{Pic}                  % Pic

%\newcommand{\PP}{\mathcal{P}}           % Poincare line bunle

\DeclareMathOperator{\Supp}{Supp}                % Supp
\DeclareMathOperator{\codim}{codim}              % codim
\DeclareMathOperator{\mreg}{mreg}                % mreg
\DeclareMathOperator{\reg}{reg}                  % reg
\DeclareMathOperator{\sing}{sing}                  
\DeclareMathOperator{\id}{id}                    % id
\DeclareMathOperator{\obj}{Obj}
\DeclareMathOperator{\ad}{ad}
\DeclareMathOperator{\morph}{Morph}
\DeclareMathOperator{\enom}{End}
\DeclareMathOperator{\iso}{Iso}
\DeclareMathOperator{\Exp}{Exp}
\DeclareMathOperator{\homo}{Hom}
\DeclareMathOperator{\enmo}{End}
\DeclareMathOperator{\spec}{Spec}
\DeclareMathOperator{\fitt}{Fitt}
\DeclareMathOperator{\odr}{\Omega^\bullet_{\textrm{DR}}}
\DeclareMathOperator{\rank}{Rank}
\DeclareMathOperator{\gdeg}{gdeg}
\DeclareMathOperator{\Alb}{Alb}
\DeclareMathOperator{\alb}{alb}
\DeclareMathOperator{\Ann}{Ann}



%\DeclareMathOperator

\DeclareMathOperator{\Char}{Char}
\DeclareMathOperator{\CC}{SS}
\DeclareMathOperator{\kn}{Ker}
\DeclareMathOperator{\im}{Im}


\def\ra{\rightarrow}


\def\bone{\mathbf{1}}
\def\bC{\mathbb{C}}
\def\cM{\mathcal{M}}
\def\cV{\mathcal{V}}
\def\Def{{\rm {Def}}}
\def\cR{\mathcal{R}}
\def\om{\omega}
\def\wti{\widetilde}
\def\al{\alpha}
\def\End{{\rm {End}}}
\def\Pic{{\rm Pic}}
\def\bP{\mathbb{P}}
\def\cH{\mathcal{H}}
\def\bL{\mathbb{L}}
\def\cX{\mathcal{X}}
\def\cI{\mathcal{I}}
\def\pa{\partial}
\def\cY{\mathcal{Y}}
\def\cD{\mathcal{D}}
\def\cO{\mathcal{O}}
\def\lra{\longrightarrow}
\def\bQ{\mathbb{Q}}
\def\ol{\overline}
\def\cL{\mathcal{L}}
\def\bH{\mathbb{H}}
\def\bZ{\mathbb{Z}}
\def\bW{\mathbf{W}}
\def\bV{\mathbf{V}}
\def\bM{\mathbf{M}}
\def\eps{\epsilon}
\def\ul{\underline}
\def\lam{\lambda}
\def\sX{\mathscr{X}}
\def\bN{\mathbb{N}}


%%==========Yagna's typsets============%%
\usepackage{tikz-cd}
\usepackage{enumitem}

\PassOptionsToPackage{pdfusetitle,pagebackref,colorlinks}{hyperref}
\usepackage{bookmark}
\hypersetup{
  linkcolor={red!70!black},
  citecolor={green!70!black},
  urlcolor={blue!80!black}
}

%Mathcal Letters =====================
\newcommand{\sA}{\mathcal{A}}
\newcommand{\sB}{\mathcal{B}}
\newcommand{\sD}{\mathcal{D}}
\newcommand{\sF}{\mathcal{F}}
\newcommand{\sG}{\mathcal{G}}
\newcommand{\sH}{\mathcal{H}}
\newcommand{\sK}{\mathcal{K}}
\newcommand{\sL}{\mathcal{L}}
\newcommand{\sM}{\mathcal{M}}
\newcommand{\sN}{\mathcal{N}}
\newcommand{\sO}{\mathcal{O}}
\newcommand{\sP}{\mathcal{P}}
\newcommand{\sQ}{\mathcal{Q}}
\newcommand{\sR}{\mathcal{R}}
\newcommand{\sT}{\mathcal{T}}
\newcommand\sV{{\mathcal V}}
\newcommand\sW{{\mathcal W}}
\newcommand{\sZ}{\mathcal{Z}}

%mathbb Letters======
\newcommand{\bbA}{\mathbb{A}}
\newcommand{\bbB}{\mathbb{B}}
\newcommand{\bbC}{\mathbb{C}}
\newcommand{\bbG}{\mathbb{G}}
\newcommand{\bbH}{\mathbb{H}}
\newcommand{\bbK}{\mathbb{K}}
\newcommand{\bbL}{\mathbb{L}}
\newcommand{\bbM}{\mathbb{M}}
\newcommand{\bbN}{\mathbb{N}}
\newcommand{\bbP}{\mathbb{P}}
\newcommand{\bbQ}{\mathbb{Q}}
\newcommand{\bbR}{\mathbb{R}}
\newcommand{\bbV}{\mathbb{V}}
\newcommand{\bbZ}{\mathbb{Z}}



%Script Letters ======================
\newcommand{\frf}{\mathfrak{f}}
\newcommand{\frM}{\mathfrak{M}}

\newcommand{\scrL}{\mathscr{L}}
\newcommand{\crI}{\mathscr{I}}
\newcommand{\scrK}{\mathscr{K}}
\newcommand{\scrB}{\mathscr{B}}
\newcommand{\scrC}{\mathscr{C}}
\newcommand{\scrD}{\mathscr{D}}
\newcommand{\scrE}{\mathscr{E}}
\newcommand{\scrI}{\mathscr{I}}
\newcommand{\scrQ}{\mathscr{Q}}
\newcommand{\scrR}{\mathscr{R}}
\newcommand{\scrX}{\mathscr{X}}
\newcommand{\scrY}{\mathscr{Y}}
\newcommand{\scrF}{\mathscr{F}}
\newcommand{\Dred}{\lceil D\rceil}

%Arrow Style =========================
\newcommand{\into}{\hookrightarrow}
\newcommand{\onto}{\rightarrow\hspace*{-.14in}\rightarrow}

%Math Operators ======================



%\DeclareMathOperator{\alg}{alg}
\DeclareMathOperator{\BM}{BM}
\DeclareMathOperator{\Bs}{Bs}
\DeclareMathOperator{\Bsp}{\mathbf{B}_+}
\DeclareMathOperator{\SB}{\mathbf{B}}

\DeclareMathOperator{\coh}{coh}
\DeclareMathOperator{\Coker}{Coker}
 \renewcommand{\div}{\text{div}}

\DeclareMathOperator{\DR}{DR}
\DeclareMathOperator{\Ch}{Ch}
\DeclareMathOperator{\discrep}{discrep}
\DeclareMathOperator{\exc}{exc}

\DeclareMathOperator{\free}{free}


\DeclareMathOperator{\HHom}{\mathcal{H}\!\mathit{om}}
\DeclareMathOperator{\image}{Im}


\DeclareMathOperator{\Ind}{Ind}
\DeclareMathOperator{\Ker}{Ker}
\DeclareMathOperator{\Lie}{Lie}
\DeclareMathOperator{\op}{op}


\DeclareMathOperator{\qcoh}{qcoh}

%\DeclareMathOperator{\reg}{reg}

\DeclareMathOperator{\RHHom}{\mathbf{R}\mathcal{H}\!\mathit{om}}


\DeclareMathOperator{\Spf}{Spf}
%\DeclareMathOperator{\Supp}{Supp}
\DeclareMathOperator{\tors}{tors}
\DeclareMathOperator{\torsion}{torsion}
\DeclareMathOperator{\Var}{Var}
%\DeclareMathOperator{\Sym}{Sym}


\newcommand{\Ab}{\mathbf{Ab}}
\newcommand{\Aff}{\mathbf{Aff}}
\newcommand{\tB}{\tilde{B}}
%\newcommand{\et}{{\acute{e}t}}

\newcommand{\Sets}{\mathbf{Sets}}
%\newcommand{\cX}{\bar{X}}
\newcommand{\cP}{\bar{P}}
%\newcommand{\cV}{\bar{V}}
\newcommand{\cW}{\bar{W}}
\newcommand{\sorry}[1]{\textcolor{red}{#1}}
\newcommand{\dual}[1]{\sD^{\Omega}_{M^{#1}}}



\title{}








\begin{document}  
\title[Cohomology jump loci and holomorphic 1-forms with zeros]{On linearity of holomorphic 1-forms with zeros} 

\author{Yajnaseni Dutta}

%\address{}
%\email{}

\author{Feng Hao}

%\address{}
%\email{}

\author{Yongqiang Liu}

%\address{}
%\email{}


%\date{\today}
%\subjclass[2010]{} 
%\keywords{} 



\begin{abstract} 
The goal of this article is two-fold, first we 
\end{abstract}

\maketitle
\section{Introduction}\label{intro}
Given a smooth porjective variety $X$, in this article we relate the cohomology jump loci of $\bbC_X$ with the 
stratification arising from the decomposition theorem for the albnese morphism. Such relation fell out of our interest in the
study of linearity of holomorphic 1-forms. This is inspired by the work of  \cite{CL73}, Carrell and Lieberman 
who show that the set of global holomorphic tangent vector fields with zeros is a subvector space of $H^0(X, T_X)$. We deal with a similar question for global holomorphic 1-forms. In this case the story is far from complete and is deeply connected to the generic
vanishing theory. To this end, we obtained that the set of 1-forms that admit codimension 1 zeros forms a linear subset of $H^0(X, \Omega_X^1)$. We also obtain the following geometric/cohomological property of smooth projective varieties that
admits a non-vanishing 1-form. 



\begin{alphtheorem}\label{thm:smooth}
Let $f:X\to A$ be a morphism from a smooth projective variety to a simple abelian variety $A$. Then $f$ is cohomologically a fiber bundle if and only if there is a global holomorphic 1-form $\omega$ on $A$ such that $f^*\omega$ is nowhere vanishing.  In particular, the
singular fibres of $f$ carry a pure Hodge structure.
\end{alphtheorem}

By a \textsl{cohomological fiber bundle} we mean that the higher direct
images $\bbR^if_*\bbC$ are integrable connections for all $i$. In particular, the derived direct image
$\bbR f_*\bbC$ decomposes like the smooth morphism in the derived category $D^b(\bbC)$. In fact a bit more can be said; When $A$ is not necessarily simple existence of a non-vanishing global holomorphic 1-form
is equivalent to certain restrictions on the singular support 
(see Definition \ref{def:ss}) of 
these pushforwards. See Theorem \ref{thm:nonvanishing} for more details.

\begin{remark}[Previous results]
\begin{enumerate}
\item \label{item:ps} When $X$ is of general type, it was conjectured in
	\cite{HK05, LZ05} and was proved
	by Popa and Schnell \cite{PS14} every global holomorphic 1-		
	form vanishes on $X$. 
\item \label{item:hk} On the opposite extreme for any $A$, not necessarily simple when $f$ is singular along a divisor of general
		type in $A$, Hacon and Kov\'acs \cite[Proposition 3.5.]{HK05} show that
		$f^*\omega$ always admits zero. See Corollary \ref{cor:hk}
		for a reinterpretation of their argument. 
\end{enumerate}	
\end{remark}


In order to state the linearity results we define
the following set
\[V(X):=\{ \omega\in H^0(X, \Omega_X^1) | Z(\omega)\neq \emptyset\},\]
It is known due to the generic vanishing theory
%theory of cohomology jumploci 
that only a piece of $V(X)$, arising from the generic vanishing theory (see \S \ref{se:gv} for more details) is linear, i.e.\
%\emph{the resonant 1-forms} (see Definition \ref{def:resonance} below) 
this set forms a finite union of subvector spaces inside $H^0(X, \Omega_X^1)$. To be more specific, this set is given by
\cite[p.\ 311]{Ara}
\begin{equation}
\begin{split}
\sR(X)\coloneqq \bigcup_i\{\omega\in H^0(X, \Omega_X^1)| \exists \sL\in \Pic^0(X) \text{ and } p\in \bbZ_{\geq 0} \text{ such that},
H^q(H^{p}(X, \Omega_X^{\bullet}\otimes \sL), \wedge\omega) \neq 0
\\\text{ for all } p+q=i\}
\label{eq:}
\end{split}
\end{equation}
This set is also the holomorphic piece of the tangent cone to the cohomology jump loci 
$\sV^i(A,\sP)$ at finitely many local systems associated to $\rho\in\Char(X) = Hom(H_1(X,\bbZ)/\torsion, \bbC^{\star})$.
%\cite[Theorem 4.2]{Sim93} (see also \cite{DiPa13} from where
%we borrow the terminology). 
%Henceforth we shall call finite unions of subvector spaces as \emph{linear subvarieties}. 
%In the relative setting
%of a morphism $f\colon X\to A$ from a smooth projective variety to an abelian variety $A$,
%we similarly define
%\[V(f):=\{ \omega\in H^0(A, \Omega_A^1) | Z(f^*\omega)\neq \emptyset\}.\]
On the other hand by Kashiwara's estimate
of character varieties, it is known that given a proper morphism
$f\colon X\to A$ to an abelian variety $A$, we have
\[SS(Rf_*\bbC) \subseteq f_*(df^{-1}(0_X))\subset T^*A\]
where $SS(Rf_*\bbC)$ is the singular support of the constructible
sheaf $Rf_*\bbC$ (see Definition \ref{def:ss}) and $0_X$ is the
zero section $T^*_XX$. Furthermore, $f_*(df^{-1}(0_X))$ is
defined via the following diagram:
\begin{equation}
\begin{tikzcd}
A\times H^0(A,\Omega_A^1)  \ar[dr]
& X\times H^0(A, \Omega_A^1)\ar[r, "df"']\ar[d] \ar[l, "f\times id"]
& T^*X \ar[d]\\
&A  & X\ar[l, "f"]
\end{tikzcd}
\label{eq:maindiagram}
\end{equation}
Note that under the projection $\pi\colon A\times H^0(A,\Omega_A^1)\to H^0(A,\Omega_A^1)$ we have
\[\pi(f_*(df^{-1}(0_X))) = V(f) \coloneqq \{\omega\in H^0(A,\Omega_A^1)|
Z(f^*\omega) \neq \emptyset\}.\]
Therefore, when $f$ is the albanese morphism, we obtain
\[SS(Rf_*\bbC)\subseteq V(X).\]
We show in Proposition \ref{prop:vansimple} 
that $SS(Rf_*\bbC)$ is a finite union of linear subspaces of $H^0(A,\Omega_A^1)$. We
have the following
\begin{alphtheorem}
Let $a\colon X\to \Alb_X$ denotes the albanese morphism. Then
\[\sR(X) =  \SS(Ra_*\bbC). \]
\label{thm:linearity}
\end{alphtheorem}

\begin{remark}
Note that it is not clear a priori why $\sR(X)\subseteq V(X)$. However, it is clear from our Theorem. Although this was known by a special case of an earlier result of Budur--Wang--Yoon
\cite{BWY}.

\end{remark}
The theorem follows from a more general statement
about perverse sheaves on abelian varieties. Recall
\begin{definition}
Associated to $\rho\in \Char(X)$ the collection of local systems $\bbC_{\rho}$ defined as
\[\sV^i(A, \sP) \coloneqq \{\rho\in\Char(X)| H^i(A, \sP\otimes \bbC_{\rho}) \neq 0\}\]
is called the cohomology jump loci of $\sP$.
\label{def:cjl}
\end{definition}
\begin{alphtheorem}
Let $A$ be an abelian variety. Let $P$ be a complex of perverse sheaves with complex coefficient on $A$. 
Then we have the equality that
$$\pi(CC(P)) = H^0(X, \Omega_X^1)\cap \bigcup_{\rho} \rho^{-1} TC_{\rho} \sV^0(A,\sP), $$
where the union is running over a representative point from every irreducible components of $\sV^0(A,P)$
$TC_{\rho} \sV^0(A,P) \subseteq H^1(X, \bbC)$ denotes the tangent cone at $\rho$. 
\label{thm:perverse}
\end{alphtheorem}

For an
interpretation of these results in the category of regular holonomic D-modules
see Theorem \ref{thm:linearitydmodule}. The key technique we
use to relate the the generic vanishing theory on abelian varieties with the support of $Rf_*\bbC$ is Kashiwara's
global index theorem. 

The aforementioned linearity can be seen in this context
as follows. Recall that because of the group structure, the total space $T^*A$ of the cotangent bundle (sometimes referred to simply as cotangent bundle) of an abelian variety $A$ satisfies $T^*A\simeq A\times H^0(A, \Omega_A^1)$. We also show 
(see Proposition \ref{van-nonsimple}) that the image of $T^*_{S_{\alpha_j}}A\subset T^*A$ under the projection onto $H^0(A,\Omega_A^1)$ is a finite union of linear subsets. Therefore, $\sR(X)$
is  linear. 

We do know at the moment whether the equality holds in $\sR(X) \subseteq V(X)$. 
%An example of Debarre, Jiang and Lahoz 
%\cite[Example 1.11]{DJL17} shows that the
%there exists a bi-elliptic surface $S$ admitting a 1-form 
%$\omega$ for which $(H^*(X, \mathbb{C}), \wedge \omega)$
%is exact, yet $\omega$ admits zeros on $S$. Theorem \ref{thm:linearity} shows that such forms are not under the realm of generic vanishing theory. 

\begin{remark}
Note that if $\chi(X)>0$, we have $\sR(X) = V(X) = H^0(X, \Omega_X^1)$. Hence in this case the set of 1-forms that admit zeros is indeed linear. 
This follows immediately from the generic vanishing theory. Indeed, by Hodge decomposition
we have
$H^k(X,\bbC) \simeq \bigoplus_{p+q = k} H^p(X,\Omega_X^q)$. Hence the complexes $(H^p(X, \Omega_X^{\bullet}), \wedge\omega)$ can be summed together to form the complex $(H^p(X, \bbC), \wedge\omega)$. Since $\chi(X)>0$ the latter complex cannot be exact. Hence $H^0(X,\Omega_X^1)
= \sR(X)$.
\end{remark}




More generally we expect the following to be true.
\begin{conjecture} \label{linear-v1}
Let $f: X\to A$ be a morphism from a smooth complex projective variety $X$ to an abelian variety $A$. Then $V^i(f)$ are linear subvarieties of $H^0(X, \Omega_X^1)$, i.e., $V^i(X)$ is a finite union of linear subspaces of $H^0(X, \Omega_X^1)$ for each nonnegative integer $i$. In particular $V^i(X)$ are linear.
\end{conjecture}

To this end, using a result of Spurr \cite{Sp88}, we directly get $V^1(X)$ is linear.

\begin{alphtheorem}
Let $f: X\to A$ be a morphism from a smooth complex projective variety $X$ to an abelian variety $A$. Then $V^1(f)$ is linear. In particular, $V^1(X)$ is linear.
\end{alphtheorem}


 %What's more, we show that $V^1(f)$ consists the resonant holomorphic 1-forms vanishing along a divisor and ``negative'' 1-forms (holomorphic 1-forms, which is not  necessarily resonant, but vanishes along a rigid divisor). Also, in this article, 
We also generalize a result of Spurr \cite{Sp88} to quasi-projective varieties. 

\begin{alphtheorem} \label{log-linear-v1}
Let $(X, D)$ be a pair with $X$ a complex smooth projective variety and $D$ a simple normal crossing divisor of $X$. Let $H$ be a very ample divisor on $X$. If $(X, D)$ carries a holomorphic log 1-form $0\not=\omega\in H^0(X, \Omega_X^1(\log D))$  which pullbacks to zero on an effective divisor $E$ with $E^2\cdot H^{n-2}\geq0$, then there is a morphism $f: X-D\to C$ to a smooth quasi-projective curve $C=\bar{C}-B$ (where $\bar{C}$ is the completion of $C$ and $B$ can be empty) with 

(1)  $\omega=f^*\eta$ for some $\eta\in H^0(X, \Omega_{\bar{C}}^1(\log B))$.

(2) $E$ is contained in the fiber of $f$ and $E^2\cdot H^{n-2}=0$.
\end{alphtheorem}

As a corollary we prove the linearity of the set of logarithmic holomorphic 1-forms admitting codimension one zeros.

\begin{corollary}
Let $(X, D)$ be a pair with $X$ a complex smooth projective variety and $D$ a simple normal crossing divisor of $X$. Then the set $$ V^1(X,D):=\{ w \in W(U) \mid \codim_X Z(w) \leq i \}$$ is linear.
\end{corollary}

We show that $V^1(X,D)$ consists of resonant holomorphic logarithmic 1-forms vanishing along a divisor, holomorphic logarthmic 1-forms vanishing along a rigid divisor, and holomorphic logarithmic 1-forms vanishing along some components of the boundary divisor $D$.


In this paper, all complex of sheaves and perverse sheaves are defined with complex
coeffcients. All the varieties are complex quasi-projective varieties. 

\subsection*{Acknowledgements}



\section{Preliminary}
\subsection{Classical Generic vanishing theory}
Given a holomorphic 1-form $\omega\in H^0(X,\Omega_X^1)$
the kernel of the associated Koszul complex
\begin{equation}
\sK^{\bullet}_{\omega} \coloneqq [\sO_X\overset{\wedge\omega}{\to} \Omega_X^1 \to \cdots\to \Omega_X^{n-1}\to \Omega^n_X.]
\label{eq:koszul}
\end{equation}
defines a rank 1 local system $L(\omega)$ (see \cite[\S 2.1]{sch}
for a construction). This gives in the way of the generic vanishing theory developed by \cite{GL, Ara, Sim} into the 
study of zeros of holomorphic 1-forms. In this section we discuss 
the relevant bits of this vast theory that we will use in various
proofs.
%
%\begin{definition}[Zero scheme of 1-forms]\label{def:zeroscheme}
%For $\omega\in H^0(X, \Omega_X^1)$, the \emph{zero set} $Z(\omega)$ of $\omega$ is the algebraic set of closed point $x$ in $X$, such that $\omega(v)=0$
%for all tangent vectors $v\in T_xX$ at $x$. 
%
%The \emph{zero scheme} of $\omega$ is the closed subscheme $\sZ(\omega)$ defined by the ideal sheaf $\mathcal{I}_{\omega}$ given by the image of the morphism 
%\[\mathcal{T}_X\overset{\langle\omega, \cdot\rangle}{\longrightarrow} \mathcal{O}_X.\] 
%Here $\mathcal{T}_X$ is the tangent sheaf of $X$ and $\langle\omega, \cdot\rangle$ denotes the pairing of tangent field with 
%the 1-form $\omega$.
%\end{definition}

 
The generic vanishing theory 
\cite[Proposition 3.4]{GL} ensures that
if $Z(\omega)\neq \emptyset$ then the sequence
\[\cdots\overset{\wedge\omega}{\to} H^k(X, \Omega^{i-1}
\overset{\wedge\omega}){\to}H^k(X, \Omega_X^{i})
\overset{\wedge\omega}{\to} H^k(X,\Omega_X^{i+1})
\overset{\wedge\omega}{\to}\cdots\]
is not exact for all $k\geq 0$. Putting these together 
by the Hodge decomposition for $H^k(X,\bbC)$ we get
\begin{equation}
(H^{\bullet}(X,\bbC), \wedge\omega)\coloneqq [\ldots\to H^{i-1}(X,\C)\overset{\wedge\omega}{\longrightarrow}H^{i}(X,\C)\overset{\wedge\omega}{\longrightarrow}H^{i+1}(X,\C)\to\ldots]
\label{eq:resonance}
\end{equation}
is not exact whenever $Z(\omega)\neq \emptyset$. This prompts the following


\begin{definition}[Resonant forms]\label{def:resonance}
We call a 1-form $\omega$ \emph{resonant} if the complex 
in Equation (\ref{eq:resonance}) is not exact. The set of all such forms 
is denoted $\sR(X,\bbC)$.

On the other hand for when the the complex 
in Equation (\ref{eq:resonance}) is exact and\footnote{Note that
with this assumption it is not always the case that $Z(\omega) = \emptyset$. See Example \ref{ex:DJL}.} 
$Z(\omega) \neq \emptyset$ then we call such form non-resonant. 

Moreover a 1-forms $\omega\in V(X)$ is called \emph{universally nonresonant} if the complex
\newline $(H^{\bullet}(X',\bbC), \wedge\tau^*\omega)$
%\[\ldots\to H^{i-1}(X',\C)\overset{\wedge\tau^*\omega}{\longrightarrow}H^{i}(X',\C)\overset{\wedge\tau^*\omega}{\longrightarrow}H^{i+1}(X',\C)\to\ldots\]
 is exact for any \'etale over $\tau\colon X'\to X$. 

We will refer to the sequence $(H^{\bullet}(X,\bbC), \wedge\omega)$ above as the \emph{resonance sequence}.
%Also, we call a holomorphic 1-form to be a resonant 1-form if it is not a nonresonant 1-form and has zeros.
\end{definition}

We consider more generally all local systems of rank 1. To make our notations precise recall 
\begin{definition}[Character variety]
$\Char(X) \coloneqq Hom(H_1(X, \bbZ)/\torsion, \bbC^{\star})$, 
\end{definition}
By the Riemann--Hilbert correspondence any such representation $\rho\in \Char(X)$ 
uniquely gives a local system of rank 1. 
Recall also the non-abelian Hodge
correspondence
\[\Char(X) \xrightarrow{\Psi} \Pic^{0}(X)\times H^0(X,\Omega_X^1).\]
mapping a local system $\bbL_{\rho}$ to the line bundle $\sL_{\rho}\simeq \bbL\otimes_{\bbC}\sO_X$
along with the 1-form $\omega_{\rho}$ given by the inverse image of $\bbL$ under
the exponential map
\begin{equation}
H^1(X,\bbC) \xrightarrow{\exp} \Char(X).
\label{eq:exponential}
\end{equation}
More precisely, $\omega_{\rho} = $ (0,1)-piece of $|\log(\rho)|$. Conversely, a pair $(\sL, \omega)$ 
is mapped to the kernel of $\sL\xrightarrow{\partial_{\sL}+\omega} \Omega_X^1\otimes \sL$ under this correspondence.
This induces an isomorphism of topological groups.

We now define after Simpson \cite[p.\ 365]{Sim93}; the notion of linearity of subsets of 
$\Pic^0(X)\times H^0(X,\Omega_X^1)$. Via the albanese
map $a\colon X\to \Alb_X$ we may and do identify
$\Pic^0(X)\times H^0(X,\Omega_X^1)$ with
$A^{\natural} \coloneqq \Pic^0(A)\times H^0(A,\Omega_A^1)$.
\begin{definition}[Linearity]\label{def:linhiggs}
A subset $Z\subset A^{\natural}$ is said to be \textsl{linear}
or \textsl{translates of triple tori}
if there exists finitely many morphisms of abelian varities
$p_i\colon A\to B_i$ and pairs $(\sL_i,\omega_i)$
such that $Z= (\sL_i,\omega_i)\otimes \im(B^{\natural}
\to A^{\natural})$.
\end{definition}
There is a notion of linear subsets of $\Char(A)$ as well. 
See Definition \ref{def:linearitychar}. It is a result of Simpson \cite[Theorem 3.1]{Sim93} that a closed algebraic subset $Z \subset Char(A)$ is linear if and only if its image $Z'\subset A^{\natural}$ remains algebraic. 

\begin{definition}
Define the \textsl{GV-forms} associated to $\sL\in \Pic^0(X)$
 to be the set
\[\sR(X, \sL) \coloneqq \bigcap_{p+q = i}\{\omega\in H^0(A,\Omega_A^1)|H^p((H^q(X,\Omega_X^{\bullet}\otimes \sL),\wedge\omega)) \neq 0
\}
\]
We denote by $\sR(X) \coloneqq \bigcup_{\sL}\sR(X, \sL)$
the set of all GV-forms.
\end{definition}

By the generic vanishing theory \cite[Theorem 3]{Ara}
we know that $\sR(X)\subset H^0(X,\Omega_X^1)$ is linear.
\begin{remark}
Note that when $\rho$ a unitary representation 
$(\sL, \omega)$ satisfies the Hodge decomposition
\[H^i(X, \bbL) \simeq \bigoplus_{p+q=i}H^{q}(X, \Omega_X^p\otimes \sL)\]
where $\Psi(\bbL ) = (\sL,\omega)$.
Therefore in this case, the GV-forms associated to 
$\sL$ can be interpreted
as Resonant forms associated to the local system $\bbL$,
i.e.\ 
\[\sR(X, \sL) = \sR(X, \bbL) \coloneqq \{\omega\in H^0(A,\Omega_A^1)| (H^{\bullet}(X,\bbL),\wedge\omega) \text{ is not exact}\}.\]
\end{remark} 

\subsection{Tangent cones and cohomology jump loci}
\label{sub:tc} Another way to understand the resonant 1-forms is via the tangent cone of the cohomology jump loci defined
as follows.
\begin{definition}
Given a perverse sheaf $\sP$ on $X$, define
\[\sV^i(X,\sP) \coloneqq \{\rho\in\Char(X)|
H^i(X,\sP\otimes \bbC_{\rho})\neq 0\}.\]
\end{definition}
Recall from (\ref{eq:exponential}) that given $\rho\in \Char(X)$, $\omega_{\rho} = $ (0,1)-piece of $\log|\rho|$. 
%Here we state the relevant results in the relative setting when $X$ admits a map $f\colon X\to A$ 
%to an abelian variety. The main reference for this part is \cite{sch}. We first set 
%\[\sR(f) \coloneqq \{\omega\in H^0(A,\Omega_A^1)| (H^{\bullet}(X, L(f^*\omega)), \wedge f^*\omega) \text{ is exact }\]
%Then $(\sL, \wedge\omega)\in \sM_{Higgs}^0$
%the identity component of the Higgs moduli space. $\sM_{Higgs} \simeq \Pic^{\tau}\times H^0(X,\Omega_X^1)$.
%This is isomorphic to $\Char(X) = Hom(\pi_1(X), \bbC^*)$ as a complex manifold under the map
%\[\Phi\colon \sM_{Higgs} \to \Char(X)\text{ define by } (\sL, \omega) \mapsto L(\omega).\]
Also by the generic vanishing theory \cite[Theorem 3]{Ara} we have 
\begin{equation}\begin{split}
\Psi(\sV^i(X,\bbC))
=\bigcap_{p+q = i}\{(\sL,\omega)| H^p((H^q(X,\Omega_X^{\bullet}\otimes \sL),\wedge\omega)) \neq 0
\\ \text{ for some } p,q\in\bbZ\}.
\end{split}\label{eq:}
\end{equation}
This implies the simple fact that
\[TC_{\rho}(\sV^i(X,\bbC)) = \sR(X).\]

\subsection{Cohomology jump loci on abelian variety}
In this section we collect together further properties of the
cohomology jump loci on an abelian variety $A$ of dimension $g$.
We put appropriate references wherever needed.

\begin{theorem}[Properties of CJL on $A$]
Let $\sP$ be a perverse sheave on an abelian variety $A$ with $\dim A = g$.
The
cohomology jump loci of $\sP$ satisfy the following
\begin{enumerate}
	\item Propagation property:
\[\sV^g(A, \sP) \subseteq \cdots\subseteq \sV^1(A, \sP) 
\subseteq \sV^0(A, \sP) \supseteq \sV^1(A, \sP) \supset\cdots\supset \sV^g(A, \sP).\]
Furthermore, $\sV^i(A, \sP) = \emptyset$, if $i \in [-g, g]$.
\item Codimension lower bound: for any $i \geq 0$,
$\codim \sV^i(A, \sP) \geq 2i$.
\item Generic vanishing: there exists a non-empty Zariski open subset $U \subset \Char(A)$
such that, for any $\rho\in U$, $H^i(A, \sP 
\otimes \sL_{\rho}) = 0$ for all $i \neq 0$.
\item Signed Euler characteristic property:
$\chi(A, P) \geq 0$.
Moreover, the equality holds if and only if $\sV^0(A, \sP) \neq \Char(A)$.
\item Structural property: $\sV^i(A, \sP)$ is a finite union of linear subvarieties of $Char(A)$ for any $i$.
\end{enumerate}
\label{thm:gvperverse}
\end{theorem} 



\subsection{Perverse sheaves and Decomposition theorem}
We refer the reader to
\citeSection 4.5]HTT08 
Given a morphism of smooth projective varieties, 
$f\colon X\to Y$ then we have the following
\begin{theorem}[Decomposition theorem]


\label{thm:}
\end{theorem}



\section{Generic vanishing theory on abelian varieties}
Given $f\colon X\to A$, in \ref{sub:tc} we saw that the set of relative resonant forms, i.e.
\[\sR(f)\coloneqq \{\omega\in H^0(A,\Omega_A^1)| H^k(H^{\bullet}(X,\bbC),\wedge f^*\omega) \neq 0 \text{ for some }k\}\]
can be identified with the set 
\[S(f)\coloneqq \{\omega\in H^0(A,\Omega_A^1)| H^{\bullet}(X, f^*L(\omega))\neq 0\}\]
via the tangent cone theorem.
In this section we relate the latter set with conical Lagrangian subsets of $T^*A$ arising from the decomposition theorem.
\begin{theorem}[= Thereom \ref{thm:linearity}]
Let $f\colon X\to A$ be as above. Let $\pi\colon T^*A\to H^0(A,\Omega_A^1)$ denote the projection. Then
\[\sR(f) = S(f) = \pi(\SS(Rf_*\bbC))\]
\end{theorem}
This follows from the following theorem about simple perverse sheaves on abelian variety
\begin{theorem}
Let $\sP$ be a simple perverse sheaf on abelian variety then
\[S(A,\sP)\coloneqq \{\omega\in H^0(A,\Omega_A^1)| H^k(A, \sP\otimes L(\omega)) \neq 0 \text{ for some } k\} = \pi(\SS(\sP))\]
\label{thm:perverse}
\end{theorem}

\begin{proof}[Proof of Theorem \ref{thm:linearity}]
By decomposition Theorem \ref{thm:decomp} applied to $f\colon X\to A$
we have, 
\[Rf_*\bbC\simeq \bigoplus \sP.\]
Applying Theorem \ref{thm:perverse} to each component we note that
\[\{\omega\in H^0(A,\Omega_A^1)| H^k(A, Rf_*\bbC\otimes L(\omega)) \neq 0 \text{ for some } k\} = \pi(\SS(Rf_*\bbC)).\]
The result follows from noting that $H^k(A, Rf_*\bbC\otimes L(\omega)) \simeq H^k(X, f^*L(\omega))$.
\end{proof}

\begin{proof}[Proof of Theorem \ref{thm:pervese}]
Denote by
\[S^k(A,\sP)\coloneqq \{L\in \Char(A)| H^k(A, \sP\otimes L) \neq 0\}\]
and $S(A,\sP) = \pi\circ\Psi(\bigcup_k S^k(A,\sP))\subset H^0(A,\Omega_A^1)$. Recall that $\Psi(L) = (\sL, \omega)$
as defined in \S \ref{sub:tc} and $\pi\circ\Psi(L) = \omega$
We split the proof in two cases.

\noindent \textbf{Case I: }$\chi(A, \sP)>0$. By theorem \cite[Theorem 7.4]{sch}
$\codim_{\Char(A)} S^k(A,\sP) \geq |2k|$. Thus for a general $L \in \Char(A)$
$\chi(A, \sP) = H^0(A, \sP\otimes L)$. Therefore, $S^0(A,\sP) = \Char(A)$ and hence $S(A, \sP) = H^0(A, \Omega_A^1)$.
On the other hand, let $\SS(A, \sP) = \bigcup_Z T^*_ZA$. We claim that not all such $Z$ would be fibered by tori. This follows from
the Kashiwara's index theorem 
\[\chi(A,P) = \langle CC(P), T^*_AA\rangle.\]
If $Z$ was fibered by a subabelian variety $B\subset A$, then $T^*_ZA = T^*_BA$. But $<T^*_BA, T^*_AA> = \chi(B, \bbC) = 0$ (see e.g.\ \cite[p.\ 124]{Dim}).
We conclude by Proposition \ref{van-nonsimple} that when $Z$ is not fibered by tori $\pi(T^*_ZA) = H^0(A,\Omega_A^1)$. 

\noindent \textbf{Case II: } $\chi(A, \sP)=0$. 
By the structure theorem \cite[Theorem 7.3]{sch}
we know that 
\[S^0(A, \sP) = \bigcup \rho_i\cdot(\Char(B_i) \to \Char(A))\]
for a finite collection of abelian varieties $B_i$
with maps $p_i\colon A\to B_i$. 
Therefore for all $L\in\Char(B_i)$ we have
\[H^0(B_i, p_{i_*}(\sP\otimes \bbC_{\rho})\otimes L) = 0.\]
This implies that $\chi(p_{i_*}(\sP\otimes \bbC_{\rho}))>0$. 
Since $p_i$ is smooth, the cotangent vectors in the singular supports of these perverse sheaves satisfy,
\[p_{i_{\dagger}}(\SS(p_{i_*}(\sP\otimes \bbC_{\rho}))) \subseteq \SS(\sP)\]
where $p_{i_{\dagger}} \coloneqq (p_i\times \id)\circ dp_i^{-1}$.  
Conversely, if $Z\subset A$ is so that 
$T^*_ZA \subset \SS(\sP)$, then
$Z$ has to be fibered by tori. Indeed, by a result of
Franecki and Kapranov
\cite[Corollary 1.4]{FK} we have that in the formula of
characterstic cucle, i.e\ $\CC(\sP) = \sum_Z n_Z Z$
one has $n_Z\geq 0$. Then the statement follows from the
fact that $T^*_ZA\cdot T^*_AA = 0$ iff and only if $Z$ is fibered by tori and Kashiwara's index theorem. 
Therefore there exists an abelian variety $A'$ and a map
$p\colon A\to A'$ such that $Z' \coloneqq p(Z)$ is 
not fibered by tori. Since $Z'$
is a singular support of $p_*\sP$
we have $\chi(p_*\sP) > 0$. Hence for all
$L\in \Char(A')$, $H^k(A', p_*\sP\otimes L) \neq 0$. Since
this cohomogy is a direct summand of $H^k(A, \sP\otimes p^*L)$,
this cohomology is also non-zero. Therefore,
$\Char(A') \subset \Char(A)$. Hence we have an
equality
\[\bigcup_i p_{i_{\dagger}}(\SS(p_*(\sP\otimes \bbC_{\rho}))) = \SS(\sP).\]
By the above argument it also follows that
$T^*Z\subset \SS(\sP)$ if and only if there is an $i$
such that
$T^*_Z \subset p_{i_{\dagger}}(\SS(p_*(\sP\otimes \bbC_{\rho})))$.
Now since for all $B_i$, there is a $Z_i\subset A$
such that $\pi(T^*_{Z_i}A) = p_i^*H^0(B_i, \Omega_{B_i}^1)$
we obtain
that
\[\pi(\SS(\sP) = \bigcup_ip_i^*H^0(B_i, \Omega_{B_i}^1)
= \pi(S^0(A, \sP)).\]
Finally, by propagation of cohomology jump loci 
\cite{??}, we know that $S^0(A, \sP)\supset S^k(A,\sP)$
for all $k\in \bbZ$. Hence,
$\pi(S^0(A, \sP)) = S(A, \sP)$.



%Then by a Theorem of Weissauer \cite[Theorem 2]{Wei12} we know that there exists an abelian variety $A'$, a map $p\colon A\to A'$,
%a character $\rho\in \Char(A)$ and a simple perverse sheaf $P'$ on
%$A'$ with $\chi(A', P')>0$ and such that $p^*P'  = P\otimes \bbC_{\rho}$ for some $\rho\in\Char(A)$. 
%We claim that in this case
%\[S^k(A,\sP) =  \rho^{-1}\cdot p^*(S^k(A', \sP'))\]
%where $p^*\colon \Char(A')\to \Char(A))$. 
%Indeed, if $L = \bbC_{\rho^{-1}}\otimes p^*L'$ for some $L'\in S^k(A'\sP')$ then
%\[H^k(A, \sP\otimes L) = H^k(A, p^*\sP'\otimes\bbC_{\rho}\otimes \bbC_{\rho^{-1}}\otimes p^*L')=
%H^k(A', \sP'\otimes L'\otimes Rp_*\bbC_A).\]
%Since $\bbC\overset{\oplus}{\into} Rp_*\bbC_A$, we obtain the desired non-vanishing. Conversely, if $L\in S^k(A,\sP)$,
%then by a similar argument we have 
%$H^k(A', \sP'\otimes Rp_*(L\otimes \bbC_{\rho})) \neq 0$.
%Since $p$ is smooth $Rp_*(L\otimes \bbC_{\rho})$ is a local system. Furthermore since $\pi_1(A)$ is abelian the local system
%splits into 1-dimensional simple representations. Let $\eta$
%be one such representation for which 
%$H^k(A', \sP'\otimes \bbC_{\eta})\neq 0$. Then,






\end{proof}






%\section{Proof of Theorem \ref{thm:linearity}}
%Given an $n$-dimensional smooth complex irregular projective variety $X$, we are interested in the set $V(X)\subseteq H^0(X, \Omega_X^1)$ of global holomorphic 1-forms
%that admit zeros on $X$. It is known that such zeros of a 1-form $\omega$ 
%are supported on the support of the cohomologies of the long exact sequence
%
%
%In particular if $\omega$ is a 1-form such the the \textit{resonance sequence}
%\[H^0(X,\bbC)\overset{\wedge\omega}{\to}H^1(X,\bbC)\overset{\omega}{\to}\cdots\]
%is not exact everywhere the set of zeros $Z(\omega)\neq \emptyset$.
%
%On the other hand from the generic vanishing theory it is known that the set of such forms, called resonance forms following Dimca,
%is a finite union of linear subspaces of $H^0(X,\Omega_X^1)$.
%
%Perhaps known to the experts already, we show in particular that this linear subspace is in fact coming from a 
%finite union of triple
%tori in the sense of Simpson. In order to make the statement precise we need to introduce some terminologies and results. 
%\begin{definition}[Resonant and (universally) nonresonant forms]\label{def:resonance}
%We call a 1-form $\omega$ \emph{resonant} if the complex 
%\[(H^{\bullet}(X,\bbC), \wedge\omega)\coloneqq [\ldots\to H^{i-1}(X,\C)\overset{\wedge\omega}{\longrightarrow}H^{i}(X,\C)\overset{\wedge\omega}{\longrightarrow}H^{i+1}(X,\C)\to\ldots]\]
 %is not exact. Accordingly the resonance varieties of $X$ is defined as:
%$$ \sR^i(X):= \{\omega\in H^1(X,\Omega_X^1)\mid  H^i  (H^*(X,\bbC), \wedge \omega) \neq 0 \} $$
%and we denote by $R(X) \coloneqq \cup_i R^i(X)$.
%\end{definition}
%
%
%Let $\C^{\ast}$ be the multiplicative group of non-zero complex numbers.
%The {\it character variety} $\Char(X)$ of $X$ is the identity component of the moduli space of rank-one $\bC$-local systems on $X$, i.e., 
%\begin{center}
%$\Char(X):= 
%\Hom (H_1(X,\bZ)/\text{Torsion}, \bC^*)\cong (\bC^*)^{b_1(X)}.$
%\end{center}
%
%\bd The {\it $i$-th cohomology jump locus of $X$} is defined as: 
%$${\sV^{i}(X)=\lbrace \rho\in \Char(X) \mid  H^{i}(X,L_{\rho})\neq 0 \rbrace},$$ 
%where $L_{\rho}$ is the unique rank-one $\bC$-local system  on $X$ associated to the representation $\rho\in \Char(X) $.\ed
%
%
%\begin{theorem}[Structure Theorem] Let $X$ be a smooth complex projective variety. Then the jump loci $\sV^i(X)$ is a finite union of torsion translated sub-tori of $\Char(X)$.
%\end{theorem}
%%\br 
%%In fact, up to a torsion translate, every irreducible component of $\sV^i(X)$ is induced by a sub-Hodge structure of $H^1(X, \Q)$. To be precise, there exists an algebraic surjective map $f: \Alb(X) \to A$ from the Albanese variety of $X$ to an abelian variety $A$ such that this irreducible component is, up to a torsion translate, the image of the following map $$\Char(A) \hookrightarrow \Char(\Alb(X)) \cong \Char(X) .$$
%%Here the last isomorphism is induced by the Albanese map.
%%\er 
%
%
%
%
%\begin{theorem} Let $X$ be a smooth complex projective variety.  Then we have the following tangent cone formula:
%$$ TC_1( \sV^i(X)) = \sR^i(X)$$
%under the exponential map $M_{\DR}(X) \to \Char(X)$. 
%\end{theorem} 
%The reason behind this theorem is the fact that the smooth projective variety is formal.  For a proof, please see \cite{S}.
%
%Together with the structure theorem, we get the following. 
%
%\bc Let $X$ be a smooth complex projective variety. Then every irreducible component of $\sR^i(X)$ is a linear sub-vector space of $H^1(X,\C)$.
%\ec
%
%
%
%
%Finally, for a subvariety $Z\subset X$ of any smooth projective variety $X$, by 
%$T^*_ZX$ we denote the closure of the conormal bundle $T^*_{Z^{\reg}}X$ of the regular locus of $Z^{\reg}$ of $Z$
%inside the cotangent bundle $T^*X$ of $X$. With this notation at our disposal note that when $X$ is an abelian variety
%$A$, the image of $T^*_ZA\to H^0(A,\Omega_A^1)$ under the projection $\pi\colon T^*A \to H^0(A,\Omega_A^1)$ is precisely the set of 1-forms that admits a zero on the regular locus of $Z$.
%We show that whether or not a 1-form $\omega\in H^0(A,\Omega_A^1)$
%comes from $T^*_ZA$ for some $Z$, is determined by one of the most fundamental
 %structural properties of subvarieties of abelian varieties;
 %namely, whether or not
%$Z$ is of general type. By a result of Ueno \cite[Theorem 10.9]{Uen75} this is equivalent to $Z$ being not fibered by any subabelian varieties of $A$. 
%When $Z$ is smooth 
%this seems to be a classical fact that whenever the normal bundle of $Z$ is ample all $\omega|_Z$ admits finitely many zeros on $X$
%(see e.g.\ \cite[Lemma 3.1]{Deb}). This was generalised to subvarieties of simple abelian varieties by 
%Hacon and Kov\'acs \cite[Proposition 3.1]{HK05}. A special case of more recent result of Popa and \cite{PS14} suggests that this is true whenever $Z$ is smooth and
%$\omega_Z$ is big,
%i.e.\ $Z$ is of general type. Although perhaps well-known to experts, we could not find any appropriate sources in literature. Nonetheless, our proof is inspired
%by their original argument in\cite[Proposition 3.1]{HK05} and the proof of \cite[Lemma 3.1]{Deb} in Debarre's expository article
%
%\begin{proposition}
%\label{van-nonsimple}
%Let $A$ be an abelian variety and $Z$ be an irreducible proper subvariety of $A$. Then the following are equivalent
%
%(1) $Z$ is not fibred by tori. 
%
%(2) The projection $\pi\colon T^*_ZX\to H^0(A, \Omega_A^1)$ is surjective.
%
%(3) For general holomorphic 1-forms $\omega\in H^0(A, \Omega_A^1)$, the restricted holomorphic 1-form $\omega|_{Z_{\textnormal{reg}}}$ admit zeros at finitely many points on the smooth locus $Z_{\textnormal{reg}}$.
%\end{proposition}
%
%
%
%\section{}
%Our theorem is the following
%Let $X$ be an irregular smooth projective variety and $a\colon X\to \Alb(X)$ the albanese morphism. We identify $M_{\DR}^0(X)$ with $M_{\DR}^0(A)$ 
%to identify
%\[\sR^i(X) = \{\omega\in H^0(A, \Omega_A^1)| H^i(H^*(X, \bbC), \wedge a^*\omega)\neq 0\}\]
%and rewrite the tangent cone theorem as 
%\[TC_1(A, \bbR a_*\bbC) = \sR^i(X).\]
%
%
%
%
%The first equality is a consequence of the tangent cone theorem and the fact that
%$H^i(X, \bbC) = H^i(A, \bbR a_*\bbC) = \cup_{\alpha} \sV^i(A, P_{\alpha})$.
%Therefore the main point of the argument is to show the second equality.
%
%\begin{proof}
%Note that if there is an $\alpha$ such that $\chi(P_{\alpha}) \neq 0$, then note that by generic vanishing theorem for
%a general character $\rho \in \Char(\pi_1(A), \bbC^*)$ $\chi(P\otimes \bbC_{\rho}) = H^0(A, P\otimes \bbC_{\rho}) \neq 0$. Therefore,
%$\sV^0(X, P) = \Char(\pi_1(A), \bbC^*)$. Therefore $TC_1(\sV(X, \bbR a_*\bbC)) = TC_1(\sV(A, P)) = H^0(A, \Omega_A^1) =  \sR(X)$.
%
%On the other hand, if characteristic cycle $\Ch(P_{\alpha}) = \cup_Z T^*_ZA$ then we claim that there exists a $Z$ in the singular support of $P$ such that
%$Z$ is not fibered by tori. Indeed,
%by Kashiwara's index theorem 
%\[\chi(A,P) = <CC(P), T^*_AA>\]
%If $Z$ was fibered by a subabelian variety $B\subset A$, then $T^*_ZA = T^*_BA$. But $<T^*_BA, T^*_AA> = \chi(B, \bbC) = 0$ (see e.g.\ [Dimca, p.\ 12	4]).
%We conclude by Proposition \ref{van-nonsimple} that when $Z$ is not fibered by tori $\pi(T^*_ZA) = H^0(A,\Omega_A^1)$. 
%
%Now let $\chi(A, P) = 0$ for all $P$ appearing in the decomposition of $\bbR a_*\bbC$. By a result of Kr\"amer--Weissauer 
%
%\end{proof}
%
%
%




\begin{thebibliography}{ADMSP}

\bibitem[Dim]{Dim} A. Dimca, {\it Sheaves in Topology} Springer-Verlag Berlin Heidelberg New York in 2004.
\textsc{doi:} \href{https://doi.org/10.10071978-3-642-18868-8}{\ttfamily 10.10071978-3-642-18868-8}.
\textsc{isbn:} {\ttfamily 978-540-20665-1 }.

\bibitem[BWY]{BWY} N. Budur, B . Wang, Y. Yoon, {\it Rank one local systems and forms of degree one}, Int. Math. Res. Not. (2016), no.13, 3849-3855.


\bibitem[LMW]{LMW} Y. Liu, L. Maxim, B. Wang, \textit{Aspherical manifolds, Mellin transformation and a question of Bobadilla-Kollar}, arXiv:2006.09295.

\bibitem[HK05]{HK05} C. D. Hacon, S. J. Kov\'acs, \textit{Holomorphic one-forms on varieties of general type}, Ann. Sci. \'Ecole Norm. Sup. \textbf{38} (2005), no. 4, 599--607.


\bibitem[Suc16]{S} A. Suciu, {\it Around the tangent cone theorem}, Configuration Spaces: Geometry, Topology and Representation Theory, 1-39, Springer INdAM series, vol. 14, Springer, Cham, 2016

\bibitem[Wei12]{Wei12} R.\ Weissauer, ``Degenerate perverse sheaves on abelian varieties", \href{https://arxiv.org/abs/1204.2247}{\ttfamily arXiv:1204.2247[math.AG]}, 2012.




\end{thebibliography}



%------------------------------------------------------------------
\end{document}
%------------------------------------------------------------------




