\pdfoutput=1
\documentclass[12pt,reqno]{amsart}
\usepackage[letterpaper,margin=1in,footskip=0.25in]{geometry}
\usepackage{mathrsfs}
\usepackage{amssymb}
\usepackage{mathtools}
\usepackage{tikz-cd}
\usepackage{enumitem}


\PassOptionsToPackage{pdfusetitle,pagebackref,colorlinks}{hyperref}
\usepackage{bookmark}
\hypersetup{
  linkcolor={red!70!black},
  citecolor={green!70!black},
  urlcolor={blue!80!black}
}

\makeatletter
  \def\th@plain{
  \thm@headfont{\bfseries} % Heading font is bold
  \thm@notefont{\itshape} % Note is same as heading
  \itshape% Regular text is also italic
}
\makeatother

\makeatletter
  \def\th@definition{
  \thm@headfont{\bfseries} % Heading font is italic
  \thm@notefont{\bfseries} % Note is same as heading
}
\makeatother

\makeatletter
  \def\th@remark{
  \thm@headfont{\bfseries} % Heading font is italic
  \thm@notefont{\bfseries} % Note is same as heading
	}
\makeatother
\newtheorem{theorem}{Theorem}
\newtheorem{lemma}[theorem]{Lemma}
\newtheorem{proposition}[theorem]{Proposition}
\newtheorem{corollary}[theorem]{Corollary}
\newtheorem{claim}[theorem]{Claim}
\newtheorem{conjecture}[theorem]{Conjecture}
\newtheorem{step}{Step}[subsection]
\renewcommand{\thestep}{\arabic{step}}

\newtheorem{alphtheorem}{Theorem}
\renewcommand{\thealphtheorem}{\Alph{alphtheorem}}
\newtheorem{alphtheoremprime}{Theorem}
\renewcommand{\thealphtheoremprime}{\Alph{alphtheoremprime}'}


\theoremstyle{question}
\newtheorem{question}[theorem]{Question}
\theoremstyle{definition}
\newtheorem{definition}[theorem]{Definition}
\newtheorem{notation}[theorem]{Notation}

\theoremstyle{remark}
\newtheorem{remark}[theorem]{Remark}

\newtheoremstyle{cited}{.5\baselineskip\@plus.2\baselineskip\@minus.2\baselineskip}{.5\baselineskip\@plus.2\baselineskip\@minus.2\baselineskip}{\itshape}{}{\bfseries}{\bfseries .}{5pt plus 1pt minus 1pt}{\thmname{#1}\thmnumber{~#2}\thmnote{ \normalfont#3}}
\theoremstyle{cited}
\newtheorem{citedthm}[theorem]{Theorem}
\newtheorem{citedconj}[theorem]{Conjecture}
\newtheorem{citedlem}[theorem]{Lemma}
\newtheorem{citedprop}[theorem]{Proposition}

\newtheoremstyle{citeddef}{.5\baselineskip\@plus.2\baselineskip\@minus.2\baselineskip}{.5\baselineskip\@plus.2\baselineskip\@minus.2\baselineskip}{}{}{\bfseries}{\bfseries .}{5pt plus 1pt minus 1pt}{\thmname{#1}\thmnumber{~#2}\thmnote{ \normalfont#3}}
\theoremstyle{citeddef}
\newtheorem{citednot}[theorem]{Notation}


%%==============Yongqiang's typsets====================%%

\newcommand{\CN}{\mathbb{C}^{n+1}}
\newcommand{\CP}{\mathbb{CP}^{n+1}}
\newcommand{\U}{\mathcal{U}}
\newcommand{\C}{\mathbb{C}}
\newcommand{\Z}{\mathbb{Z}}
\newcommand{\Hom}{\mathrm{Hom}}
\newcommand{\Q}{\mathbb{Q}}
\newcommand{\K}{\mathcal{L}}
\newcommand{\V}{\mathcal{V}}

\def\be{\begin{equation}}
\def\ee{\end{equation}}

\def\bt{\begin{theorem}}
\def\et{\end{theorem}}

\def\bc{\begin{corollary}}
\def\ec{\end{corollary}}

\def\br{\begin{remark}}
\def\er{\end{remark}}

\def\bp{\begin{proposition}}
\def\ep{\end{proposition}}

\def\bl{\begin{lemma}}
\def\el{\end{lemma}}

%\def\bn{\begin{enumerate}}
%\def\en{\end{enumerate}}

\def\bex{\begin{ex}}
\def\eex{\end{ex}}

\def\bd{\begin{definition}}
\def\ed{\end{definition}}








%\DeclareMathOperator{\Pic}{Pic}                  % Pic

%\newcommand{\PP}{\mathcal{P}}           % Poincare line bunle

\DeclareMathOperator{\Supp}{Supp}                % Supp
\DeclareMathOperator{\codim}{codim}              % codim
\DeclareMathOperator{\mreg}{mreg}                % mreg
\DeclareMathOperator{\reg}{reg}                  % reg
\DeclareMathOperator{\sing}{sing}                  
\DeclareMathOperator{\id}{id}                    % id
\DeclareMathOperator{\obj}{Obj}
\DeclareMathOperator{\ad}{ad}
\DeclareMathOperator{\morph}{Morph}
\DeclareMathOperator{\enom}{End}
\DeclareMathOperator{\iso}{Iso}
\DeclareMathOperator{\Exp}{Exp}
\DeclareMathOperator{\homo}{Hom}
\DeclareMathOperator{\enmo}{End}
\DeclareMathOperator{\spec}{Spec}
\DeclareMathOperator{\fitt}{Fitt}
\DeclareMathOperator{\odr}{\Omega^\bullet_{\textrm{DR}}}
\DeclareMathOperator{\rank}{Rank}
\DeclareMathOperator{\gdeg}{gdeg}
\DeclareMathOperator{\Alb}{Alb}
\DeclareMathOperator{\alb}{alb}
\DeclareMathOperator{\Ann}{Ann}



%\DeclareMathOperator

\DeclareMathOperator{\Char}{Char}
\DeclareMathOperator{\CC}{CC}
\DeclareMathOperator{\kn}{Ker}
\DeclareMathOperator{\im}{Im}


\def\ra{\rightarrow}


\def\bone{\mathbf{1}}
\def\bC{\mathbb{C}}
\def\cM{\mathcal{M}}
\def\cV{\mathcal{V}}
\def\Def{{\rm {Def}}}
\def\cR{\mathcal{R}}
\def\om{\omega}
\def\wti{\widetilde}
\def\al{\alpha}
\def\End{{\rm {End}}}
\def\Pic{{\rm Pic}}
\def\bP{\mathbb{P}}
\def\cH{\mathcal{H}}
\def\bL{\mathbb{L}}
\def\cX{\mathcal{X}}
\def\cI{\mathcal{I}}
\def\pa{\partial}
\def\cY{\mathcal{Y}}
\def\cD{\mathcal{D}}
\def\cO{\mathcal{O}}
\def\lra{\longrightarrow}
\def\bQ{\mathbb{Q}}
\def\ol{\overline}
\def\cL{\mathcal{L}}
\def\bH{\mathbb{H}}
\def\bZ{\mathbb{Z}}
\def\bW{\mathbf{W}}
\def\bV{\mathbf{V}}
\def\bM{\mathbf{M}}
\def\eps{\epsilon}
\def\ul{\underline}
\def\lam{\lambda}
\def\sX{\mathscr{X}}
\def\bN{\mathbb{N}}


%%==========Yagna's typsets============%%
\usepackage{tikz-cd}
\usepackage{enumitem}

\PassOptionsToPackage{pdfusetitle,pagebackref,colorlinks}{hyperref}
\usepackage{bookmark}
\hypersetup{
  linkcolor={red!70!black},
  citecolor={green!70!black},
  urlcolor={blue!80!black}
}

%Mathcal Letters =====================
\newcommand{\sA}{\mathcal{A}}
\newcommand{\sB}{\mathcal{B}}
\newcommand{\sD}{\mathcal{D}}
\newcommand{\sF}{\mathcal{F}}
\newcommand{\sG}{\mathcal{G}}
\newcommand{\sH}{\mathcal{H}}
\newcommand{\sK}{\mathcal{K}}
\newcommand{\sL}{\mathcal{L}}
\newcommand{\sM}{\mathcal{M}}
\newcommand{\sN}{\mathcal{N}}
\newcommand{\sO}{\mathcal{O}}
\newcommand{\sP}{\mathcal{P}}
\newcommand{\sQ}{\mathcal{Q}}
\newcommand{\sR}{\mathcal{R}}
\newcommand{\sT}{\mathcal{T}}
\newcommand\sV{{\mathcal V}}
\newcommand\sW{{\mathcal W}}
\newcommand{\sZ}{\mathcal{Z}}

%mathbb Letters======
\newcommand{\bbA}{\mathbb{A}}
\newcommand{\bbB}{\mathbb{B}}
\newcommand{\bbC}{\mathbb{C}}
\newcommand{\bbE}{\mathbb{E}}
\newcommand{\bbG}{\mathbb{G}}
\newcommand{\bbH}{\mathbb{H}}
\newcommand{\bbK}{\mathbb{K}}
\newcommand{\bbL}{\mathbb{L}}
\newcommand{\bbM}{\mathbb{M}}
\newcommand{\bbN}{\mathbb{N}}
\newcommand{\bbP}{\mathbb{P}}
\newcommand{\bbQ}{\mathbb{Q}}
\newcommand{\bbR}{\mathbb{R}}
\newcommand{\bbV}{\mathbb{V}}
\newcommand{\bbZ}{\mathbb{Z}}



%Script Letters ======================
\newcommand{\frf}{\mathfrak{f}}
\newcommand{\frM}{\mathfrak{M}}

\newcommand{\scrL}{\mathscr{L}}
\newcommand{\crI}{\mathscr{I}}
\newcommand{\scrK}{\mathscr{K}}
\newcommand{\scrB}{\mathscr{B}}
\newcommand{\scrC}{\mathscr{C}}
\newcommand{\scrD}{\mathscr{D}}
\newcommand{\scrE}{\mathscr{E}}
\newcommand{\scrI}{\mathscr{I}}
\newcommand{\scrQ}{\mathscr{Q}}
\newcommand{\scrR}{\mathscr{R}}
\newcommand{\scrX}{\mathscr{X}}
\newcommand{\scrY}{\mathscr{Y}}
\newcommand{\scrF}{\mathscr{F}}
\newcommand{\Dred}{\lceil D\rceil}

%Arrow Style =========================
\newcommand{\into}{\hookrightarrow}
\newcommand{\onto}{\rightarrow\hspace*{-.14in}\rightarrow}
\def\acts{\curvearrowright}

%Math Operators ======================



%\DeclareMathOperator{\alg}{alg}
\DeclareMathOperator{\BM}{BM}
\DeclareMathOperator{\Bs}{Bs}
\DeclareMathOperator{\Bsp}{\mathbf{B}_+}
\DeclareMathOperator{\SB}{\mathbf{B}}

\DeclareMathOperator{\coh}{coh}
\DeclareMathOperator{\Coker}{Coker}
 \renewcommand{\div}{\text{div}}

\DeclareMathOperator{\DR}{DR}
\DeclareMathOperator{\Ch}{Ch}
\DeclareMathOperator{\discrep}{discrep}
\DeclareMathOperator{\exc}{exc}

\DeclareMathOperator{\free}{free}


\DeclareMathOperator{\HHom}{\mathcal{H}\!\mathit{om}}
\DeclareMathOperator{\image}{Im}


\DeclareMathOperator{\Ind}{Ind}
\DeclareMathOperator{\Ker}{Ker}
\DeclareMathOperator{\Lie}{Lie}
\DeclareMathOperator{\op}{op}


\DeclareMathOperator{\qcoh}{qcoh}

\DeclareMathOperator{\dr}{DR}

\DeclareMathOperator{\RHHom}{\mathbf{R}\mathcal{H}\!\mathit{om}}


\DeclareMathOperator{\Spf}{Spf}
\DeclareMathOperator{\Proj}{Proj}
\DeclareMathOperator{\tors}{tors}
\DeclareMathOperator{\torsion}{torsion}
\DeclareMathOperator{\Var}{Var}
\DeclareMathOperator{\Sym}{Sym}


\newcommand{\Ab}{\mathbf{Ab}}
\newcommand{\Aff}{\mathbf{Aff}}
\newcommand{\tB}{\tilde{B}}
%\newcommand{\et}{{\acute{e}t}}

\newcommand{\Sets}{\mathbf{Sets}}
%\newcommand{\cX}{\bar{X}}
\newcommand{\cP}{\bar{P}}
%\newcommand{\cV}{\bar{V}}
\newcommand{\cW}{\bar{W}}
\newcommand{\sorry}[1]{\textcolor{red}{#1}}
\newcommand{\dual}[1]{\sD^{\Omega}_{M^{#1}}}



\title{}








\begin{document}  
\title[Decomposition theorem, generic vanishing and holomorphic 1-forms]{Decomposition theorem, generic vanishing and holomorphic 1-forms} 

\author{Yajnaseni Dutta}

%\address{}
%\email{}

\author{Feng Hao}

%\address{}
%\email{}

\author{Yongqiang Liu}

%\address{}
%\email{}


%\date{\today}
%\subjclass[2010]{} 
%\keywords{} 



\begin{abstract} 
The goal of this article is two-fold; first we establish an interesting link
between some sets arising in the generic vanishing theory and
the decomposition theorem for albanese morphism of a smooth projective varieties over the complex numbers. As an upshot, we show that when a smooth projective variety with simple albanese admits a fibre bundle structure over the circle, its albanese morphism is cohomologically trivial, i.e.\ the decomposition theorem behaves like that of a smooth morphism. Such fibre bundle structure is conjecturally equivalent to the existence of 1-forms without zeros. We also present stronger cohomological properties for the albanese morphism assuming the latter. This brings us to our second goal which is to address the question of linearity of set of holomorphic 1-forms
admitting zeros. It was shown by Carrel and Lieberman that the set of holomorphic vector fields that admit zeros form a linear subspace. In the light of linearity statements coming out of the generic vanishing theory, we conjecture a similar behaviour for holomorphic 1-forms and prove it for the set
of such 1-forms admitting zeros of codimension 1. 
\end{abstract}

\maketitle
\tableofcontents
\section{Introduction}\label{intro}
Given a smooth porjective complex variety $X$, in this article we relate the cohomology jump loci of the constant sheaf $\bbC_X$ to the 
stratification arising in the decomposition theorem for the albnese morphism. Such relation fell out of our interest in the
study of holomorphic 1-forms admitting zeros. It has been indicated by plethora of results
(\cite{GL87, HK05, LZ05,
Sch19, HS19, PS14} to name just a few) that a lot of the geometry and topology of the
variety depends on zeros of holomorphic 1-forms. Conjecturally the following two statements
are also a bridge between the real and the complex worlds.
\begin{enumerate}
	\item\label{kot1} $X$ admits a global holomorphic 1-form $\omega$
	such that $Z(\omega) = \emptyset$, i.e.\ there does \textsl{not}
	exist any $x\in X$ such that $\omega(T_xX) = 0$. 
	\item\label{kot2} $X$ admits a global holomorphic 1-form $\omega$
	such that $Z(\omega) = \emptyset$, $X$ admits a $\scrC^{\infty}$-fibration over the circle.
\end{enumerate}
The direction (1) $\Rightarrow$ (2) was shown by Tischler,
\cite{Tis70}. The opposite direction is conjectural (attributed
to Kotschik in \cite{Sch19}) and
was recently established by Shreieder in dimension 2 and in dimension 3 by a joint work of the second author with Schreieder
\cite{HS19}. We show the following 



\begin{alphtheorem}\label{thm:smooth}
Given a $f\colon X\to A$ be a morphism from a smooth projective variety $X$ of dimension $n$ to a simple abelian variety $A$ 
of dimension $q$ such that
both $X$, and $A$ admit $\scrC^{\infty}$-fibre bundle structures over $S^1$ and the commutative diagram
\[\begin{tikzcd}
	X\ar[dr]\ar[d, "f"]& \\
	A\ar[r]&S^1
\end{tikzcd}.\] 
Then $f$ is cohomologically trivial, i.e.\ we have the following
decomposition
\[\bbR f_*\bbC[n] \simeq \bigoplus_i \bbR^i f_*\bbC[q-i]\]
and $\bbR^i f_*\bbC$ are local system. In particular, the
singular fibres of $f$ carry a pure Hodge structure.

Furthermore, 
$R^if_*\bbZ$ are also local systems for 
all $i\geq 0$.
%Then $f$ is cohomologically a fiber bundle if and only if there is a global holomorphic 1-form $\omega$ on $A$ such that $f^*\omega$ is nowhere vanishing.  

\end{alphtheorem}

Assuming (\ref{Kot1}), which is currently a stronger (yet conjecturally
equivalent) hypothesis we have
\begin{alphtheorem}\label{thm:cxsmooth}
Let $f\colon X\to A$ be a surjective morphism to a simple abelian
variety such that there exists a holomorphic 1-form
$\omega$ with $Z(f^*\omega) = \emptyset$ then $\bbR f_*\bbL$
admits the decomposition
\[\bbR f_*\bbL \simeq \bigoplus_i \bbR f_*\bbL[-i]\]
for any semi-simple local system $\bbL$ on $X$. 
\end{alphtheorem}
In fact a bit more can be said; When $A$ is not necessarily simple existence of a nowhere vanishing global holomorphic 1-form
is equivalent to certain restrictions on the singular support of 
these pushforwards. See Theorem \ref{thm:nonvanishing} for more details. 

The link between such
 real geometry and complex geometry of $X$ can be partially
understood
by the generic vanishing theory. More precisely, Schreieder 
showed \cite[Theorem 1.2]{Sch19} that a real $\scrC^{\infty}$-fibre bundle structure over $S^1$ gives rise to the existence of 
a global
holomorphic 1-form that together with any degree zero line bundle give rise to a local system without any cohomology. These
local systems do not come from the cohomology jump loci of the constant sheaf $\bbC_X$. More on that shortly. We prove a more general
version of this statement in a relative setting (see Theorem
\ref{thm:schreieder}. 

 Another key input here is the relationship between the generic vanishing theory and the decomposition
theorem as stated in the following result. We need the following notations
Denote by $\SS(Ra_*\bbC)\subset T^*\Alb_X$ the singular support of $Ra_*\bbC$ under the albanese morphism $a\colon X\to \Alb_X$ and 
let $\pi\colon T^*\Alb_X\to 
H^0(\Alb_X, \Omega_{\Alb_X}^1)$ is the projection. A result of
Kashiwara giving an estimate on the behaviour of the singular
support of $\bbR f_*\bbC$ (see
Theorem \ref{thm:kas-ss}) ensures that all 1-forms
$\omega\in \pi(\SS(\bbR a_*\bbC))$ admit zeros. Let $\sR(X)$ denote the (1,0)-piece of the union of generic vanishing loci
\[\sigma^{p,q}(X) \coloneqq \{(\sL,\omega)\in \Pic^0(X)\times H^0(\Alb_X,\Omega_{\Alb_X}^1)|H^q(H^{p}(X, \Omega_X^{\bullet}\otimes \sL), \wedge a^*\omega) \neq 0\}.\]
Via the non-abelian Hodge correspondence a different way to understand the union over all $p,q$ of such set is via the Lie algebra associated to the  the cohomology jump loci 
\[\sV(X,\bbC) \coloneqq \{\rho\in\Char(X)| \bbH^i(X, \bbC_{\rho})\neq  0\text{ for some } i\} \]
where $\bbC_{\rho}$ is the local system associated 
to the character $\rho\in Hom(H_1(X,\bbZ)/\torsion, \bbC^{\star})$.
The Lie algebra of this set lies in $H^1(X,\bbC)$. Its holomorphic part coincides with $\sR(X)$ above
(see \S \ref{sec:gv} for more details). It is known that the tangent cone of $\sV(X,\bbC)$ is a sub-Hodge structure of $H^1(X,\bbC)$.
We show
\begin{alphtheorem}
Let $a\colon X\to \Alb_X$ denotes the albanese morphism. Then
\[\sR(X) =  \pi(\SS(Ra_*\bbC)). \]
\label{thm:linearity}
\end{alphtheorem}
For a relative version of this theorem see Theorem \ref{thm:rellinearity}.
%\begin{remark}[Previous results]
%\begin{enumerate}
%\item \label{item:ps} When $X$ is of general type, it was conjectured in
	%\cite{HK05, LZ05} and was proved
	%by Popa and Schnell \cite{PS14} every global holomorphic 1-		
	%form vanishes on $X$. 
%\item \label{item:hk} On the opposite extreme for any $A$, not necessarily simple when $f$ is singular along a divisor of general
		%type in $A$, Hacon and Kov\'acs \cite[Proposition 3.5.]{HK05} show that
		%$f^*\omega$ always admits zero. See Corollary \ref{cor:hk}
		%for a reinterpretation of their argument. 
%\end{enumerate}	
%\end{remark}

The theorem follows from a similar statement
about perverse sheaves on abelian varieties. Recall
\begin{definition}
The i-th\textsl{ cohomology jump loci} of a constructible sheaf $\sP$ on an abelian variety $A$ is defined to be the set
\[\sV^i(A, \sP) \coloneqq \{\rho\in\Char(X)| H^i(A, \sP\otimes \bbC_{\rho}) \neq 0\}\]
where we denote by $\bbC_{\rho}$ the local system associated to $\rho\in \Char(X)$.
\label{def:cjl}
\end{definition}
\begin{alphtheorem}
Let $A$ be an abelian variety. Let $\sP$ be a simple perverse sheaf with complex coefficient on $A$. 
Then we have the equality
$$\pi(\SS(\sP)) = \text{(1,0)-piece of }\bigcup_{\rho} \rho^{-1} TC_{\rho} \sV^0(A,\sP), $$
where the union is running over representative points from every irreducible components of $\sV^0(A,P)$ and
$\rho^{-1}TC_{\rho} \sV^0(A,P) \subseteq H^1(X, \bbC)$ denotes the tangent cone at $\rho$. 
\label{thm:perverse}
\end{alphtheorem}

The key technique we
use to relate the generic vanishing theory on abelian varieties with the support of $Rf_*\bbC$ is Kashiwara's
global index theorem (see Theorem \ref{thm:index}). In terms of simple holonomic D-modules
on an abelian variety we can interpret the above result as

\setcounter{alphtheoremprime}{3}
\begin{alphtheoremprime}
Let $\sM$ be a simple holonomic D-module on an abelian variety $A$.
Then the set
\[\sR_d(X)\coloneqq \{\omega\in H^0(A, \Omega_A^1)|\exists \sL \in \Pic^0(A)\text{ and } k\in\bbZ_{\geq 0} \text{ with }
H^{k}(A, \dr(\sM\otimes(\sL, \nabla_{\omega}))) \neq 0\}\]
satisfies
\[\sR_d(X) = \pi(\SS(\sM)).\]
Here $\nabla_{\omega}\colon \sL \to \Omega_A^1\otimes \sL$
denotes the integrable connection on $\sL$ associated to the Higgs
field $\omega$ and $\SS(\sM)$ is the singular support of $\sM$.
\label{thm:dlinearity}
\end{alphtheoremprime}
A consequence of Theorem \ref{thm:linearity} is that $\sR(X)\subseteq V(X)$. This was also known by the generic vanishing theory
(see \cite{BWY} for more general statement). 


This brings us to the question of linearity. It is well-known \cite[p.\ 311]{Ara92} that $\sR(X)$  is a finite union of linear subspaces of $H^0(\Alb_X, \Omega_{\Alb_X}^1)$. 
On the other hand, by Proposition \ref{van-nonsimple} we know that for any subvariety $Z$ of an abelian variety $A$, the collection
of 1-forms appearing in the conormal spaces $\overline{T^*_{Z^{\reg}}A}$ is also linear in the above sense. Furthermore, both sets are subsets of the set $V(X)$ of holomorphic 1-forms that admit zeros. Hence, the result of Theorem \ref{thm:linearity} can be interpreted as identifying these two linear pieces of $V(X)$. 
However, in the spirit of Carrell and Lieberman \cite{CL73}, who showed that
the set of global holomorphic tangent vector fields with zeros
is linear,
one can ask the following 
\begin{question}
Is the following set
\[V(X):=\{ \omega\in H^0(X, \Omega_X^1) | Z(\omega)\neq \emptyset\}\]
linear, i.e.\ a finite
union of linear subspaces of the vector space $H^0(X, \Omega_X^1)$?

Or, more generally
for $i\geq 0$, is the following 
\[V^i(X):=\{ \omega\in H^0(X, \Omega_X^1) | \codim(Z(\omega))\leq i\}\]
linear, i.e.\ a finite
union of linear subspaces of the vector space $H^0(X, \Omega_X^1)$?

\end{question} 

The answer to this question is yes up-to 3-folds by
a result of the second author obtained jointly with Schreieder
\sorry{insert theorem reference}. 


%\cite[Theorem 4.2]{Sim93} (see also \cite{DiPa13} from where
%we borrow the terminology). 
%Henceforth we shall call finite unions of subvector spaces as \emph{linear subvarieties}. 
%In the relative setting
%of a morphism $f\colon X\to A$ from a smooth projective variety to an abelian variety $A$,
%we similarly define
%\[V(f):=\{ \omega\in H^0(A, \Omega_A^1) | Z(f^*\omega)\neq \emptyset\}.\]



\begin{remark}\label{rem:linearity}
\begin{enumerate} 
\item Note that if $\chi(X)>0$, we have $\sR(X) = V(X) = H^0(X, \Omega_X^1)$. Hence in this case the set of 1-forms that admit zeros is linear as already mentioned above. 
This follows immediately from the generic vanishing theory. Indeed, by Hodge decomposition
we have
$H^k(X,\bbC) \simeq \bigoplus_{p+q = k} H^p(X,\Omega_X^q)$. Hence the complexes $(H^p(X, \Omega_X^{\bullet}), \wedge\omega)$ can be summed together to form the complex $(H^p(X, \bbC), \wedge\omega)$. Since $\chi(X)>0$ the latter complex cannot be exact. Hence $H^0(X,\Omega_X^1)
= \sR(X)$.
\item \label{exa:djl} The equality $\sR(X) = V(X)$ is known to fail
in case of \cite[Example 1.11]{DJL17} given by the surface $X \coloneqq C_1\times C_2/\sigma_1\times \sigma_2$ where $C_1$ is a curve of genus $>1$, $C_2$ is an Elliptic curve and the diagonal action is induced by $\sigma_1 \acts C_1$ with $E \coloneqq C_1/\sigma_1$ an elliptic curve and $\sigma_2\acts C_2$ is a free and properly discontinuous action. In this case, $\sR(X, f) = 0$
but $V(X, f) = H^0(E, \Omega_E^1)$ for $f\colon X\to A$. 
However the
finite \'etale cover given by $C_1\times (C_2/\sigma_2)=:X'\xrightarrow{\tau} X$ such that for all 1-form $\omega$ on $X'$ coming from $E$,
satisfies $(H^{\bullet}(X', \bbC), \wedge\omega)$ is not exact. 
\item Furthermore, Schreieder's result (see Theorem \ref{thm:schreieder}) implies that conjecturally any form $\omega\notin \sR(X)$
must admit a zero up-to a finite \'etale cover of $X$. In that case, generic vanishing establishes the linearity question.
This is precisey how the linearity question was answered positively up to dimension 3. In the light of this, it is natural to ask if it always happen. We pose 
\end{enumerate}
\end{remark}
\begin{conjecture}[Kotschick--Schreieder conjecture]
Given $\omega \in H^0(X,\Omega_X^1)$, such that for all finite \'etale morphism $\tau\colon  X' \to X$
and for all unitary rank 1 local systems $\bbL$ on $X$
we have $(H^{\bullet}(X', \tau^*\bbL), \wedge\tau^*\omega)$ is exact. Then $Z(\omega) = \emptyset$. 
\end{conjecture}
%An example of Debarre, Jiang and Lahoz 
%\cite[Example 1.11]{DJL17} shows that the
%there exists a bi-elliptic surface $S$ admitting a 1-form 
%$\omega$ for which $(H^*(X, \mathbb{C}), \wedge \omega)$
%is exact, yet $\omega$ admits zeros on $S$. Theorem \ref{thm:linearity} shows that such forms are not under the realm of generic vanishing theory. 


Finally, in any dimension we show the following
\begin{alphtheorem}\label{thm:codim1}
Let $X$ be a smooth complex projective variety. Then $V^1(X)$ is linear.
\end{alphtheorem}
As usual more generally we show a relative version on this theorem (see \S \ref{sec:codim1}). This follows from
a statement of Spurr \cite[Theorem 2]{Sp88}. He showed that whenever a divisor $D\subset Z(\omega)$ for some 1-form $\omega$, then
either $D$ is rigid in the sense that it is negative with respect to a polarisation of $X$ or $\omega$ comes from a curve. We generalise this statement
in the quasi-projective setting
\begin{alphtheorem} \label{thm:spurr}
Let $(X, D)$ be a pair with $X$ a complex smooth projective variety and $D$ a simple normal crossing divisor of $X$. Let $H$ be a very ample divisor on $X$. If $(X, D)$ carries a holomorphic log 1-form $0\not=\omega\in H^0(X, \Omega_X^1(\log D))$  which pullbacks to zero on an effective divisor $E$ with $E^2\cdot H^{n-2}\geq0$, then there is a morphism $f: X-D\to C$ to a smooth quasi-projective curve $C=\bar{C}-B$ (where $\bar{C}$ is the completion of $C$ and $B$ can be empty) with 

(1)  $\omega=f^*\eta$ for some $\eta\in H^0(X, \Omega_{\bar{C}}^1(\log B))$.

(2) $E$ is contained in the fiber of $f$ and $E^2\cdot H^{n-2}=0$.
\end{alphtheorem}

As a corollary we prove the linearity of the set of logarithmic holomorphic 1-forms admitting codimension one zeros.

\begin{corollary}
Let $(X, D)$ be a pair with $X$ a complex smooth projective variety and $D$ a simple normal crossing divisor of $X$. Then the set $$ V^1(X,D):=\{ w \in W(U) \mid \codim_X Z(w) \leq i \}$$ is linear.
\end{corollary}

To show the Corollary we prove that $V^1(X,D)$ consists of the following three types of logarithmic 1-forms: those certain generic vanishing type property, those vanishing along rigid divisors, and those vanishing along some components of the boundary divisor $D$.









In this paper, all complex of sheaves and perverse sheaves are defined with complex
coeffcients. All the varieties are complex quasi-projective varieties. 

\subsection*{Acknowledgements}



\section{Preliminaries}
\subsection{Generic Vanishing and Non-abelian Hodge theory}\label{sec:gv}
Given a holomorphic 1-form $\omega\in H^0(X,\Omega_X^1)$
the kernel of the following de Rham complex
\begin{equation}
\sK^{\bullet}_{\omega} \coloneqq [\sO_X\overset{d+\wedge\omega}{\to} \Omega_X^1 \to \cdots\overset{d+\wedge\omega}{\to}  \Omega_X^{n-1}\overset{d+\wedge\omega}{\to}  \Omega^n_X.]
\label{eq:koszul}
\end{equation}
defines a rank 1 local system $L(\omega)$. This gives in the way of the generic vanishing theory developed by \cite{GL, Ara, Sim} into the 
study of zeros of holomorphic 1-forms. In this section we discuss 
the relevant bits of this vast theory that we will use in various
proofs.
%
%\begin{definition}[Zero scheme of 1-forms]\label{def:zeroscheme}
%For $\omega\in H^0(X, \Omega_X^1)$, the \emph{zero set} $Z(\omega)$ of $\omega$ is the algebraic set of closed point $x$ in $X$, such that $\omega(v)=0$
%for all tangent vectors $v\in T_xX$ at $x$. 
%
%The \emph{zero scheme} of $\omega$ is the closed subscheme $\sZ(\omega)$ defined by the ideal sheaf $\mathcal{I}_{\omega}$ given by the image of the morphism 
%\[\mathcal{T}_X\overset{\langle\omega, \cdot\rangle}{\longrightarrow} \mathcal{O}_X.\] 
%Here $\mathcal{T}_X$ is the tangent sheaf of $X$ and $\langle\omega, \cdot\rangle$ denotes the pairing of tangent field with 
%the 1-form $\omega$.
%\end{definition}

 
The generic vanishing theory 
\cite[Proposition 3.4]{GL} ensures that
if $Z(\omega)\neq \emptyset$ then the sequence
\[\cdots\overset{\wedge\omega}{\to} H^k(X, \Omega^{i-1})
\overset{\wedge\omega}{\to}H^k(X, \Omega_X^{i})
\overset{\wedge\omega}{\to} H^k(X,\Omega_X^{i+1})
\overset{\wedge\omega}{\to}\cdots\]
is not exact for all $k\geq 0$. Putting these together 
by the Hodge decomposition for $H^k(X,\bbC)$ we get
\begin{equation}
(H^{\bullet}(X,\bbC), \wedge\omega)\coloneqq [\ldots\to H^{i-1}(X,\C)\overset{\wedge\omega}{\longrightarrow}H^{i}(X,\C)\overset{\wedge\omega}{\longrightarrow}H^{i+1}(X,\C)\to\ldots]
\label{eq:resonance}
\end{equation}
is not exact whenever $Z(\omega)\neq \emptyset$. This prompts the following


\begin{definition}[Resonant forms]\label{def:resonance}
We call a 1-form $\omega$ \emph{resonant} if the complex 
in Equation (\ref{eq:resonance}) is not exact. The set of all such forms 
is denoted $\sR(X,\bbC)$.

On the other hand for when the the complex 
in Equation (\ref{eq:resonance}) is exact and\footnote{Note that
with this assumption it is not always the case that $Z(\omega) = \emptyset$. See Example \ref{rem:linearity}.\ref{exa:djl}}
then we call such form non-resonant. 

%Moreover a 1-forms $\omega\in V(X)$ is called \emph{universally nonresonant} if the complex
%\newline $(H^{\bullet}(X',\bbC), \wedge\tau^*\omega)$
%%\[\ldots\to H^{i-1}(X',\C)\overset{\wedge\tau^*\omega}{\longrightarrow}H^{i}(X',\C)\overset{\wedge\tau^*\omega}{\longrightarrow}H^{i+1}(X',\C)\to\ldots\]
 %is exact for any \'etale over $\tau\colon X'\to X$. 

We will refer to the sequence $(H^{\bullet}(X,\bbC), \wedge\omega)$ above as the \emph{resonance sequence}.
%Also, we call a holomorphic 1-form to be a resonant 1-form if it is not a nonresonant 1-form and has zeros.
\end{definition}




More generally for any local system we consider a similar situation. 
To make our notations precise recall 
\begin{definition}[Character variety]
For a smooth projective variety $X$, we define the \textsl{character variety} to be \[\Char(X) \coloneqq Hom(H_1(X, \bbZ)/\torsion, \bbC^{\star})
\footnote{This is connected component around the trivial character of the Betti moduli $M_B(X) \coloneqq \Hom(\pi_1(X), \bbC^{\star})$ via the short exact sequence \[0\to \frac{H^1(X, \bbC)}{H^1(X, \bbZ(1))}\to M_B(X)\to H^2(X, \bbZ(1))_{\torsion}\to 0.\] Unlike us, in literature this $M_B(X)$ is often called the character variety $\Char(X)$.} .\]
This is isomorphic to
$(\bbC^{\star})^{2q}$ where $q = \dim \Alb(X)$.
\end{definition}
By the Riemann--Hilbert correspondence any such representation $\rho\in \Char(X)$ 
uniquely gives a local system of rank 1. 
Recall also the non-abelian Hodge
correspondence
\[\Char(X) \xrightarrow{\Psi} \Pic^{0}(X)\times H^0(X,\Omega_X^1).\]
mapping a local system $\bbC_{\rho}$ to the line bundle $\sL_{\rho}\simeq \bbC_{\rho}\otimes_{\bbC}\sO_X$
along with the 1-form $\omega_{\rho}$ given by the inverse image of $\bbL$ under
the exponential map
\begin{equation}
H^1(X,\bbC) \xrightarrow{\exp} \Char(X).
\label{eq:exponential}
\end{equation}
More precisely, $\omega_{\rho} = $ (0,1)-piece of $|\log(\rho)|$. Conversely, a pair $(\sL, \omega)$ 
is mapped to the kernel of $\sL\xrightarrow{\partial_{\sL}+\omega} \Omega_X^1\otimes \sL$ under this correspondence.
This induces an isomorphism of topological groups. 

\begin{remark}[Unitary local systems]
Note that a unitary local system given by
\[\eta\in\Char(X)^u \coloneqq \Hom(H_1(X,\bbZ)/\torsion , U(1))\]
corresponds to $(\bbC_{\eta}\otimes_{\bbC}\sO_X, 0)$
under the above non-abelian Hodge correspondence \cite{Sim91}. Thus
we get an embedding 
\[\Pic^0(X) \to \Char(X).\]
This is not a complex submanifold.
\end{remark}


\subsubsection{Hodge decomposition for trivial line bundles with Higgs fields}
Given $(\sL, \omega)\in \Pic^0(X)\times H^0(X,\Omega_X^1)$, following \cite{Ara92} we define the $(p,q)$-th cohomology groups as
\[H^{p,q}(X, (\sL, \omega)) \coloneqq 
\frac{\ker\left(\omega\colon H^q(X, \Omega_X^p\otimes \sL)
\to H^q(X, \Omega_X^{p+1}\otimes\sL)\right)}{\im\left(\omega\colon
H^q(X,\Omega_X^{p-1}\otimes\sL)\to H^q(X,\Omega_X^{p}\otimes\sL)\right)}\]
We have the following 
\begin{theorem}[{\cite[Theorem 3]{Ara92}}]
Let $\rho\in \Char(X)$ be the character corresponding to 
$(\sL,\omega)$ then we have
\[H^k(X, \bbC_{\rho}) \simeq \bigoplus_{p+q =k}H^{p,q}(X,\sL,\omega)\]
\label{thm:genhodgedecomp}
\end{theorem}

Note that when $\omega = 0$, i.e.\ $\bbC_{\rho}$ is unitary
this recovers the usual Hodge decomposition for 
unitary local systems corresponding to $\rho \in Hom(\pi_1(X)/\torsion, U(1))$
\[H^p(X, \bbC_{\rho}) \simeq H^p(X, \Omega_X^q\otimes\sL).\]

We now defined
\begin{definition}[generalised resonant 1-forms]
Given a local system $\bbC_{\rho}$ associated to a unitary character $\rho$, we define
the \textsl{set of resonant forms associated to $\bbC_{\rho}$} to be
\[\sR^k(X, \bbC_{\rho}) \coloneqq
\{\omega\in H^0(X, \Omega_X^1)| H^k(H^{\bullet}(X, \bbC_{\rho}), \wedge\omega) \neq 0 .\}\]
We denote as usual 
$\sR(X, \bbC_{\rho}) \coloneqq \bigcup_k \sR^k(X, \bbC_{\rho})$.
\end{definition}

\begin{remark}
From our discussion above it immediately follows that
if $\sL = \Psi(\rho)$ then
\[\sR^k(X, \bbC_{\rho}) =\bigcup_{p+q = k} \{\omega| H^{p,q}(X, \sL,\omega) \neq 0\}.\]
\end{remark}


\subsubsection{Tangent cones and cohomology jump loci}
\label{sub:tc} Another way to understand the resonant 1-forms is via the tangent cone of the cohomology jump loci defined
as follows.
\begin{definition}
Given a perverse sheaf $\sP$ on $X$, define
\[\sV^i(X,\sP) \coloneqq \{\rho\in\Char(X)|
H^i(X,\sP\otimes \bbC_{\rho})\neq 0\}.\]
\end{definition}
Recall from (\ref{eq:exponential}) that given $\rho\in \Char(X)$, $\omega_{\rho} = $ (0,1)-piece of $\log|\rho|\in H^1(X, \bbC)$. 
%Here we state the relevant results in the relative setting when $X$ admits a map $f\colon X\to A$ 
%to an abelian variety. The main reference for this part is \cite{sch}. We first set 
%\[\sR(f) \coloneqq \{\omega\in H^0(A,\Omega_A^1)| (H^{\bullet}(X, L(f^*\omega)), \wedge f^*\omega) \text{ is exact }\]
%Then $(\sL, \wedge\omega)\in \sM_{Higgs}^0$
%the identity component of the Higgs moduli space. $\sM_{Higgs} \simeq \Pic^{\tau}\times H^0(X,\Omega_X^1)$.
%This is isomorphic to $\Char(X) = Hom(\pi_1(X), \bbC^*)$ as a complex manifold under the map
%\[\Phi\colon \sM_{Higgs} \to \Char(X)\text{ define by } (\sL, \omega) \mapsto L(\omega).\]
By the generic vanishing theory \cite[Theorem 3]{Ara92} we have 
\begin{equation}
\Psi(\sV^k(X,\bbC))
=\bigcup_{p+q = k}\{(\sL,\omega)| H^{p,q}(X,(\sL,\omega)) \neq 0
 \text{ for some } p,q\in\bbZ_{\geq 0}\}.
\label{eq:arapura}
\end{equation}

Therefore we have the following
\begin{proposition} 
Given a character $\rho\in \sV^k(X,\bbC)$,
we identify $TC_{\rho}(\sV^k(X,\bbC))$ as a subspace in the
 Lie algebra
$H^1(X,\bbC)$. Then
\[\bigcup_{\rho\in\sV^k(X,\bbC)}TC_{\rho}(\sV^k(X,\bbC))^{(1,0)} = \bigcup_{\eta\in \Char(X)^{u}}\sR^k(X, \bbC_\eta).\]
Here $TC_{\rho}(\sV^k(X,\bbC))^{(1,0)}$ denotes the holomorphic
piece of $TC_{\rho}(\sV^k(X,\bbC))$.
\label{prop:equivalence}
\end{proposition}
\begin{proof}
The result follows essentially from (\ref{eq:arapura})
above. Indeed, by Theorem \ref{thm:genhodgedecomp} $\omega \in \sR^k(X, \bbC_{\eta})$
implies $H^k(X, \Psi^{-1}(\sL, \omega)) \neq 0$
where $\sL\coloneqq \bbC_{\eta}\otimes_{\bbC}\sO_X$.
Similarly, if $\rho\in \sV^k(X, \bbC)$ i.e.\ $H^k(X, \bbC_{\rho})
\neq 0$ then again by \ref{thm:genhodgedecomp} letting
$(\sL, \omega) = \Psi(\rho)$,
we have $\omega\in \sR^k(X, \bbC_{\Psi^{-1}(\sL, 0)})$.

\end{proof}


\subsubsection{Linearity}
Simpson \cite[p.\ 365]{Sim93a} introduced the notion of linearity of subsets of 
$\Pic^0(X)\times H^0(X,\Omega_X^1)$. Via the albanese
map $a\colon X\to \Alb_X$ we may and do identify
$\Pic^0(X)\times H^0(X,\Omega_X^1)$ with
$A^{\natural} \coloneqq \Pic^0(A)\times H^0(A,\Omega_A^1)$.
\begin{definition}[Linearity]\label{def:linhiggs}
A subset $Z\subset A^{\natural}$ is said to be \textsl{linear}
or \textsl{translates of triple tori}
if there exists finitely many morphisms of abelian varities
$p_i\colon A\to B_i$ and pairs $(\sL_i,\omega_i)\in B_i^{\natural}$
such that $Z= (p_i^*\sL_i,p_i^*\omega_i)\otimes \im(B_i^{\natural}
\to A^{\natural})$.

Similarly a subset $\sZ\subset \Char(A)$ is said to be linear if here exists finitely many morphisms of abelian varities
$p_i\colon A\to B_i$ and character $\rho_i\in \Char(B_i)$
such that $\sZ= p_i^*\rho_i\cdot\im(\Char(B_i)\to \Char(A))$.
\end{definition}




It is a result of Simpson \cite[Theorem 3.1]{Sim93b} that a closed algebraic subset $\sZ \subset Char(A)$ is linear if and only if its image $Z\subset A^{\natural}$ remains algebraic. Furthermore we have 
\begin{theorem}
The cohomology jump loci $\sV^k(X, \bbC)$ are linear and $\rho_i$'s are torsion characters. In particular, $\sR(X)\coloneqq \bigcup_{\eta}\sR(X,\eta)\subset H^0(X,\Omega_X^1)$ is linear, i.e.\ a finite union of linear subspaces.
\label{thm:}
\end{theorem} 

Finally we need a similar linearity statement
for perverse sheaves on abelian varieties. More on this is discussed in the subsequent section \S \ref{sec:perverse}.
We collect together further properties of the
cohomology jump loci on an abelian variety $A$ of dimension $q$.
We put appropriate references wherever needed. \sorry{add reference}

\begin{theorem}[Properties of CJL on $A$]
Let $\sP$ be a perverse sheave on an abelian variety $A$ with $\dim A = q$.
The
cohomology jump loci of $\sP$ satisfy the following
\begin{enumerate}
	\item Propagation property:
\[\sV^q(A, \sP) \subseteq \cdots\subseteq \sV^1(A, \sP) 
\subseteq \sV^0(A, \sP) \supseteq \sV^1(A, \sP) \supset\cdots\supset \sV^q(A, \sP).\]
Furthermore, $\sV^i(A, \sP) = \emptyset$, if $i \in [-q, q]$.
\item Codimension lower bound: for any $i \geq 0$,
$\codim \sV^i(A, \sP) \geq 2i$.
\item Generic vanishing: there exists a non-empty Zariski open subset $U \subset \Char(A)$
such that, for any $\rho\in U$, $H^i(A, \sP 
\otimes \sL_{\rho}) = 0$ for all $i \neq 0$.
\item Signed Euler characteristic property:
$\chi(A, P) \geq 0$.
Moreover, the equality holds if and only if $\sV^0(A, \sP) \neq \Char(A)$.
\item Structural property: $\sV^i(A, \sP)$ is a finite union of linear subvarieties of $Char(A)$ for any $i$.
\end{enumerate}
\label{thm:gvperverse}
\end{theorem} 




\subsection{Perverse sheaves, $D$-modules and Decomposition theorem}\label{sec:perverse}
Perverse sheaves on a smooth projective variety $X$ are, roughly speaking, a generalisation of local systems. 
Under the Riemann--Hilbert correspondence they underlie regular, holonomic $D$-modules and the two categories are equivalent. 
We refer the readers to \cite[Chapter 6,7]{HTT} for a 
comprehensive background on this topic. 
The corresponding $D$-module $\sM$, being regular and holonomic admits an increasing
 filtration $F_{\bullet}\sM$ with respect to the filtration on $D_X$
given by the order of differential operators. Furthermore, the pieces $F_{\bullet}\sM$'s are coherent sheaves and the corresponding
graded module $gr^F_{\bullet}\sM$ is coherent as a module over the graded abelian algebra $gr^F_{\bullet}D_X$. We identify the latter
 with the pushforward of the
structure sheaf $\sO_{T^*X}$ of the cotangent sheaf, via the bundle map $\pi\colon T^*X\to X$. We define
\begin{definition}[Conormal Sheaf]
Given $Z\subset X$ a subvariety of a smooth projective variety $X$, we define the conormal sheaf 
\[T^*ZX\coloneqq \overline{T^*_{Z^{\reg}}X} \coloneqq \overline{\{(z,\omega)\in T^*X| z\in Z^{\reg} \text{ and } \omega|_{Z}(T_zZ) = 0\}}\]

\end{definition}
By results of Kashiwara and Gabber, the holonomicity condition ensures that (see e.g.\ \cite[Theorem E.3.6]{HTT}) the support of $gr^F_{\bullet}\sM$ is a finite union of
conical Lagrangian subvarieties given by closure of conormal sheaves along various subvarieties of $X$.
 
\begin{definition}[Singular Support]
Given a perverse sheaf $\sP$ and its corresponding regular holonomic $D$-module together with a good filtration $(\sM, F_{\bullet})$ we define the singular support of $\sM$ to be the support of the coherent $\sO_{T^*X}$-module $\widetilde{gr_{\bullet}^F\sM}$. The module structure is
induced via the bundle map $\pi\colon T^*X\to X$ and is
\[\widetilde{gr_{\bullet}^F\sM} \coloneqq gr_{\bullet}^F\sM\otimes_{\pi^{-1}gr_{\bullet}^FD_X}\sO_{T^*X}.\]
We write
\[\SS(\sP) (\text{ or, } \SS(\sM)) \coloneqq \bigcup_{Z\subset X}T^*_ZX\]

This notion helps us distinguish between local systems from more complicated perverse sheaves. More precisely, it detects the singularity of a $D$-module (or, equivalently constructible sheaves) effectively. We have
\begin{theorem}[{\cite[Proposition 2.2.5]{HTT}}]
A perverse sheaf $\sP$ is a local system if and only if $\SS(\sP) = T^*_XX$ (the zero section of $T^*X$).
\label{thm:support}
\end{theorem}

\end{definition}
\begin{definition}[Characteristic Cycle]
The characteristic cycle associalted to a perverse sheaf $\sP$ is defined to be the support of $\widetilde{gr_{\bullet}^F\sM}$ with
each irreducible components 
taken with multiplicities given by the length of the module localised at generic points of the components. we write
\[\CC(\sP) \coloneqq \sum_i n_ZT^*_ZX\]

\end{definition}
The euler characteristic of perverse sheaves are determined by intersection theory of its support much like the Grothendiek--Riemann--Roch theorem for coherent sheaves. This was shown by Kashiwara \cite{Kas85}.
\begin{theorem}[Kashiwara's index]

\label{thm:index}
\end{theorem}
Simple perverse sheaves are minimal and irreducible objects among perverse sheaves. They correspond to the minimal extension of a local system $\bbL_U$ supported on an open subset $U\subseteq Z$ of some subvariety $Z\subseteq X$, and are denoted $IC_Z(\bbL)$.  A perverse sheaf admits Jordan--Holder type
filtration with simple perverse sheaves as quotients. However, the ones arising geometrically are in fact semi-simple. This fact is known as the decomposition theorem and is the topic of our next short overview. We refer the reader to
\cite[Section 4.5]{HTT08} for more general discussion.
\begin{theorem}[Decomposition theorem]
Given a morphism $f\colon X\to A$,

\label{thm:}
\end{theorem}


Given $f\colon X\to A$ a surjective morphism of a smooth projective variety to an abelian
variety, via Kashiwara's estimate the singular support of $\bbR f_*\bbC$ produces a breeding ground for 1-forms with zeros. This phenomenon was exploited in
\cite{PS14}
in showing that all 1-forms admit zeros on smooth projective
varieties of general type. Here we recall the estimate in our
current framework. 
\begin{theorem}[Kashiwara's estimate on the behaviour of singular support]
Given $f\colon X\to A$, consider the following commutative diagram.
\[
\begin{tikzcd}
T^*X \ar[dr]& f^*T^*A \ar[d]\ar[r, "df"]\ar[l, "f\times\id"'] & T^*A\ar[d]\\
&X\ar[r, "f"] &A
\end{tikzcd}
\]
Then 
\[\SS(\bbR f_* \bbC[n]) \subseteq (f\times \id)(df^{-1}(0_X))\]
where $0_X$ denotes the zero section $T^*_XX$ of $T^*X$.

\label{thm:}
\end{theorem}












\section{Generic vanishing and decomposition theorem}

\subsection{Generic vanishing on abelian varieties}
Before we establish the curious connection between the generic vanishing theory and the decomposition theorem, we need to understand
the generic vanishing theory for perverse sheaves on abelian variety. This subsection is devoted to
 Theorem \ref{thm:perverse} establishing a connection between the cohomology jump loci of simple perverse sheaves and their singular support.
%
%\begin{proof}[Proof of Theorem \ref{thm:linearity}]
%By the decomposition Theorem \ref{thm:decomp} applied to $f\colon X\to A$
%we have, 
%\[Rf_*\bbC\simeq \bigoplus_{\alpha,i} \sP_{\alpha,i}[-i].\]
%Here $\sP_{\alpha, i}$ are simple perverse sheaves supported on various strata of the morphism $f$. 
%Note that
%\[\{\rho\in \Char(A)| H^k(X, f^*\bbC_{\rho})\neq 0 \text{ for some } k\} = \bigcup_{\alpha, i, k}\sV^k(A, \sP_{\alpha, i}).\]
%Applying Theorem \ref{thm:perverse} to each 
%of these $\sP_{\alpha,i}$ we obtain the result.
%\end{proof}


\begin{proof}[Proof of Theorem \ref{thm:pervese}]
By the propagation property of the cohomology jump loci
$\sV^i(A, \sP)$ as in Theorem \ref{thm:cjl}(1) we
only need to consider $\sV^0(A,\sP)$. We split the argument in 
two cases:
\noindent \textbf{Case I: }$\chi(A, \sP)>0$. 
For a general character $\rho$, $H^i(A, sP\otimes \bbC_{\rho})= 0$ for $i\neq 0$ by Theorem \ref{thm:cjl}(3). Hence for a general
$\rho$,
in this case we must have $H^0(A, \sP\otimes \bbC_{\rho})\neq 0$. Hence, 
\[TC_{\rho}\sV^0(A, \sP) = H^1(X,\bbC)\]
On the other hand we have
the Kashiwara index theorem \ref{thm:indextheorem}
that states
\[\chi(A,P) = \langle \CC(\sP), T^*_AA\rangle\]
where $CC(P)$ is the characteristic cycle of $\sP$ as
in Definition \ref{def:cc}.
Note that if $Z\subset A$ is fibered by a subabelian variety $B\subset A$, then $T^*_ZA = T^*_BA$. But $<T^*_BA, T^*_AA> = \chi(B, \bbC) = 0$ (see e.g.\ \cite[p.\ 124]{Dim}).  Therefore,
the $\SS(\sP)$ must contain a subvariety $Z\subset A$ such that
$Z$ is not fibered by tori. 

We conclude by Proposition \ref{van-nonsimple} that when $Z$ is not fibered by tori $\pi(T^*_ZA) = H^0(A,\Omega_A^1)$. Hence
we obtain the desired equality.

\noindent \textbf{Case II: } $\chi(A, \sP)=0$. 
By a result of Weissauer \cite[Theorem 2]{Wei}
we know that there exists an abelian variety $B$ and a surjective morphism $\varphi\colon A\to B$, a simple perverse sheaf $\sP_B$ on $B$, and a local system $\bbL$ on $A$
such that 
\[\sP\otimes \bbL\simeq \varphi^*\sP_B.\]
Since $\sP$ and $\sP\otimes \bbL$ have the same singular support
and upto translation by $\bbL$ the same cohomology jump loci, we 
may and do replace $\sP$ by $\sP\otimes \bbL$.

On the other hand, by a result of
Franecki and Kapranov
\cite[Corollary 1.4]{FK} the formula of
characterstic cucle, i.e\ \[\CC(\sP) = \sum_Z n_Z Z\]
satisfies $n_Z\geq 0$. Hence from Kashiwara's index theorem 
\ref{thm:indextheorem}, one can deduce that all $Z\subset \SS(\sP)$ must be fibered by tori. Similarly, there must be 
$Z$, with $Z' = \varphi(Z)$ not fibered by tori.
Therefore, from the previous case we have 
\[(TC_{\rho}\sV^0(B, \sP_B))^{1,0} = \pi(\SS(\sP_B).\]
We then conclude 
that $\varphi^*(\Char(B))\subset \sV^0(A, \sP)$ from observing that
\[H^0(B, \sP_B\otimes \bbC_{\rho'}) \overset{\oplus}{\into}
H^0(A, \sP\otimes \varphi^*\bbC_{\rho'}).\]
To see the converse note that 
for any $\rho \in \sV^0(A, \sP)$ we must 
have $\bbC_{\rho}|_{\operatorname{Fibre}(\varphi)} = \bbC$.
Indeed, $H^0(A, \sP\otimes \bbC_{\rho}) = H^0(B, \sP_B\otimes \varphi_*\bbC_{\rho})$.
Hence $\varphi_*\bbC_{\rho} = \bbC_{\rho'}$ and $\bbC_{\rho} = 
\varphi^*\bbC_{\rho'}$.

\end{proof}

\subsection{Linearity and comparison}
Linearity of cohomology jump loci is a well-known fact in the generic vanishing theory. We start by showing the linearity of the input
from decomposition theorem in the statement of Theorem \ref{thm:linearity}. 
\begin{proposition}\label{van-nonsimple}
Let $A$ be an abelian variety and $X$ be a proper subvariety of $A$. Then the following are equivalent
\begin{enumerate}
	\item $X$ is not fibred by tori or of dimension 0. 
	\item Any holomorphic 1-form $\omega\in H^0(A, \Omega_A^1)$ restricted to $X^{\reg}$, i.e.\ $\omega|_{X_{\textnormal{reg}}}$ admits zeros on the smooth locus $X_{\textnormal{reg}}$.
\end{enumerate}
\end{proposition}
The result is well-known to the expert, especially when $Z$ is smooth. 
\begin{proof}\label{proof:van-nonsimple} (2)$\Rightarrow$ (1): Suppose $X$ is fibred by tori and $\dim X>0$, i.e.\ there exists a subabelian variety $B\subset A$ such that the fibres of the composition 

$X\into A \onto A/B$ are $B$. Taking an \'etale cover $\tau: A'\to A$, we assume $A'=B\times C$ and $X':=\tau^{-1}(X)=B\times Y\subset A'$. Then the 1-forms coming from $B$ do not vanish on smooth locus of $X'$, hence also do not on the smooth locus of $X$. 

(1)$\Rightarrow$(2): Denote $d=\dim X$, $g=\dim A$. If $d=0$ it is trivial, so we assume $d>0$. Denoting by $N$ the
vector bundle on $X^{\reg}$ given by $N_{X/A}|_{X^{\reg}}$ where $N_{X/A}$ is the normal sheaf of associated to the immersion
 \cite[\href{https://stacks.math.columbia.edu/tag/01R1}{Tag 01R1}]{stacks-project}. Associated to \[T_A|_{X^{\reg}}\onto N,\]
we obtain the following chain of maps
\[\varphi\coloneqq (\bbP(N) \to 	X^{\reg}\times \bbP(T_0{A}) \to \bbP(T_0A)=\bbP^{g-1})\]
where for a vector bundle $E$ on $X^{\reg}$ we use the notation $\bbP(E) \coloneqq \Proj_{X^{\reg}})(\Sym^d(E^{\vee}))$.
Denote by $p\colon \bbP(N)\to X^{\reg}$ the bundle map. Given
a point $s\in \bbP^g$, by construction we may associate a hyperplane $H_s\subset T_{A,0}$. Then
\begin{equation}
p\colon \varphi^{-1}(s) \overset{p}{\xrightarrow{\sim}}  \{x\in X^{\reg}| T_xX^{\reg} \subset H_s\}.
\label{eq:van-nonsimple}
\end{equation}
Denote by $S\coloneqq \varphi(\bbP(N))$ the image. If $\dim S<g$, for all $s\in S$, a general
 fibre $Z_s \coloneqq \varphi^{-1}(s)$
must have dimension $g - 1 - \dim S$. Let $A_0$ denote the abelian variety spanned by $Z_s$, then $\dim A_0 > g- 1 - \dim S$. Indeed,
our assumption implies that $Z_s$ cannot itself be an abelian variety. Since for a general $s\in S$, $H_s \supset T_xA_0$
for all $x\in p(\varphi^{-1}(s))$ by \eqref{eq:van-nonsimple}. Normalising $x$ to $0$, we obtain that $T_0A_0$
is cut out by $\dim S$ many hyperplanes in $T_0A$. But $\dim T_0A_0 \gneq g -1 - \dim S$. Hence $\varphi$ is quasi-finite
dominant morphism which shows the claim above.
\end{proof}




We are now ready to prove Theorem \ref{thm:linearity}. However we prove a much more general version for semi-simple local system
of any rank and in the relative setting. First, we recall the notion of general cohomology jump loci.
\begin{definition}{Cohomology Jump Loci}
Let $\bbE$ be complex local system of rank $r$ on a smooth projective variety $X$. We define, the \textsl{$k$-th cohomology jump loci}
associated to $\bbE$ as
\[\sV^k(X, \bbE) \coloneqq \{\rho\in \Char(X)| H^k(X, \bbE\otimes_{\bbC}\bbC_{\rho})\neq 0\}.\]
Similarly, given a proper morphism $f\colon X\to A$ to an abelian variety, we define 
\[\sV^k(X, \bbE, f)\coloneqq \{\rho\in \Char(A)| H^k(X, \bbE\otimes_{\bbC}f^*\bbC_{\rho})\neq 0\}.\]
Note that $\sV^k(X, \bbE) = \sV^k(X,\bbE, a)$ where $a\colon X\to \Alb_X$ is the albanese morphism.
\end{definition}

Under the non-abelian Hodge correspondence, we can recover the $(1,0)$ part of the tangent cone of these loci. We define
\[\sR(X, \bbE, f) \coloneqq \bigcup_k\pi\circ\Psi(\sV^k(X,\bbE, f)).\]

We have

\bt
Let $f\colon X\to A$ be a proper morphism to an abelian variety. Let $\bbE$ be a semisimple local system on $X$ then
\[\sR(X,\bbE) = \pi(SS(Rf_*\bbE))\]
\et

\begin{proof}
The proof is a direct consequence of the comparison Theorem \ref{thm:perverse}. Indeed,
By the decomposition theorem, we have
\[\bbR f_*\bbE = \bigoplus\sP_{i,\alpha}[-i]\]
for various simple perverse sheaves $\sP_{i,\alpha}$ on $A$. For each of these we apply Theorem \ref{thm:perverse}
to conclude that
\[\bigcup_{\rho}(TC_{\rho}\sV^k(A, \sP_{i,\alpha}))^{(1,0)} = \pi(\SS(\sP_{i,\alpha})).\]
Putting these together we obtain
\[\bigcup_{\rho}(TC_{\rho}\sV^k(A, \bbR f_*\bbE))^{(1,0)} = \pi(\SS(\bbR f_*\bbE)).\]
But, for any character $\rho\in \Char(A)$, we have
$H^k(A, \bbR f_*\bbE\otimes \bbC_{\rho}) = H^k(X, \bbE\otimes f^*\bbC_{\rho})$. Hence the result.
\end{proof}







\section{Some Geometric consequences}
In this section we prove Theorem \ref{thm:smooth} and \ref{thm:cxsmooth}, thereby establishing a geometric consequence of our theorem. 
Note that by a result of Tischler \cite[Theorem 1]{Tis70} and Sackstder \cite[p.\ 96]{Sac65} we know that for a real manifold $X$
a $\scrC^{\infty}$-fibre bundle structure over $S^1$ is equivalent to having a real $\scrC^{\infty}$ closed 1-form $\alpha$ without zeros. 
The following generalisation of Schreider's \cite[Theorem 1.2]{Sch19} relates this with the complex geometry when $X$ admits
the structure of a smooth projective variety. We give a slightly different proof than the one in \cite{Sch19}.   

\begin{theorem}\label{thm:schreieder}
Given a $f\colon X\to A$ be from a smooth projective variety $X$ to an abelian variety $A$ such that
both $X$ and $A$ admit $\scrC^{\infty}$-fibre bundle structure over $S^1$ with the commutative diagram
\[\begin{tikzcd}
	X\ar[dr, "p_X"']\ar[d, "f"]& \\
	A\ar[r, "p_A"]&S^1
\end{tikzcd}.\]  
Then there exists a global holomorphic 1-form $\omega\in H^0(A,\Omega_A^1)$ such that for any finite \'etale cover $\tau\colon X' \to X$ and
any unitary local system $\bbL\in \Char(A)$, we have $H^{\bullet}(X, \tau^*f^*\bbL), \wedge\tau^*f^*\omega)$ is exact. 
\label{thm:schreieder}
\end{theorem} 

\begin{proof}
By \cite[Proposition 5.4]{LW18}, given $\rho\in \Char(S^1)$, the local system $\bbC_{\rho}$ associated to $p_X^*(\rho)$ admits
a non-trivial cohomology if and only if $\rho$ is an eigenvalue of the monodromy operation on $H^{\bullet}(F, \bbC)$. Hence, 
we have the strict inclusion
\[\sV(X, f) \coloneqq \{\rho\in\Char(A)| H^i(X, f^*\bbC_{\rho}) \neq 0 \text{ for some }i\}\subsetneq \Char(A).\]
Since the tangent cones $TC_{\rho}\sV(X, f)\subsetneq H^1(A, \bbC)$ is a sub-Hodge structure, we obtain the same strict inclusion
for the $(1,0)$-piece of the tangent cone into $H^0(A, \Omega_A^1)$. Consider an $\omega$ in the complement on this inclusion. We now use the non-abelian Hodge correspondednce (Proposition
\ref{prop:equivalence}) to conclude that $\omega\notin \sR(X, \bbC_{f^*\rho})$ for all unitary local systems $\rho\in \Char(A)$. Then
by Hodge decomposition (Theorem \ref{thm:arapura}) we obtain the result.

\end{proof}
We should mention that when $A = \Alb_X$, one can deduce this immediately from Schreieder's original result as follows. By
the generic vanishing theory we know $\sV(X)$ is a finite union of torsion translate of subtori in $\Char(A)$. Therefore, 
all cohomology jump loci pass through $\bbC_X$ after an \'etale cyclic cover of $\Alb_X$.

\begin{proof}[Proof of Theorem \ref{thm:smooth}]
If there exists a simple perverse sheaf $\sP$ in the decomposition of $\bbR^i f_*\bbC[n]$ such that $\sP$ is not a local system,
we must have $Z\in\SS(\sP)$ by Theorem \ref{thm:support}. Since $A$ is simple $\pi(T^*_ZA) = H^0(A, \Omega_A^1)$. 
But Theorem \ref{thm:schreieder} above ensures that the tangent cone to $\sV(X, f)$ cannot be all of $H^1(A, \bbC)$.
This is a contradiction to our comparison Theorem \ref{thm:linearity}.

\end{proof}


Assuming the existence of a holomorphic 1-form.





\section{Linearity of holomorphic 1-forms with codimension one zeros}

In this section, we consider (logarithmic) holomorphic 1-forms with codimension one zeros. We will prove that $V^1(X)$ is linear and give a more finer analysis on $V^1(X)$. 

\subsection{Projective case}

%We give some preliminaries, which is used in studying (logarithmic) holomorphic 1-forms with codimension one zeros, especially the linearity and finer structure of $V^1(X)$. 

First we recall the definition of cohomology jump loci. Let $X$ be a connected finite CW-complex with $\pi_{1}(X)=G$. Then we have the Betti moduli of rank one local system $\textnormal{M}_B(X):= \Hom(G,\C^{\ast})$ is a commutative affine algebraic group. Each character $\rho \in \textnormal{M}_B(X)$ defines a rank one local system on $X$, denoted by $L_{\rho}$. The tangent space $T_1\textnormal{M}_B(X)$ of $\textnormal{M}_B(X)$ at trivial local system 1 is canonically isomorphic to $H^1(X, \C)$.
The cohomology jump loci of $X$ are defined as follows:
$$\mathcal{S}^i(X)=\lbrace \rho\in \textnormal{M}_B(X) \mid  H^{i}(X, L_{\rho})\neq 0 \rbrace,$$ 
which by the structure theorem are finite union of torsion translated subtori of $\textnormal{M}_B(X)$.

To study $V^1(X)$ we introduce a homotopy invariant of $X$, denoted by $T^1(X)${\color{red} (The name TBA, translated tangent cone?)}, which is contained in $H^1(X, \C)$. The following definition of $T^1(X)$ is based on Arapura's work. Roughly speaking, Arapura \cite{Ara97} showed that every positive dimension component of $\mathcal{S}^1(X)$ can be induced by a so called strict orbifold map. A morphism $f:X\to C$ from $X$ to a smooth projective curve $C$ is called a  {\it strict orbifold map} if $f $ is surjective, has connected fibers and one of the following condition holds:
\begin{itemize}
\item $g(C)\geq 2$
\item $g(C)=1$ and $f$ has at least one multiple fiber.
\end{itemize}
A strict orbifold map always induces an injective map: $f^*: H^1(C, \C) \to H^1(X,\C)$. In fact, there are only at most finitely many equivalent strict orbifold maps. Define $$T^1(X):= \bigcup_f \im f^*,$$
where the union runs over all possible strict orbifold maps for $X$. By another equivalent definition of Budur-Wang-Yoon \cite{BWY16} , $T^1(X)$ is a linear subvariety, which is the union of torsion translation of affine tangent space of each component of degree 1 cohomology jump loci $\mathcal{S}^1(X)$. We denote the holomorphic part of the tangent cone $T^1(X)$ to be $T^1_h(X)$, i.e., $T_h^1(X):=T^1(X)\cap H^0(X, \Omega_X^1)$, which is a linear subvariety.

We will see that except for $T_h^1(X)$, another part of $V^1(X)$ is coming from negative divisors.
\begin{definition}
Let $X$ be a smooth projective variety of dimension $n$, with a fixed very ample line bundle $H$ and a prime divisor $E$ on $X$. We say that $E$ is $H$-negative if $E^2\cdot H^{n-2}<0$, where $$E^2\cdot H^{n-2}=\int_X c_1^2(E)\wedge c_1(H)^{n-2}.$$ Similarly we can define $H$-trivial and $H$-positive divisors corresponding to the above integration is zero and positive, respectively. We say $E$ a negative (trivial or positive respectively) if it is $H$-negative ($H$-trivial or $H$-positive respectively) for all ample divisor $H$.
\end{definition}  
With the above definition, we define $$V_{H\textnormal{-neg}}(X):=\{\omega\in H^0(X, \Omega_X^1)\ |\ \omega(p)=0\  \textnormal{for any point}\ p\in E, \ \textnormal{with $H$-negative divisor}\  E\}$$  

\begin{lemma}\label{lem:welldef}
Let $E$ be a integral divisor in $X$ with $\dim X=n$. Suppose there exists a holomorphic 1-form $\omega$ such that $\omega(p)=0$ for each $p\in E$. Then the sign of the intersection number $E^2\cdot H^{n-2}$does not depend on the choice of ample divisor $H$.
\end{lemma}

\begin{proof}
By Theorem \cite[Theorem 2]{Sp88}, $E^2\cdot H^{n-2}\leq 0$ for any ample divisor $H$ and when $E^2\cdot H^{n-2}=0$, $E$ is set theoretically contained in a fibre of a strict orbifold map $f: X\to C$ to a smooth projective curve $C$.  Claim that $E$ is the only component of the fiber containing $E$ if $E^2\cdot H^{n-2}=0$ for some ample divisor $H$. In fact, assume there is an effective divisor $E':=f^{-1}(f(E))\backslash E$ such that $aE+bE'$ is linearly equivalent to $F$ with $F$ a general fiber of $f$ and $a$, $b$ are positive integers. Since $f$ has connected fibres, $E\cdot E'\not=0$.  First notice that $(aE+bE')\cdot E'\cdot H^{n-2}=0$. Also, $(aE+bE')^2\cdot H^{n-2}=0$. Hence we get $E^2\cdot H^{n-2}<0$, which is a contradiction. Since set theoretically $E$ is the whole fiber of a morphism $f: X\to C$, $E^2\cdot H^{n-2}=0$ for all ample divisor $H$. 
\end{proof}


With the Lemma \ref{lem:welldef}, we have the following well defined notation$$V_{\textnormal{neg}}(X):=V_{H\textnormal{-neg}}(X)\subset H^0(X, \Omega_X^1),$$ where $H$ is any ample divisor of $X$. With the above notations, we have the following theorem which is based on Spurr \cite[Theorem 2]{Sp88}.

\begin{theorem}\label{Thm:Proj-codim1}
 Let $X$ be a complex smooth projective variety, with the above notations, we have that 
$$V^1(X)= T_h^1(X)\cup V_{\textnormal{neg}}(X).
$$
Every irreducible component of $V^1(X)$ is a linear complex subvector space of $H^0(X, \Omega_X^1)$, i.e., $V^1(X)$ is a linear subvariety. %Moreover, $V_{\textnormal{neg}}(X)$ is a union of  one dimensional complex vector spaces of $H^0(X, \Omega_X^1)$.
\end{theorem}

\begin{remark} (1) The Theorem \ref{Thm:Proj-codim1} complements the result of Green-Lazarsfeld \cite{GL87}: $T_h^1(X) \subset V^1(X)$. In general one should not expect $T_h^1(X) = V^1(X)$. Let $X$ be a complex abelian surface and $Y$ be the blowup of $X$ along a point. Then $T_h^1(X)=T_h^1(Y)$, but  $V^1(X)\subsetneq V^1(Y)$.

(2) In dimension 2, \cite[Theorem 2]{Sp88} also holds for compact K\"ahler surface and even any compact complex surface. For the compact complex surface $X$, it is not clear if $V^1(X)$ has finitely many components, since it is not algebraic in general.

(3) The two parts $T_h^1(X)$ and $V_{\textnormal{neg}}(X)$ may have overlap. For example if $f: S\to C$ is a morphism from a smooth complex projective surface $S$ to a smooth projective curve $C$ with genus $g(C)\geq2$. Take a holomorphic 1-form $\omega\in H^0(C, \Omega_C^1)$ which has a zero $p\in C$. Pick any point $s\in S$ over $p$ and blow up $S$ along $s$ with $(-1)$ curve $E$. Then consider the natural morphism $f': S'\to C$, then $f^*\omega(E)=0$ and hence $f^*\omega\in T_h^1(X)\cap V_{\textnormal{neg}}(X)$.
\end{remark}

\begin{lemma} \label{countable} Let $X$ be a $n$-dimensional smooth complex projective variety with very ample divisor $H$. Then there are at most countablely many $H$-negative divisors in $X$.
\end{lemma}

\begin{proof}
Let $E$ be any $H$-negative divisor. After fixing $n-2$ general hyperplanes $H_1, \ldots, H_{n-2}$ in the linear system $|H|$, we have over the complete intersection surface $S:=H_1\cap H_2\cap\ldots\cap H_{n-2}$, $E\cap S$ is a negative curve in $S$, which has no infinitesimal deformation, i.e., the $H^0(E\cap S, N_{E\cap S/S})=0$, where $N_{E\cap S/S}$ is the normal bundle of $E\cap S$ in $S$. Then they are isolated points in the Hilbert scheme $\textnormal{Hilb}_S^{P(m)}$ with some Hilbert polynomial{\color{red}(Hartshorne deformation theory reference)}. Note also that $\textnormal{Hilb}_X^{P(m)}$ is of finite type and $P(m)$ are polynomials with $\Q$ coefficients. Hence there are at most countably many negative curves on $S$ with above properties. Hence there are at most countably many $H$-negative divisors.
\end{proof}

\begin{proof}[Proof  of theorem \ref{Thm:Proj-codim1}]
{\color{red}(Remember to discuss the case $\dim X=1$ case, after merging.)}With the notation in theorem \ref{Thm:Proj-codim1},  for any holomorphic 1-form $\omega\in V^1(X)$, there is a integral divisor $D\subset X$ such that $\omega(D)=0$. Then by \cite[theorem 2]{Sp88}, we have either $D$ is $H$-negative or $H$-trivial for any chosen ample divisor $H$. If $D$ is $H$-negative for any ample divisor $H$ then $\omega\in V_{\textnormal{neg}}$. Still by \cite[theorem 2]{Sp88}, If $D$ is $H$-trivial for some ample divisor $H$, then $D$ is contained in a fiber of a morphism $f: X\to C$ with $C$ a smooth projective curve of positive genus and $f$ admitting connected fibers, and $\omega=f^*\eta$ for $\eta\in H^0(C, \Omega_C^1)$. Claim that $D$ is the whole fiber. In fact, if we denote $F:=f^{-1}(f(D))$ and assume by contradiction that $F=D\cup D'$ with $D'\not=\emptyset$. Then the intersection number $D'\cdot D\cdot H^{n-2}>0$. Also, note that $F\cdot D\cdot H^{n-2}=0$, thus $D^2\cdot H^{n-2}<0$, which is a contradiction. Since $D$ is a zero set of a form $\omega$ pullback from $C$ and $D$ is the whole fiber, thus (1) $D$ is multiple and genus $g(C)=1$ or (2) $g(C)>1$ and $\eta(f(D))=0$. Hence $f$ is a strict orbifold fibration. Then $\omega\in T_h^1(X)$. Notice that $T_h^1(X)\subset V^1(X)$ by \cite{GL87}. Hence $$V^1(X)= T_h^1(X)\cup V_{\textnormal{neg}}(X).
$$
By lemma \ref{countable}, $V_{\textnormal{neg}}$ is a union of at most countably many linear subspaces in $H^0(X, \Omega_X^1)$ of the following type $$V_E(X):=\{\omega\in H^0(X, \Omega_X^1)\ |\ \omega(p)=0\  \textnormal{for any point}\ p\in E\},$$ where $E$ is a negative divisor. Since $V^1(X)$ is an algebraic set and $T_h^1(X)$ is a linear subvariety, $V^1(X)$ is a linear subvariety.
 \end{proof} 
 
 In Theorem \ref{Thm:Proj-codim1}, we can write holomorphic 1-forms with codimension one zeros s union of two parts: The 1-forms in the {\color{red} translated tangent cone?} and the one vanishes along a negative divisor.  Another way to understand  $V^1(X)$ is one can divide $V^1(X)$ into set of resonant 1-forms $$V^1_r(X):=V^1(X)\cap R_h(X),$$ and set of nonresonant 1-forms with codimension one zeros, denoted to be $V^1_{nr}(X)$, which is locally closed. Notice that $V_r^1(X)$ is a linear subvariety, since $V^1(X)$ is linear by Theorem \ref{Thm:Proj-codim1}  and $R_h(X)$ is linear by Theorem \ref{thm: resonant}. For holomorphic nonresonant 1-forms,  we have the following proposition.
 \begin{proposition}
Let $X$ be a smooth complex projective variety of dimension $n$. There are only finitely many integral divisors $D$ on $X$, such that there is a holomorphic nonresonant 1-form $\omega\in H^0(X, \Omega_X^1)$ with $\omega(D)=0$.
 \end{proposition}
 
 \begin{proof}
For any holomorphic nonresonant 1-form $\omega$ such that the zero set $Z(\omega)$ contains a integral divisor $D\subset X$, by Theorem \ref{thm:stefan} (1), $H^{n-1}(D, {\omega_X}|_D)=0$, where $\omega_X$ is the canonical line bundle of $X$. Hence $H^0(D, N_{D/X})=0$ by Serre duality. Then by the same Hilbert scheme argument as that is the proof of   Lemma \ref{countable}, we have there are at most countably many integral divisors that are contained in zero locus of some holomorphic nonresonant 1-forms.  Notice also since $V^1(X)$ is algebraic and $V^1_{nr}(X)=V^1(X)\backslash V_r^1(X)$ is open in $V^1(X)$, $\overline{V_{nr}^1(X)}$ has finitely many irreducible components. Since $$\overline{V_{nr}^1(X)}=\bigcup_{D}V_D(X),$$ where the union runs over all integral divisors $D$ contained in zero loci of holomorphic nonresonant 1-forms. Hence the proposition holds.
 \end{proof}
 
 \begin{remark}
By the H.-Y. Lin and Schreieder \cite[appendix]{SS19}, the zero set of any universal nonresonant 1-form is of codimension at least two. 
 \end{remark}

 To end this subsection, we consider nonresonant 1-forms on surfaces, which is for its own interest. Let $S$ be smooth complex surface, one can bound genus of curves along which a nonresonant 1-form may vanish.
 
 \begin{proposition}
 Let $S$ be smooth complex projective surface and $C$ be a curve on $S$. Suppose there is a nonresonant holomorphic 1-form $\omega\in H^0(S, \Omega_S^1)$ such that $\omega(C)=0$ Then $C$ is a rational curve or an elliptic curve.
 \end{proposition}
 
 
 \begin{proof}
 Since $C\subset Z(\omega)$, we have $H^1(C, {\omega_S}|_C)=0$ by Theorem \ref{thm:stefan} (1). Consider the normalization $\eta: \widetilde{C}\to C$ and the exact sequence $$0\to \mathcal{O}_C\to \eta_*\mathcal{O}_{\widetilde{C}}\to \delta\to 0, $$ where $\delta$ is the skyscraper sheaf defined as the coker. Then we have the short exact sequence $$0\to {\omega_S}|_C\to \eta_*{{\omega_X}|_{\widetilde{C}}}\to \delta\to 0.$$ Then we get $$\dim H^0(\widetilde{C}, {\omega_X}|_{\widetilde{C}})\leq \dim H^0(\delta);   H^1(C, {\omega_S}|_C)=H^1(\widetilde{C}, {\omega_X}|_{\widetilde{C}})=0.$$ By Riemann-Roch, $$\deg {\omega_X}|_{\widetilde{C}}+1-g(\widetilde{C})\leq \dim H^0(\delta).$$ Notice also, $$p_a(C)=g(\widetilde{C})+\dim H^0(\delta), $$ and by genus formula $$\deg {\omega_X}|_{\widetilde{C}}+C^2=2p_a(C)-2.$$ Then by \cite[Theorem 1]{Sp88}, $C^2\leq 0$. Hence $p_a(C)\leq1$.
 \end{proof}
 

 \subsection{Quasi-projective case}
 

Let $U$ be a smooth complex quasi-projective variety. Let $X$ be a good compactification of $U$, i.e., a smooth compactification of $U$ with boundary $D:=X-U$ a simple normal crossing divisor. Note that the space of logarithmic 1-forms $H^0(X, \Omega_X^1(\log D))$ does not depend on the choice of the good compactification, so we may denote it by $W(U)$. We write $H^0(X, \Omega_X^1(\log D))$ when we want to emphasize the boundary divisor. Similar to projective case,  one can define $$V^i(X,D):=\{ \omega\in H^0(X, \Omega_X^1(\log D)) \mid \codim_X Z(\omega) \leq i \}.$$
where $Z(\omega)$ is the zero locus of $\omega$. Again by Chevalley's upper-semicontinuity theorem, $V^i(X, D)$ are all algebraic sets.

\begin{remark} For a smooth quasi-projective variety $U$, $V^i(X,D)$ clearly depends on the good compactification $(X,D)$. For example, let $X$ be a smooth projective variety admitting a holomorphic 1-form $\omega\in H^0(X, \Omega_X^1)$ with isolated zero set. Let $D\subset X$ be a divisor containing one point $p \in Z(\omega)$, and $Y$ be the blowup of $X$ along $p$ with exceptional divisor $E$. Then $V^1(X, D)\subsetneq V^1(Y, D\cup E)$. To get an invariant of $U$, we define
$$V^i(U)= \bigcap_{(X,D)} V^i(X,D) ,$$
where the intersection runs over all possible good compactification $(X,D)$ of $U$. Clearly, $V^i(U) $ are closed algebraic varieties of $W(U)$.
\end{remark} 

For a good compactfication $(X,D)$ of $U$ where $D=\cup_{j=1}^r D_j$. Set $$V_{\textnormal{bound}}(X, D):= \{\omega\in H^0(X, \Omega_X^1(\log D)) \mid D_j \subset Z(\omega)\  \text{ for some }\  j \}.$$ Similar to projective case, we denote for a subvariety $E$, $$V_E(X, D):=\{\omega\in H^0(X, \Omega_X^1(\log D)) \mid  E \subset Z(\omega) \}$$ and 
 $$V_{\textnormal{neg}}(X, D):= \bigcup_E V_E(X, D),$$ where the union runs over all the negative divisor $E$ in $X$ such that $E$ is not contained in $D$. 
 
Define $$T_h^1(X, D):= \bigcup_f f^*H^0(C, \Omega_C^1(\log B)),$$
where the union runs over all possible surjective morphism $f: X-D\to C$ to a smooth quasi-projective curve $C=\bar{C}-B$ ($B$ can be empty), such that $f$ has  connected fibers, and one of the following condition holds when $C$ is projective:
\begin{itemize}
\item $g(C)\geq 2$
\item $g(C)=1$ and $f$ has at least one multiple fiber.
\end{itemize} 
{\color{red} (Check the correctness on poles and zeros for $g(C)=0,1$)} 
 
\begin{theorem} \label{thm:codim1quasiprojective} Let $(X, D)$ be a pair consists of an $n$-dimensional smooth complex projective variety $X$ and a simple normal crossing divisor $D$ with irreducible components $\{D_j\}_{j=1}^r$. Then 
$$
V^1(X,D)= T^1_h(X, D) \cup V_{\textnormal{neg}}(X, D) \cup V_{\textnormal{bound}}(X, D),
$$
and every irreducible component of $V^1(X,D)$ is a linear subvector space of $H^0(X, \Omega_X^1(\log D))$. 
\end{theorem}

In order to show the above theorem, we first generalize Spurr \cite[Theorem 2]{Sp88} and ideas of the proof to quasi-projective varieties. 

\begin{proposition} \label{Prop:Spurr-no-intersection}
Let $(X, D)$ be a pair with $X$ a smooth complex projective variety and $D$ a simple normal crossing divisor of $X$. Let $H$ be an ample divisor on $X$. Suppose $(X, D)$ carries a nontrivial logarithmic 1-form $\omega\in H^0(X, \Omega_X^1(\log D))$  which restricts to zero on an integral divisor $E$ with $E^2\cdot H^{n-2}\geq0,$ and $E\cap D=\emptyset,$ then there is a surjective morphism $f: X-D\to C$ to a smooth quasi-projective curve $C=\bar{C}-B$ ($B$ can be empty) with 

(1)  $\omega=f^*\eta$ for some $\eta\in H^0(C, \Omega_C^1(\log B))$.

(2) $f$ has connected fibers.

(3) $E$ is contained in the fiber of $f$ and $E^2\cdot H^{n-2}=0$.
\end{proposition}

\begin{proof}
Firstly, we prove the theorem in the case $\dim X=2$. We may assume that $\omega$ is not everywhere holomorphic, otherwise we are done by \cite[Theorem 1]{Sp88}.

Denote $U=X-D$. Let $\phi: N\to E$ be the normalization map and $\varphi: N\to U$ be the natural map ($E\cap D=\emptyset$). After choosing base points $a\in N$ and $\varphi(a)\in U$, we have the following diagram 
$$
\xymatrix{
 N\ar[r]^{\varphi}\ar[d]^{\alpha_N}& U \ar[d]^{\alpha_U} \\
  A_N\ar[r]^{\psi} & A_U,}
$$ where $\alpha_N$ and $\alpha_U$ are albanese maps for the chosen base points $a$ and $\varphi(a)$, and $\psi$ is induced by universal property of albanese map of $N$. Following the construction of $A_U$, we pick a basis $\{\theta_1,\ldots\theta_q\}$ for $H^0(X, \Omega_X^1)$ and $\{\omega_1, \ldots, \omega_r\}\in H^0(X, \Omega_X^1(\log D))$ such that $\{\theta_1, \ldots, \theta_q, \omega_1, \ldots, \omega_r\}$ is a basis of $H^0(X, \Omega_X^1(\log D))$ and $\omega_1=\omega$. Also, we pick basis $\{\gamma_1, \ldots, \gamma_{2q}\}$ for $H_1(X, \mathbb{Z})$ and a basis $\{\delta_1, \ldots, \delta_r\}$ for $\ker\{ H_1(U, \mathbb{Z})\to H_1(X, \mathbb{Z})\}$. Then we have periods as a semi-lattice for $A_U$ $$\Lambda=\sum_{i=1}^{2q}\mathbb{Z}(\int_{\gamma_i}\theta_1, \ldots, \int_{\gamma_i}\omega_r)+\sum_{i=1}^r\mathbb{Z}(\int_{\delta_i}\theta_1, \ldots, \int_{\delta_i}\omega_r).$$ The the semi-albanese map $\alpha_U$ is given by $$\alpha_U(x)=[\sum_{i=1}^q(\int_{\varphi(a)}^x\theta_i)\theta^*_i+\sum_{i=1}^r(\int_{\varphi(a)}^x\omega_i)\omega_i^*]\slash\Lambda,$$ where $\theta_i^*$ and $\omega_i^*$ are the dual bases in $H^0(X, \Omega_X^1(\log D))^{\vee}$.  Now we consider the transpose of the pullback map $$\varphi^{\vee t}: H^0(N, \Omega_N^1)^{\vee}\to H^0(X, \Omega_X^1(\log D))^{\vee},$$ which induces the morphism  $\psi$. Since $\omega|_{E}=\omega_1|_{E}=0$, $\varphi^{\vee}(\omega_1)=0$. Then we get $$\varphi^{\vee t}(H^0(N, \Omega_N^1))\subset \{z_1=0\},$$ where $z_1$ is the coefficient coordinate of $\omega_1^*$. Thus we get a proper subtorus $T_0:=\psi(A_N)$ which is contained in $\{z_1=0\}/\Lambda$.  Defining $T:=A_U/T_0$, we get morphism $\beta: U\to T$.  Since we can pick a point $x\not\in E$ such that $\int_{\varphi(a)}^x\omega_1\not=0$, hence $\beta(U)$ is not constant. Also, $\alpha_U(E)=T_0$, hence $E$ is contracted by the morphism $\beta$.

Claim that $\dim\beta(U)=1$. Otherwise, we get $\beta: U\to \beta(U)$ be a generically finite morphism. Projectivize $\beta(U)$, we get a projective variety $\overline{\beta(U)}$ and rational map $U\dasharrow \overline{\beta(U)}$. Resolving the indeterminant locus, we may get a generically finite morphism $\overline{\beta}:\overline{U}\to\overline{\beta(U)}$ with $\overline{U}$ smooth and projective. %{\color{blue} and $\overline{U}-U$ a simple normal crossing divisor}. 
Note that resolving the indeterminant locus does not affect $E$, since $E\cap D=\emptyset$. Hence $E^2<0$ in $\overline{U}$ (See e.g., \cite[Theorem 10.1]{KK}). Note that $\overline{U}$ does not need to be $X$. However, $E^2$ being negative only depends on the normal bundle of $E$ in $U$. However, this contradicts the assumption $E^2\geq 0$. 

Taking the Stein factorization, we get the following commutative diagram:
$$\xymatrix{
\overline{U} \ar[rd] \ar[r]^f & \overline{ C} \ar[d]  \\
& \overline{\beta(U)}
},$$
 where $\overline{C}$ is smooth, since it is normal and of dimension one. Let $C$ be the image of the map $f|_U$. Then we have the following commutative diagram:
 $$ \xymatrix{
U \ar[r]^{f|_U} \ar[d] & C \ar[r] \ar[d] & \beta(U) \ar@{^{(}->}[d] \\ 
\Alb(U) \ar[r] & \Alb(C) \ar[r] & T }
 $$
Notice that the the holomorphic 1-form $dz_1$ over $T$ which pulls back to logarithmic 1-form $\omega$. Therefore to finish the proof, it suffices to show that $T$ is indeed isomorphic to $\Alb(C)$. 
Note that since $f|_U:U\to C$ is surjective and has connected fibers, the induced map on the first homology groups $H_1(U,\Z) \to H_1(C,\Z)$ is surjective{\color{red}(Check!!!)}. It implies that $\Alb(U) \to \Alb(C)$ is surjective. Now it is clear that all the horizontal maps in the diagram are surjective. 
%Then we have the following commutative diagram:
%$$\xymatrix{
%\overline{C} \ar[rd] \ar[r]^f & \overline{\beta (U)} \ar[d]  \\
%\Alb(\overline{C})& A(T)
%}$$
Since $\beta(E)$ is a point in $T$, so $E$ maps to a point in $\beta(U)$. Thus $E$ maps to a point in $C$. Up to choosing a base point, we get that $ T_0$ is contained in the kernel of $A_U \to A_C$, since $\alpha_U(E)=T_0$. Hence $A_C=T$, i.e., $T$ is the Albanese variety of $C$


%Hence $C$ cannot be $\mathbb{P}^1$ or $\mathbb{C}$ and the Albanese map of $C$ is injective. In fact, for positive genus case, this follows from the fact that the Albanese map of a positive genus smooth projective curve is an embedding; while for zero genus case, this is obvious. The case where $C$ being isomorphic to $\C$ is impossible as its image in $T$ has to be just one point.

%Then  $A_C=T$ implies $C$ is isomprhic to $\beta(U)$. Hence $\beta(U)$ and $\overline{\beta(U)}$ are both smooth.

%Again, $\overline{U}$ could be different from $X$. But $H^0(\overline{U},\Omega_{\overline{U}}^1(\log (\overline{U}-U)) )=H^0(X, \Omega_X^1(\log D))$. Then we are done.} 
The higher dimensional case follows immediately from Lefschetz hyperplane theorem. (TBA more details)
\end{proof}

\begin{theorem}\label{thm:Spurr-quasi-proj}
Let $(X, D)$ be a pair with $X$ a complex smooth projective variety and $D$ a simple normal crossing divisor of $X$. Let $H$ be an ample divisor on $X$. Suppose $(X, D)$ carries a nontrivial logarithmic 1-form $\omega\in H^0(X, \Omega_X^1(\log D))$  which restricts to zero on an integral divisor $E$ with $E^2\cdot H^{n-2}\geq0$, then there is a surjective morphism $f: X-D\to C$ to a smooth quasi-projective curve $C=\bar{C}-B$ ($B$ can be empty) with 

(1)  $\omega=f^*\eta$ for some $\eta\in H^0(C, \Omega_C^1(\log B))$.

(2) $f$ has connected fibers.

(3) $E$ is contained in the fiber of $f$ and $E^2\cdot H^{n-2}=0$.
\end{theorem}

\begin{proof} With out losing of generality, we may assume that $D=\sum_{i=1}^s D_i$ and $E$ intersects each component of $\{D_1, \ldots, D_t\}_{t\leq s}$ nontrivially and does not intersect other components $D':=\sum_{i=t+1}^sD_i$.
Consider the following commutative diagram of short exact sequences
$$
\xymatrix{
 0\ar[r]&\Omega_X^1(\log D')\otimes\mathcal{O}_X(-E) \ar[r]\ar[d]& \Omega_X^1(\log D')\ar[r]\ar[d] & \Omega_X^1(\log D')|_{E} \ar[d] \ar[r]&0\\
   0\ar[r]& \Omega_X^1(\log D)\otimes\mathcal{O}_X(-E)\ar[r]& \Omega_X^1(\log D) \ar[r]& \Omega_X^1(\log D)|_{E}\ar[r] &0
.  }
$$Then it induces the following commutative diagram of exact sequences
$$\label{holo-log-form-compare}
\xymatrix{
 H^0(X, \Omega_X^1(\log D')\otimes\mathcal{O}_X(-E)) \ar[r]\ar[d]& H^0(X, \Omega_X^1(\log D'))\ar[r]\ar[d] & H^0(\Omega_X^1(\log D')|_{E}) \ar[d] \\
   H^0(X, \Omega_X^1(\log D)\otimes\mathcal{O}_X(-E))\ar[r]& H^0(X, \Omega_X^1(\log D)) \ar[r]& H^0(X, \Omega_X^1(\log D)|_{E})
.  }
$$

Firstly, we want to show the left vertical arrow is an isomorphism under the assumption that $E$ intersects each component of $\{D_i\}_{i=1}^t$ nontrivially. Now consider the short exact sequence $$0\to\Omega_X^1(\log D')\to \Omega_X^1(log D)\to \nu_*\mathcal{O}_{\widetilde{D}}\to 0,$$ where $\widetilde{D}$ is a disjoint union of $D_1, \ldots, D_t$ and  $\nu: \widetilde{D}\to X$ is the natural map. Then we get the short exact sequence $$0\to\Omega_X^1(\log D')\otimes\mathcal{O}_X(-E)\to \Omega_X^1(log D)\otimes\mathcal{O}_X(-E)\to \nu_*\mathcal{O}_{\widetilde{D}}\otimes\mathcal{O}_X(-E)\to 0.$$ Thus to show left vertical arrow in diagram (\ref{holo-log-form-compare}) is an isomorphism, we just need to show that $$H^0(X, \nu_*\mathcal{O}_{\widetilde{D}}\otimes\mathcal{O}_X(-E))=0.$$ In fact, $$H^0(X, \nu_*\mathcal{O}_{\widetilde{D}}\otimes\mathcal{O}_X(-E))=H^0(\widetilde{D}, \nu^*\mathcal{O}_X(-E)).$$ Since $E$ intersects each component of $\{D_1, \ldots, D_t\}_{t\leq s}$ nontrivially, $\nu^*\mathcal{O}_X(E)=\nu^*\mathcal{O}_X(-E)^{\vee}$ is a nontrivial effective line bundle over each connected component of $\widetilde{D}$. Hence we get $H^0(\widetilde{D}, \nu^*\mathcal{O}_X(-E))=0$. 

Then by chasing in diagram (\ref{holo-log-form-compare}), the logarithmic 1-form $$\omega\in H^0(X, \Omega_X^1(\log D')).$$ Since $D'\cap E=\emptyset$ and $H^0(X, \Omega_X^1(\log D'))\subset H^0(X, \Omega_X^1(\log D))$, the whole theorem follows from Proposition \ref{Prop:Spurr-no-intersection}.
\end{proof}

\begin{proof}[Proof of Theorem \ref{thm:codim1quasiprojective}]

The theorem follows immediately from Theorem \ref{thm:Spurr-quasi-proj}.
\end{proof}












\bibliographystyle{halpha}
\bibliography{main_new}

%\begin{thebibliography}{Ara92}
%\expandafter\ifx\csname url\endcsname\relax
  %\def\url#1{\texttt{#1}}\fi
%\expandafter\ifx\csname doi\endcsname\relax
  %\def\doi#1{\burlalt{doi:#1}{http://dx.doi.org/#1}}\fi
%\expandafter\ifx\csname urlprefix\endcsname\relax\def\urlprefix{URL }\fi
%\expandafter\ifx\csname href\endcsname\relax
  %\def\href#1#2{#2}\fi
%\expandafter\ifx\csname burlalt\endcsname\relax
  %\def\burlalt#1#2{\href{#2}{#1}}\fi
%
%\bibitem[Ara92]{Ara92}
%Donu Arapura.
%\newblock Higgs line bundles, {G}reen-{L}azarsfeld sets, and maps of
  %{K}\"{a}hler manifolds to curves.
%\newblock {\em Bull. Amer. Math. Soc. (N.S.)} {\bfseries 26}(2), pp.\ 310--314,
  %1992.
%\newblock \doi{10.1090/S0273-0979-1992-00283-5}.
%
%\bibitem[CL73]{CL73}
%James~B. Carrell and David~I. Lieberman.
%\newblock Holomorphic vector fields and {K}aehler manifolds.
%\newblock {\em Invent. Math.} 21, pp.\ 303--309, 1973.
%\newblock \doi{10.1007/BF01418791}.
%
%\bibitem[GL87]{GL87}
%Mark Green and Robert Lazarsfeld.
%\newblock Deformation theory, generic vanishing theorems, and some conjectures
  %of {E}nriques, {C}atanese and {B}eauville.
%\newblock {\em Invent. Math.} {\bfseries 90}(2), pp.\ 389--407, 1987.
%\newblock \doi{10.1007/BF01388711}.
%
%\bibitem[HK05]{HK05}
%Christopher~D. Hacon and S\'{a}ndor~J. Kov\'{a}cs.
%\newblock Holomorphic one-forms on varieties of general type.
%\newblock {\em Ann. Sci. \'{E}cole Norm. Sup. (4)} {\bfseries 38}(4), pp.\
  %599--607, 2005.
%\newblock \doi{10.1016/j.ansens.2004.12.002}.
%
%\bibitem[HS19]{HS19}
%Feng Hao and Stefan Schreieder.
%\newblock Holomorphic one-forms without zeros on threefolds.
%\newblock {\em to appear Geometry $\&$ Topology} , 2019.
%\newblock \urlprefix\url{arXiv:1906.07606}.
%
%\bibitem[LZ05]{LZ05}
%Tie Luo and Qi~Zhang.
%\newblock Holomorphic forms on threefolds.
%\newblock In {\em Recent progress in arithmetic and algebraic geometry}, volume
  %386 of {\em Contemp. Math.}, pages 87--94. Amer. Math. Soc., Providence, RI,
  %2005.
%\newblock \doi{10.1090/conm/386/07219}.
%
%\bibitem[PS14]{PS14}
%Mihnea Popa and Christian Schnell.
%\newblock Kodaira dimension and zeros of holomorphic one-forms.
%\newblock {\em Ann. of Math. (2)} {\bfseries 179}(3), pp.\ 1109--1120, 2014.
%\newblock \doi{10.4007/annals.2014.179.3.6}.
%
%\bibitem[Sch19]{Sch19}
%Stefan Schreieder.
%\newblock Zeros of holomorphic one-forms and topology of k\"ahler manifolds,
  %(appendix written jointly with h.-y. lin).
%\newblock {\em to appear Geometry $\&$ Topology} , 2019.
%\newblock \urlprefix\url{arXiv:1906.07606}.
%
%\end{thebibliography}


%------------------------------------------------------------------
\end{document}
%------------------------------------------------------------------




