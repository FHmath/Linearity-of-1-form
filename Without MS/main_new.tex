\pdfoutput=1
\documentclass[11pt,reqno]{amsart}
\usepackage[letterpaper,margin=1in,footskip=0.25in]{geometry}
\usepackage{mathrsfs}
\usepackage{amssymb}
\usepackage{mathtools}
\usepackage{tikz-cd}
\usepackage{enumitem}

\PassOptionsToPackage{pdfusetitle,pagebackref,colorlinks}{hyperref}
\usepackage{bookmark}
\hypersetup{
  linkcolor={red!70!black},
  citecolor={green!70!black},
  urlcolor={blue!80!black}
}

\newtheorem{theorem}{Theorem}[section]
\newtheorem{lemma}[theorem]{Lemma}
\newtheorem{proposition}[theorem]{Proposition}
\newtheorem{corollary}[theorem]{Corollary}
\newtheorem{claim}[theorem]{Claim}
\newtheorem{conjecture}[theorem]{Conjecture}
\newtheorem{step}{Step}[subsection]
\renewcommand{\thestep}{\arabic{step}}

\newtheorem{alphtheorem}{Theorem}
\renewcommand{\thealphtheorem}{\Alph{alphtheorem}}

\theoremstyle{definition}
\newtheorem{definition}[theorem]{Definition}
\newtheorem{notation}[theorem]{Notation}

\theoremstyle{remark}
\newtheorem{remark}[theorem]{Remark}

\newtheoremstyle{cited}{.5\baselineskip\@plus.2\baselineskip\@minus.2\baselineskip}{.5\baselineskip\@plus.2\baselineskip\@minus.2\baselineskip}{\itshape}{}{\bfseries}{\bfseries .}{5pt plus 1pt minus 1pt}{\thmname{#1}\thmnumber{~#2}\thmnote{ \normalfont#3}}
\theoremstyle{cited}
\newtheorem{citedthm}[theorem]{Theorem}
\newtheorem{citedconj}[theorem]{Conjecture}
\newtheorem{citedlem}[theorem]{Lemma}
\newtheorem{citedprop}[theorem]{Proposition}

\newtheoremstyle{citeddef}{.5\baselineskip\@plus.2\baselineskip\@minus.2\baselineskip}{.5\baselineskip\@plus.2\baselineskip\@minus.2\baselineskip}{}{}{\bfseries}{\bfseries .}{5pt plus 1pt minus 1pt}{\thmname{#1}\thmnumber{~#2}\thmnote{ \normalfont#3}}
\theoremstyle{citeddef}
\newtheorem{citednot}[theorem]{Notation}


\def\be{\begin{equation}}
\def\ee{\end{equation}}

\def\bt{\begin{theorem}}
\def\et{\end{theorem}}

\def\bc{\begin{corollary}}
\def\ec{\end{corollary}}

\def\br{\begin{remark}}
\def\er{\end{remark}}

\def\bp{\begin{proposition}}
\def\ep{\end{proposition}}

\def\bl{\begin{lemma}}
\def\el{\end{lemma}}

%\def\bn{\begin{enumerate}}
%\def\en{\end{enumerate}}

\def\bex{\begin{ex}}
\def\eex{\end{ex}}

\def\bd{\begin{definition}}
\def\ed{\end{definition}}








%\DeclareMathOperator{\Pic}{Pic}                  % Pic

%\newcommand{\PP}{\mathcal{P}}           % Poincare line bunle

\DeclareMathOperator{\Supp}{Supp}                % Supp
\DeclareMathOperator{\codim}{codim}              % codim
\DeclareMathOperator{\mreg}{mreg}                % mreg
\DeclareMathOperator{\reg}{reg}                  % reg
\DeclareMathOperator{\sing}{sing}                  
\DeclareMathOperator{\id}{id}                    % id
\DeclareMathOperator{\obj}{Obj}
\DeclareMathOperator{\ad}{ad}
\DeclareMathOperator{\morph}{Morph}
\DeclareMathOperator{\enom}{End}
\DeclareMathOperator{\iso}{Iso}
\DeclareMathOperator{\Exp}{Exp}
\DeclareMathOperator{\homo}{Hom}
\DeclareMathOperator{\enmo}{End}
\DeclareMathOperator{\spec}{Spec}
\DeclareMathOperator{\fitt}{Fitt}
\DeclareMathOperator{\odr}{\Omega^\bullet_{\textrm{DR}}}
\DeclareMathOperator{\rank}{Rank}
\DeclareMathOperator{\gdeg}{gdeg}
\DeclareMathOperator{\Alb}{Alb}
\DeclareMathOperator{\alb}{alb}
\DeclareMathOperator{\Ann}{Ann}



%\DeclareMathOperator

\DeclareMathOperator{\Char}{Char}
\DeclareMathOperator{\CC}{SS}
\DeclareMathOperator{\kn}{Ker}
\DeclareMathOperator{\im}{Im}


\def\ra{\rightarrow}


\def\bone{\mathbf{1}}
\def\bC{\mathbb{C}}
\def\cM{\mathcal{M}}
\def\cV{\mathcal{V}}
\def\Def{{\rm {Def}}}
\def\cR{\mathcal{R}}
\def\om{\omega}
\def\wti{\widetilde}
\def\al{\alpha}
\def\End{{\rm {End}}}
\def\Pic{{\rm Pic}}
\def\bP{\mathbb{P}}
\def\cH{\mathcal{H}}
\def\bL{\mathbb{L}}
\def\cX{\mathcal{X}}
\def\cI{\mathcal{I}}
\def\pa{\partial}
\def\cY{\mathcal{Y}}
\def\cD{\mathcal{D}}
\def\cO{\mathcal{O}}
\def\lra{\longrightarrow}
\def\bQ{\mathbb{Q}}
\def\ol{\overline}
\def\cL{\mathcal{L}}
\def\bH{\mathbb{H}}
\def\bZ{\mathbb{Z}}
\def\bW{\mathbf{W}}
\def\bV{\mathbf{V}}
\def\bM{\mathbf{M}}
\def\eps{\epsilon}
\def\ul{\underline}
\def\lam{\lambda}
\def\sX{\mathscr{X}}
\def\bN{\mathbb{N}}

%-----------
%-----------Yagna's typsets--------
\usepackage{tikz-cd}
\usepackage{enumitem}

\PassOptionsToPackage{pdfusetitle,pagebackref,colorlinks}{hyperref}
\usepackage{bookmark}
\hypersetup{
  linkcolor={red!70!black},
  citecolor={green!70!black},
  urlcolor={blue!80!black}
}

%Mathcal Letters =====================
\newcommand{\sA}{\mathcal{A}}
\newcommand{\sB}{\mathcal{B}}
\newcommand{\sD}{\mathcal{D}}
\newcommand{\sF}{\mathcal{F}}
\newcommand{\sG}{\mathcal{G}}
\newcommand{\sH}{\mathcal{H}}
\newcommand{\sK}{\mathcal{K}}
\newcommand{\sL}{\mathcal{L}}
\newcommand{\sM}{\mathcal{M}}
\newcommand{\sN}{\mathcal{N}}
\newcommand{\sO}{\mathcal{O}}
\newcommand{\sP}{\mathcal{P}}
\newcommand{\sQ}{\mathcal{Q}}
\newcommand{\sR}{\mathcal{R}}
\newcommand{\sT}{\mathcal{T}}
\newcommand\sV{{\mathcal V}}
\newcommand\sW{{\mathcal W}}
\newcommand{\sZ}{\mathcal{Z}}

%mathbb Letters======
\newcommand{\bbA}{\mathbb{A}}
\newcommand{\bbB}{\mathbb{B}}
\newcommand{\bbC}{\mathbb{C}}
\newcommand{\bbG}{\mathbb{G}}
\newcommand{\bbH}{\mathbb{H}}
\newcommand{\bbK}{\mathbb{K}}
\newcommand{\bbL}{\mathbb{L}}
\newcommand{\bbM}{\mathbb{M}}
\newcommand{\bbN}{\mathbb{N}}
\newcommand{\bbP}{\mathbb{P}}
\newcommand{\bbQ}{\mathbb{Q}}
\newcommand{\bbR}{\mathbb{R}}
\newcommand{\bbV}{\mathbb{V}}
\newcommand{\bbZ}{\mathbb{Z}}



%Script Letters ======================
\newcommand{\frf}{\mathfrak{f}}
\newcommand{\frM}{\mathfrak{M}}

\newcommand{\scrL}{\mathscr{L}}
\newcommand{\crI}{\mathscr{I}}
\newcommand{\scrK}{\mathscr{K}}
\newcommand{\scrB}{\mathscr{B}}
\newcommand{\scrC}{\mathscr{C}}
\newcommand{\scrD}{\mathscr{D}}
\newcommand{\scrE}{\mathscr{E}}
\newcommand{\scrI}{\mathscr{I}}
\newcommand{\scrQ}{\mathscr{Q}}
\newcommand{\scrR}{\mathscr{R}}
\newcommand{\scrX}{\mathscr{X}}
\newcommand{\scrY}{\mathscr{Y}}
\newcommand{\scrF}{\mathscr{F}}
\newcommand{\Dred}{\lceil D\rceil}

%Arrow Style =========================
\newcommand{\into}{\hookrightarrow}
\newcommand{\onto}{\rightarrow\hspace*{-.14in}\rightarrow}

%Math Operators ======================



%\DeclareMathOperator{\alg}{alg}
\DeclareMathOperator{\BM}{BM}
\DeclareMathOperator{\Bs}{Bs}
\DeclareMathOperator{\Bsp}{\mathbf{B}_+}
\DeclareMathOperator{\SB}{\mathbf{B}}

\DeclareMathOperator{\coh}{coh}
\DeclareMathOperator{\Coker}{Coker}
 \renewcommand{\div}{\text{div}}

\DeclareMathOperator{\DR}{DR}
\DeclareMathOperator{\Ch}{Ch}
\DeclareMathOperator{\discrep}{discrep}
\DeclareMathOperator{\exc}{exc}

\DeclareMathOperator{\free}{free}


\DeclareMathOperator{\HHom}{\mathcal{H}\!\mathit{om}}
\DeclareMathOperator{\image}{Im}


\DeclareMathOperator{\Ind}{Ind}
\DeclareMathOperator{\Ker}{Ker}
\DeclareMathOperator{\Lie}{Lie}
\DeclareMathOperator{\op}{op}


\DeclareMathOperator{\qcoh}{qcoh}

%\DeclareMathOperator{\reg}{reg}

\DeclareMathOperator{\RHHom}{\mathbf{R}\mathcal{H}\!\mathit{om}}


\DeclareMathOperator{\Spf}{Spf}
%\DeclareMathOperator{\Supp}{Supp}
\DeclareMathOperator{\tors}{tors}
\DeclareMathOperator{\torsion}{torsion}
\DeclareMathOperator{\Var}{Var}
%\DeclareMathOperator{\Sym}{Sym}


\newcommand{\Ab}{\mathbf{Ab}}
\newcommand{\Aff}{\mathbf{Aff}}
\newcommand{\tB}{\tilde{B}}
%\newcommand{\et}{{\acute{e}t}}

\newcommand{\Sets}{\mathbf{Sets}}
%\newcommand{\cX}{\bar{X}}
\newcommand{\cP}{\bar{P}}
%\newcommand{\cV}{\bar{V}}
\newcommand{\cW}{\bar{W}}
\newcommand{\sorry}[1]{\textcolor{red}{#1}}
\newcommand{\dual}[1]{\sD^{\Omega}_{M^{#1}}}



\title{}







\date{\today}


\subjclass[2010]{}

\begin{document}
\maketitle


\section{}


Given a smooth complex irregular projective variety $X$, we are interested in the set $V(X)\subseteq H^0(X, \Omega_X^1)$ of global holomorphic 1-forms
that admit zeros on $X$. It is known that such zeros of a 1-form $\omega$ 
are supported on the support of the cohomologies of the long exact sequence
\[\sO_X\overset{\wedge\omega}{\to} \Omega_X^1 \to \cdots\to \Omega_X^{n-1}\to omega_X.\]
In particular if $\omega$ is a 1-form such the the \textit{resonance sequence}
\[H^0(X,\bbC)\overset{\wedge\omega}{\to}H^1(X,\bbC)\overset{\omega}{\to}\cdots\]
is not exact everywhere the set of zeros $Z(\omega)\neq \emptyset$.

On the other hand from the generic vanishing theory it is known that the set of such forms, call resonance forms after Dimca,
is a finite union of linear subspaces of $H^0(X,\Omega_X^1)$.

Perhaps known to the experts already, we show in particular that this linear subspace is in fact coming from a 
finite union of triple
tori in the sense of Simpson. In order to make the statement precise we need to introduce some terminologies and results. 
\begin{definition}[Resonant and (universally) nonresonant forms]\label{def:resonance}
We call a 1-form $\omega$ \emph{resonant} if the complex 
\[(H^{\bullet}(X,\bbC), \wedge\omega)\coloneqq [\ldots\to H^{i-1}(X,\C)\overset{\wedge\omega}{\longrightarrow}H^{i}(X,\C)\overset{\wedge\omega}{\longrightarrow}H^{i+1}(X,\C)\to\ldots]\]
 is not exact. Accordingly the resonance varieties of $X$ is defined as:
$$ \sR^i(X):= \{\omega\in H^1(X,\Omega_X^1)\mid  H^i  (H^*(X,\bbC), \wedge \omega) \neq 0 \} $$
and we denote by $R(X) \coloneqq \cup_i R^i(X)$.
\end{definition}


Let $\C^{\ast}$ be the multiplicative group of non-zero complex numbers.
The {\it character variety} $\Char(X)$ of $X$ is the identity component of the moduli space of rank-one $\bC$-local systems on $X$, i.e., 
\begin{center}
$\Char(X):= 
\Hom (H_1(X,\bZ)/\text{Torsion}, \bC^*)\cong (\bC^*)^{b_1(X)}.$
\end{center}

\bd The {\it $i$-th cohomology jump locus of $X$} is defined as: 
$${\sV^{i}(X)=\lbrace \rho\in \Char(X) \mid  H^{i}(X,L_{\rho})\neq 0 \rbrace},$$ 
where $L_{\rho}$ is the unique rank-one $\bC$-local system  on $X$ associated to the representation $\rho\in \Char(X) $.\ed


\begin{theorem}[Structure Theorem] Let $X$ be a smooth complex projective variety. Then the jump loci $\sV^i(X)$ is a finite union of torsion translated sub-tori of $\Char(X)$.
\end{theorem}
%\br 
%In fact, up to a torsion translate, every irreducible component of $\sV^i(X)$ is induced by a sub-Hodge structure of $H^1(X, \Q)$. To be precise, there exists an algebraic surjective map $f: \Alb(X) \to A$ from the Albanese variety of $X$ to an abelian variety $A$ such that this irreducible component is, up to a torsion translate, the image of the following map $$\Char(A) \hookrightarrow \Char(\Alb(X)) \cong \Char(X) .$$
%Here the last isomorphism is induced by the Albanese map.
%\er 




\begin{theorem} Let $X$ be a smooth complex projective variety.  Then we have the following tangent cone formula:
$$ TC_1( \sV^i(X)) = \sR^i(X)$$
under the exponential map $M_{\DR}(X) \to \Char(X)$. 
\end{theorem} 
The reason behind this theorem is the fact that the smooth projective variety is formal.  For a proof, please see \cite{S}.

Together with the structure theorem, we get the following. 

\bc Let $X$ be a smooth complex projective variety. Then every irreducible component of $\sR^i(X)$ is a linear sub-vector space of $H^1(X,\C)$.
\ec




Finally, for a subvariety $Z\subset X$ of any smooth projective variety $X$, by 
$T^*_ZX$ we denote the closure of the conormal bundle $T^*_{Z^{\reg}}X$ of the regular locus of $Z^{\reg}$ of $Z$
inside the cotangent bundle $T^*X$ of $X$. With this notation at our disposal note that when $X$ is an abelian variety
$A$, the image of $T^*_ZA\to H^0(A,\Omega_A^1)$ under the projection $\pi\colon T^*A \to H^0(A,\Omega_A^1)$ is precisely the set of 1-forms that admits a zero on the regular locus of $Z$.
We show that whether or not a 1-form $\omega\in H^0(A,\Omega_A^1)$
comes from $T^*_ZA$ for some $Z$, is determined by one of the most fundamental
 structural properties of subvarieties of abelian varieties;
 namely, whether or not
$Z$ is of general type. By a result of Ueno \cite[Theorem 10.9]{Uen75} this is equivalent to $Z$ being not fibered by any subabelian varieties of $A$. 
When $Z$ is smooth 
this seems to be a classical fact that whenever the normal bundle of $Z$ is ample all $\omega|_Z$ admits finitely many zeros on $X$
(see e.g.\ \cite[Lemma 3.1]{Deb}). This was generalised to subvarieties of simple abelian varieties by 
Hacon and Kov\'acs \cite[Proposition 3.1]{HK05}. A special case of more recent result of Popa and \cite{PS14} suggests that this is true whenever $Z$ is smooth and
$\omega_Z$ is big,
i.e.\ $Z$ is of general type. Although perhaps well-known to experts, we could not find any appropriate sources in literature. Nonetheless, our proof is inspired
by their original argument in\cite[Proposition 3.1]{HK05} and the proof of \cite[Lemma 3.1]{Deb} in Debarre's expository article

\begin{proposition}
\label{van-nonsimple}
Let $A$ be an abelian variety and $Z$ be an irreducible proper subvariety of $A$. Then the following are equivalent

(1) $Z$ is not fibred by tori. 

(2) The projection $\pi\colon T^*_ZX\to H^0(A, \Omega_A^1)$ is surjective.

(3) For general holomorphic 1-forms $\omega\in H^0(A, \Omega_A^1)$, the restricted holomorphic 1-form $\omega|_{Z_{\textnormal{reg}}}$ admit zeros at finitely many points on the smooth locus $Z_{\textnormal{reg}}$.
\end{proposition}



\section{}
Our theorem is the following
Let $X$ be an irregular smooth projective variety and $a\colon X\to \Alb(X)$ the albanese morphism. We identify $M_{\DR}^0(X)$ with $M_{\DR}^0(A)$ 
to identify
\[\sR^i(X) = \{\omega\in H^0(A, \Omega_A^1)| H^i(H^*(X, \bbC), \wedge a^*\omega)\neq 0\}\]
and rewrite the tangent cone theorem as 
\[TC_1(A, \bbR a_*\bbC) = \sR^i(X).\]



\begin{alphtheorem}
Suppose the decomposition theorem
for $a$ is given by
\[\bbR a_*\bbC = \bigoplus_{\alpha}P_{\alpha}\]
with $P_{\alpha}$ a shifted simple perverse sheaf suppored on some strata $S_{\alpha}$ of $a$, then
\[R(X) = \bigcup_i (TC_1 \sV^i(A, P_{\alpha})) = \bigcup_{\alpha}\pi(T^*_{S_{\alpha}}A) \]
\end{alphtheorem}

The first equality is a consequence of the tangent cone theorem and the fact that
$H^i(X, \bbC) = H^i(A, \bbR a_*\bbC) = \cup_{\alpha} \sV^i(A, P_{\alpha})$.
Therefore the main point of the argument is to show the second equality.

\begin{proof}
Note that if there is an $\alpha$ such that $\chi(P_{\alpha}) \neq 0$, then note that by generic vanishing theorem for
a general character $\rho \in \Char(\pi_1(A), \bbC^*)$ $\chi(P\otimes \bbC_{\rho}) = H^0(A, P\otimes \bbC_{\rho}) \neq 0$. Therefore,
$\sV^0(X, P) = \Char(\pi_1(A), \bbC^*)$. Therefore $TC_1(\sV(X, \bbR a_*\bbC)) = TC_1(\sV(A, P)) = H^0(A, \Omega_A^1) =  \sR(X)$.

On the other hand, if characteristic cycle $\Ch(P_{\alpha}) = \cup_Z T^*_ZA$ then we claim that there exists a $Z$ in the singular support of $P$ such that
$Z$ is not fibered by tori. Indeed,
by Kashiwara's index theorem 
\[\chi(A,P) = <CC(P), T^*_AA>\]
If $Z$ was fibered by a subabelian variety $B\subset A$, then $T^*_ZA = T^*_BA$. But $<T^*_BA, T^*_AA> = \chi(B, \bbC) = 0$ (see e.g.\ [Dimca, p.\ 12	4]).
We conclude by Proposition \ref{van-nonsimple} that when $Z$ is not fibered by tori $\pi(T^*_ZA) = H^0(A,\Omega_A^1)$. 

Now let $\chi(A, P) = 0$ for all $P$ appearing in the decomposition of $\bbR a_*\bbC$. By a result of Kr\"amer--Weissauer we know that there exists an abelian variety $A'$, a map $p\colon A\to A'$,
a character $\rho\in \Char(A)$ and a simple perverse sheaf $P'$ on
$A'$ with $\chi(P')>0$ and such that $p^*P'  = P\otimes \bbC_{\rho}$. Note that in this case the cohomology jump loci are given by
\[\bigcup \rho\cdot(\Char(A')\to \Char(A))\]
(see schnell: $\Phi : A^{\natural} \to \Char(A)$. $\Phi(S^k(\sM)) = S^k(\DR(\sM)$ and $S^k(\sM)$ are supported on the support of $FM(\sM)$ which 
is supported on $(L,\nabla)\otimes(A'^{\natural}\to A^{\natural})$.)
Letting $B = \ker(A\to A')$ note that the local systems corresponding to the forms $\omega\in H^0(B, \Omega_B^1)$ are no tangent to $\sV(A, P)$ and hence 
$\sR(X)\subseteq p^*H^0(A', \Omega_{A'}^1)$. On the other hand any element in the image of $T^*A' \to T^*A$
maps under $\Phi$ to a character admitting an element in $\sV(X, P)$. Therefore $p^*H^0(A', \Omega_{A'}^1)\subseteq \sR(X)$. 
On the other hand, as argued before since $\chi(A',P')>0$, the singular support of $P'$ must have a component $Z'$ that does not admit further fibration by tori.
Since $p$ is smooth $p^{-1}Z'$ must be a strata of $P$. This plus the fact that $CC(P) = \sum n_{Z}\chi(Z)$ with $n_Z\geq 0$ [see Franecki--Kapranov Corollary 1.4]
we obtain that all $Z$ must be fibered by tori. Depending on whether or not $p(Z)$ is fibered by tori, we will have that $\pi(T^*_ZA) \subseteq p^*(H^0(A',\Omega_{A'}^1)$ is a strict inequality or equality. Nonetheless we know that there exists at least one $Z$ for which this inclusion is an equality. Hence again
$\sR(X) = \bigcup_Z \pi(T^*_{Z}A)$ for $\cup Z = \Ch(P)$


\end{proof}




\begin{thebibliography}{ADMSP}



\bibitem[BWY]{BWY} N. Budur, B . Wang, Y. Yoon, {\it Rank one local systems and forms of degree one}, Int. Math. Res. Not. (2016), no.13, 3849-3855.


\bibitem[LMW]{LMW} Y. Liu, L. Maxim, B. Wang, \textit{Aspherical manifolds, Mellin transformation and a question of Bobadilla-Kollar}, arXiv:2006.09295.

\bibitem[HK05]{HK05} C. D. Hacon, S. J. Kov\'acs, \textit{Holomorphic one-forms on varieties of general type}, Ann. Sci. \'Ecole Norm. Sup. \textbf{38} (2005), no. 4, 599--607.


\bibitem[S]{S} A. Suciu, {\it Around the tangent cone theorem}, Configuration Spaces: Geometry, Topology and Representation Theory, 1-39, Springer INdAM series, vol. 14, Springer, Cham, 2016




\end{thebibliography}



%------------------------------------------------------------------
\end{document}
%------------------------------------------------------------------




